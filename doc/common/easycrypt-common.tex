% --------------------------------------------------------------------
\usepackage{amsmath}

\newcommand{\rel}[1]{\mathrel{#1}}

% --------------------------------------------------------------------
\newenvironment{tightcenter}{%
  \setlength\topsep{0pt}
  \setlength\parskip{0pt}
  \begin{center}}
{\end{center}}

% --------------------------------------------------------------------
% Acronyms, names, ...

\usepackage{xspace}

\def\EasyCrypt{\textsc{EasyCrypt}\xspace}
\def\prhl{\textsc{prhl}\xspace}
\def\phl{\textsc{phl}\xspace}
\def\hl{\textsc{hl}\xspace}

% --------------------------------------------------------------------
% Inference rules
\usepackage{mathpartir}

\newenvironment{cmathpar}
{\begin{tightcenter}\begin{mathpar}}
{\end{mathpar}\end{tightcenter}}

% --------------------------------------------------------------------
% EasyCrypt listings
\usepackage[final]{listings}

\newcommand{\ensuretext}[1]{\ensuremath{\text{#1}}}

\documentclass[a4paper,notitlepage,oneside]{book}

\usepackage[utf8]{inputenc}
\usepackage[T1]{fontenc}
\usepackage{lmodern}

\usepackage{multicol}
\usepackage[authoryear,longnamesfirst,round]{natbib}
\usepackage{multind}
\usepackage{xspace}
\usepackage{mdframed}
\usepackage[procnames]{listings}
\usepackage{amssymb,dsfont,stmaryrd}
\usepackage{infer}
\usepackage[pagebackref,colorlinks=true,linkcolor=black,linktoc=all,citecolor=blue]{hyperref}
\usepackage[usenames,dvipsnames]{xcolor}
\usepackage{tabularx}

\DeclareMathVersion{sans}
\SetSymbolFont{operators}{sans}{OT1}{cmbr}{m}{n}
\SetSymbolFont{letters}{sans}{OML}{cmbrm}{m}{it}
\SetSymbolFont{symbols}{sans}{OMS}{cmbrs}{m}{n}
\SetMathAlphabet{\mathit}{sans}{OT1}{cmbr}{m}{sl}
\SetMathAlphabet{\mathbf}{sans}{OT1}{cmbr}{bx}{n}
\SetMathAlphabet{\mathtt}{sans}{OT1}{cmtl}{m}{n}
\SetSymbolFont{largesymbols}{sans}{OMX}{iwona}{m}{n}

\makeindex{easycrypt}
\makeindex{ambient}
\makeindex{pL}

% !TeX root = easycrypt.tex
%% Misc
\newcommand{\DONE}{}% {{\color{red}DONE}}
\newcommand{\Example}{\paragraph*{Example}}
\newcommand{\Syntax}{\paragraph*{Syntax}}
\newcommand{\Description}{\paragraph*{Description}}
\setcounter{secnumdepth}{3}
\renewcommand{\thesubsubsection}{\arabic{chapter}.\arabic{section}.\arabic{subsection}.\arabic{subsubsection}}
\newbox\minicodebox
\newenvironment{minicode}[1]{%
\minipage[t]{#1\linewidth} %
\centering %
\verbatim %
}{%
\endverbatim %
\endminipage% 
}




% \newcounter{todos}\setcounter{todos}{1}
% \newcommand{\TODO}[1]{%
%   \textsf{\textcolor{red}{$^{[\thetodos]}$}}%
%   \marginpar{%
%     \framebox[1.2\marginparwidth][t]{%
%       \parbox[t]{\marginparwidth}{%
%         \raggedright\scriptsize{
%           \textcolor{red}{\thetodos: #1}%
%         }}}}%
%   \stepcounter{todos}%
% }
\newcounter{alarmcounter}
\setcounter{alarmcounter}{1}
\newcommand{\alarm}[1]
            {\begingroup
              \def\thefootnote{{\normalsize\color{red}(\arabic{footnote})}}
              \footnote{\textsf{\textbf{{\color{red}\sc \bf ALARM:}
                  #1}}}\endgroup}

% \providecommand{\note}[1]{
%  \noindent{\color{red}
%  \framebox[\textwidth][t]{%
%   \parbox[t]{0.98\textwidth}{\textcolor{red}{#1}}
% }}}

\newcommand{\infr}[2]{
{\renewcommand{\arraystretch}{1.1}
\begin{array}{c}
{#1}\\
\hline
{#2}
\end{array}}}

\providecommand{\eqref}[1]{\textup{(\ref{#1})}}
\providecommand{\eqdef}{\raisebox{-.2ex}[.2ex]{$\stackrel{\textrm{\tiny def}}{~=~}$}}
\newcommand{\gl}{\boldsymbol}
\newcommand{\game}[1]{\ensuremath{\mathsf{#1}}}
\newcommand{\tactic}{\texttt}
\newcommand{\fail}{F}
\newcommand{\bad}{\mathbf{bad}}
\newcommand{\result}{\mathsf{res}}
\newcommand{\guess}[1]{\tilde{#1}}
\newcommand{\challenge}[1]{\gl{#1^*}}
\newcommand{\uptobad}[2]{\mathsf{uptobad}({#1},{#2})}
\providecommand{\implies}{\Longrightarrow}
\newcommand{\cnt}{\mathsf{cntr}}
\newcommand{\qG}{q_\mathsf{G}}
\newcommand{\qH}{q_\mathsf{H}}
\newcommand{\qS}{q_\mathsf{S}}
\newcommand{\qD}{q_\Dec}

%% Names

\newcommand{\CertiCrypt}{\textsf{CertiCrypt}\xspace}
\newcommand{\CertiPriv}{\textsf{CertiPriv}\xspace}
\newcommand{\ProVerif}{\textsf{ProVerif}\xspace}
\newcommand{\CryptoVerif}{\textsf{CryptoVerif}\xspace}
\newcommand{\Coq}{\textsf{Coq}\xspace}
\newcommand{\CIL}{\textsf{CIL}\xspace}
\newcommand{\pWHILE}{\textsf{p}\textsc{While}\xspace}
\newcommand{\Why}{\textsf{Why}\xspace}
\newcommand{\altergo}{\textsf{alt-ergo}\xspace}
\newcommand{\simplify}{\textsf{Simplify}\xspace}

%% Security properties and schemes

\newcommand{\INDCPA}{\textsf{IND-CPA}\xspace}
\newcommand{\INDCCAone}{\textsf{IND-CCA1}\xspace}
\newcommand{\INDCCA}{\textsf{IND-CCA}\xspace}
\newcommand{\EFCMA}{\textsf{EF-CMA}\xspace}
\newcommand{\LCDH}{\ensuremath{\mathsf{LCDH}}\xspace}
\newcommand{\CDH}{\textsf{CDH}\xspace}
\newcommand{\DDH}{\textsf{DDH}\xspace}
\newcommand{\ElGamal}{\textsf{ElGamal}\xspace}
\newcommand{\HElGamal}{\textsf{HElGamal}\xspace}
\newcommand{\RSA}{\textsf{RSA}\xspace}
\newcommand{\OAEP}{\textsf{OAEP}\xspace}
\newcommand{\FDH}{\textsf{FDH}\xspace}
\newcommand{\HMAC}{\textsf{HMAC}\xspace}
\newcommand{\CS}{\textsf{CS}\xspace}
\newcommand{\Skein}{\textsf{Skein}\xspace}
\newcommand{\TCR}{\textsf{TCR}\xspace}

\newcommand{\A}{\mathcal{A}}
\newcommand{\B}{\mathcal{B}}
\newcommand{\C}{\mathcal{C}}
\newcommand{\D}{\mathcal{D}}
\newcommand{\Oracle}{\mathcal{O}}
\newcommand{\KG}{\mathcal{KG}}
\newcommand{\Enc}{\mathcal{E}}
\newcommand{\Dec}{\mathcal{D}}
\newcommand{\Sign}{\mathsf{Sign}}
\newcommand{\Verify}{\mathsf{Verify}}

%% Complexity and termination

\newcommand{\lossless}{\mathsf{lossless}}
\newcommand{\PPT}[1]{\mathsf{PPT}(#1)}
\newcommand{\PPTX}{\mathsf{PPT}}

\newcommand{\bounded}[1]{\mathsf{bounded}(#1)}

%% Sets

\newcommand{\zeroone}{[0,1]}
\newcommand{\bit}{\{0,1\}}
\newcommand{\bitstring}[1]{\ensuremath{\bit^{#1}}}
\newcommand{\zq}{\mathbb{Z}_q}
\newcommand{\bool}{\mathbb{B}}
\newcommand{\nat}{\mathbb{N}}
\newcommand{\real}{\mathbb{R}}
\newcommand{\option}[1]{#1_\bot}

%% Mathematics

\renewcommand{\Pr}[2]{\mathrm{Pr}\left[#1 : #2\right]}
\newcommand{\Prm}[3]{\mathrm{Pr}\left[#1,#3 : #2\right]}
\newcommand{\labs}{\left\lvert}
\newcommand{\rabs}{\right\rvert}
\newcommand{\charfun}{\mathds{1}}
\newcommand{\indicator}[1]{\mathbb{I}_{#1}}
\newcommand{\lub}{\sup}
\newcommand{\negl}{\mathsf{negl}}
\newcommand{\Adv}[2]{\mathbf{Adv}_{\mbox{\scriptsize $#1$}}^{\mbox{\scriptsize $#2$}}}
\newcommand{\Succ}[2]{\mathbf{Succ}_{\mbox{\scriptsize $#1$}}^{\mbox{\scriptsize $#2$}}}
\newcommand{\PER}{\mathsf{PER}}
\newcommand{\sym}{\mathsf{SYM}}

%% Distribution monad

\newcommand{\distr}{\mathcal{D}}
\newcommand{\unit}{\mathsf{unit}}
\newcommand{\bind}{\mathsf{bind}}
\newcommand{\supp}{\mathsf{support}}
\newcommand{\range}[2]{\mathsf{range}~{#1}~{#2}}
\newcommand{\lift}[3]{{#2}\,\mathcal{L}(#1)\,{#3}}

%% Semantics 

\newcommand{\sem}[1]{\llbracket #1 \rrbracket}
\newcommand{\subst}[2]{\left[{}^{#2}/{}_{#1}\right]}
\newcommand{\evalexpr}[1]{\sem{#1}_{\Expr}}
\newcommand{\evaldistr}[1]{\sem{#1}_{\DExpr}}
\newcommand{\eval}[1]{\sem{#1}}
\newcommand{\Proc}{\mathcal{P}}
\newcommand{\Var}{\mathcal{V}}
\newcommand{\Expr}{\mathcal{E}}
\newcommand{\DExpr}{\mathcal{DE}}
\newcommand{\Instr}{\mathcal{I}}
\newcommand{\Cmd}{\mathcal{C}}
\newcommand{\Mem}{\mathcal{M}}
\newcommand{\Decl}{\mathsf{decl}}
\newcommand{\Params}{\mathsf{params}}
\newcommand{\Body}{\mathsf{body}}
\newcommand{\RE}{\mathsf{re}}
\newcommand{\Global}{\mathsf{global}}
\newcommand{\Local}{\mathsf{local}}
\newcommand{\List}[1]{{#1}^*}
\newcommand{\fv}{\mathsf{fv}}
\newcommand{\modifies}{\mathsf{mod}}
\newcommand{\depends}{\mathsf{dep}}
\newcommand{\valid}{\mathsf{valid}}
\newcommand{\glob}{\mathsf{glob}}
\newcommand{\loc}{\mathsf{loc}}
\newcommand{\col}{\textsf{col}}

%% Well-formed adversaries

\newcommand{\Gadv}{\mathcal{RW}}
\newcommand{\Gcomm}{\mathcal{R}}
\newcommand{\Gorcl}{\mathcal{O}}
\newcommand{\GoodWrite}[1]{\mathsf{writable}(#1)}
\newcommand{\GoodRead}[2]{\fv(#2) \subseteq #1}
\newcommand{\GoodAdvc}[3]{#1 \vdash #2\!:\!#3}
\newcommand{\GoodAdv}[2]{#1 \vdash_{\mathrm{wf}} #2}

%% Program equivalence statements 

\newcommand{\Equiv}[4]{\models {#1} \sim {#2} : {#3} \Longrightarrow {#4}}
\newcommand{\AEquiv}[6]{\models {#2} \sim_{#5,#6} {#3} : {#1} \Longrightarrow {#4}}
\newcommand{\JAEquiv}[6]{{#2} \sim_{#5,#6} {#3} : {#1} \Longrightarrow {#4}}
\newcommand{\EquivMem}[2]{\models {#1} \equiv {#2}}
\newcommand{\EqObs}[4]{\models {#1} \simeq^{#3}_{#4} {#2}}
\newcommand{\AEqObs}[5]{\models {#1} \simeq^{#3}_{{#4}} {#2} \preceq {#5}} 
\newcommand{\ACEqObs}[7]
        {\AEqObs{\left[ #1 \right]_{#6}}{\left[ #2 \right]_{#7}}{#3}{#4}{#5}}
\newcommand{\Triple}[3]{\sem{#2} {#3} \preceq {#1}}
\newcommand{\DTriple}[3]{\sem{#2} {#3} \succeq {#1}}
\newcommand{\dequiv}[3]{{#1} \simeq_{#3} {#2}}
\newcommand{\fequiv}[3]{{#1} =_{#3} {#2}}
\newcommand{\Pre}{\Psi}
\newcommand{\Post}{\Phi}
\newcommand{\Inv}{\Phi}
\newcommand{\side}[1]{\langle #1 \rangle}
\newcommand{\sidel}{\side{1}}
\newcommand{\sider}{\side{2}}
\newcommand{\eqobsin}{\mathsf{eqobs\_in}}
\newcommand{\eqobsout}{\mathsf{eqobs\_out}}
\newcommand{\pre}{\Psi}
\newcommand{\post}{\Phi}

%% Variables

\newcommand{\LH}{\gl{L}_H}
\newcommand{\LD}{\gl{L}_\Dec}
\newcommand{\cdef}{\gl{\gamma_\mathsf{def}}}

%% Constants and operators
% TODO: Update!
\newcommand{\true}{\mathsf{true}}
\newcommand{\false}{\mathsf{false}}
\newcommand{\nil}{\mathsf{nil}}
\newcommand{\hd}{\mathsf{hd}}
\newcommand{\tl}{\mathsf{tl}}
\newcommand{\app}{\mathbin{+\mkern-7mu+}}
\newcommand{\concat}{\parallel}
\newcommand{\xor}{\oplus}
\newcommand{\msb}[2]{[#1]^{#2}}
\newcommand{\lsb}[2]{[#1]_{#2}}
\newcommand{\dom}{\mathsf{dom}}
\newcommand{\ran}{\mathsf{ran}}
\newcommand{\fst}{\mathsf{fst}}
\newcommand{\snd}{\mathsf{snd}}
\newcommand{\some}[1]{#1}
\newcommand{\none}{\bot}

%% Language
% TODO: Update!
\newcommand{\Skip}{\mathsf{skip}}
\newcommand{\Seq}[2]{#1;\ #2}
\newcommand{\Ass}[2]{#1 \leftarrow #2}
\newcommand{\Rand}[2]{#1 \stackrel{\raisebox{-.25ex}[.25ex]%
{\tiny $\mathdollar$}}{\raisebox{-.2ex}[.2ex]{$\leftarrow$}} #2}
\newcommand{\Randi}[2]{\Rand{#1}{[0..#2]}}
\newcommand{\Randb}[1]{\Rand{#1}{\bit}}
\newcommand{\Randbs}[2]{\Rand{#1}{\bitstring{#2}}}
\newcommand{\Cond}[3]{\mathsf{if}\ #1\ \mathsf{then}\ #2\ \mathsf{else}\ #3}
\newcommand{\Condt}[2]{\mathsf{if}\ #1\ \mathsf{then}\ #2}
\newcommand{\Else}{\mathsf{else}\ }
\newcommand{\Elsif}{\mathsf{elsif}\ }
\newcommand{\nWhile}[3]{\mathsf{while}_{#1}\ #2\ \mathsf{do}\ #3}
\newcommand{\While}[2]{\mathsf{while}\ #1\ \mathsf{do}\ #2}
\newcommand{\Call}[3]{#1 \leftarrow #2\mathsf{(}#3\mathsf{)}}
\newcommand{\Return}{\mathsf{return}}
\newcommand{\Assert}[1]{\mathsf{assert}~#1}

%% Language definition
\lstnewenvironment{easycrypt}[2][]%
  {\lstset{language=easycrypt,caption=#2,#1}}%
  {}

\newcommand{\rawec}[2][]{\lstinline[language=easycrypt,#1]{#2}}
\newcommand{\ec}[2][]{\lstinline[language=easycrypt,style=easycrypt-pretty,#1]{#2}}

\lstdefinelanguage{easycrypt}{
  style=easycrypt-default,
  procnamekeys={op,pred,fun},
  procnamestyle={\sffamily\itshape},
  morekeywords=[1]{theory,end},
  morekeywords=[2]{type,op,axiom,lemma,module,pred,lambda},
  morekeywords=[3]{var,fun},
  morekeywords=[4]{while,if},
  morekeywords=[5]{bool,int,real,bitstring,array,list,matrix,word},
  morekeywords=[6]{forall,exists},
%  moredirectives={prover,print}, % Incomplete
  morecomment=[n][\itseries]{(*}{*)},
  morecomment=[n][\bfseries]{(**}{*)}
}

\lstdefinestyle{easycrypt-default}{
  columns=fullflexible,
  captionpos=b,
  frame=tb,
  xleftmargin=.1\textwidth,
  xrightmargin=.1\textwidth,
  basicstyle=\small\sffamily,
  identifierstyle={},
  keywordstyle=[1]{\bfseries},
  keywordstyle=[2]{\bfseries},
  keywordstyle=[3]{\bfseries},
  keywordstyle=[4]{\bfseries},
  keywordstyle=[5]{\itshape},
  keywordstyle=[6]{\bfseries}
}

\lstdefinestyle{easycrypt-pretty}{
    basicstyle=\small\sffamily,
    literate={:=}{{$\mathrel{\gets}$}}1
              {<=}{{$\binop{\leq}$}}1
              {>=}{{$\geq$}}1
              {=\$}{{$\stackrel{\$}{\gets}$}}1
              {forall}{{$\forall$}}1
              {exists}{{$\exists$}}1
              {->}{{$\rightarrow\;$}}1
              {=>}{{$\Rightarrow\;$}}1
              {==>}{{$\Rrightarrow\;$}}1
              {\/\\}{{$\wedge$}}1
              {\\\/}{{$\vee$}}1
              {.\[}{{[}}1
              {'a}{{$\alpha\,$}}1
              {'b}{{$\beta\,$}}1
              {'c}{{$\gamma\,$}}1
              {'x}{{$\chi\,$}}1
              {lambda}{{$\lambda\,$}}1
}

%% Typesetting
\newcommand{\titledbox}[4]{{\color{#1}\fbox{\begin{minipage}{#2}{\textbf{#3:} \color{black}#4}\end{minipage}}}}
\newcommand{\warningbox}[1]{\titledbox{red}{.9\textwidth}{Warning}{#1}}

%%% Local Variables: 
%%% mode: latex
%%% TeX-master: "easycrypt"
%%% End: 



\lstset{numberbychapter=false} %% Set to true if chapter numbering is desired


\title{\EasyCrypt Manual}
\date{Version \ECversion{} --- \today}
\author{The \EasyCrypt Team}

\begin{document}

\maketitle
\thispagestyle{empty}

\tableofcontents

\part{User Manual}
% Getting Started: Installation and Basic Usage
% !TeX root = easycrypt.tex

\chapter{Getting Started}
\section{Installation}

\section{Basic Example (Tutorial)}

%%% Local Variables: 
%%% mode: latex
%%% TeX-master: "easycrypt"
%%% End: 

% Theories: Types and Operators, Modules, and Cloning
% !TeX root = easycrypt.tex

\chapter{Theories (Fran\c{c}ois + Gilles)\label{chap:theories}}

Definitions and lemmas can be grouped in theories, that can be imported to
provide new language functionalities as required for a particular proof.
Currently, \EC supports user-defined types and operators
(Section~\ref{sec:types}) and user-defined distributions
(Section~\ref{sec:distributions}). In addition, theories can declare modules and
functors (Section~\ref{sec:modules}), used to represent schemes, oracles and
experiments, as well as abstract types for such modules, which can be used to
modularize proofs or represent abstract adversaries. Finally, theories can be
cloned and refined (Section~\ref{sec:cloning}), which allows the user to
consider small variants of a theory without having to create a new one from
scratch, or to consider concrete implementation details only when necessary for
the proof.

We illustrate each of the basic concepts with commented excerpts from the
standard library distributed with \EC. The standard library is documented
further in Chapter~\ref{chap:libraries}.

\section{Types and Operators\label{sec:types}}

At the core of any \EC specification is a set of types and functional operators
on those types. Types are \emph{non-empty} sets of values, and operators are
\emph{mathematical} functions between them.

\warningbox{Currently, the easiest way to define types and operators is to
declare them abstractly and specify them using axioms. It is \emph{very}
important for the consistency of the logical context to remember that types are
\emph{always} assumed to be non-empty, and that operators are functionals
(total, deterministic and side-effect free) when writing such definitions.}

\subsection{Basic Usage}
The following \EC code declares an abstract type of polymorphic arrays, equipped
with a length operator (whose result is never negative), an indexing operator
and extensional equality.

\begin{easycrypt}[label={lst:arrays}]{[Polymorphic arrays]A type for polymorphic arrays}
type 'a array.

op length: 'x array -> int.
axiom length_pos: forall (xs:'x array), 0 <= length xs.

op "_.[_]": 'x array -> int -> 'x.

pred (==)(xs0:'x array, xs1:'x array) =
  length xs0 = length xs1 /\
  forall i, 0 <= i => i < length xs0 => xs0.[i] = xs1.[i].

axiom extensionality: forall (xs0 xs1:'x array),
  xs0 == xs1 => xs0 = xs1.
\end{easycrypt}

The language of operators and predicates is functional in style, and
comma-separated lists of parameters are in fact currified, in this context, to
yield functional symbols that can be partially applied. (For example, the infix
extensional equality predicate \ec{(==)} defined in Listing~\ref{lst:arrays} has
type \ec{'x array -> 'x array -> bool}.)

Special syntax is used to introduce two hard-coded mixfix operators (generally
considered to correspond to set and get operations on various types). The
``\rawec[literate={\_}{{$\cdot$}}1]{_.[_]}'' operator (named \emph{get}
throughout this manual) is hard-coded as a binary operator. The
``\rawec[literate={_}{{$\cdot$}}1]{_.[_<-_]}'' operator (not yet encountered,
and named \emph{set} in the rest of this manual) is hard-coded as a ternary
operator. In addition, a constant operator ``\rawec{[]}'' can be defined, and
will often be referred to as the \emph{empty} constant, depending on its type
and the current scope.

Infix operators are declared between parentheses, and can be used either in
infix syntax using the symbol itself (for example, \rawec{x == y}), or in prefix
syntax using their parenthesized form (for example, \rawec{(==) x y}).

As illustrated in the definition of the extensional equality predicate, type
annotations are optional when types can be inferred.

\warningbox{One could expect the type annotations in the \rawec{extensionality}
axiom to be optional. However, they are currently required so that the type
variable can be instantiated.}

\subsection{Higher-Order Operators}
Operators can be higher-order. For example, the code sample shown in
Listing~\ref{lst:init_arrays} illustrates how one can axiomatically specify an
operator which creates a fresh array and initializes its elements with
index-dependent values. We illustrate its usage by defining an operator
\ec{init} that, given an integer $n$, produces the array containing integers $0$
through $n$, in ascending order.

\begin{easycrypt}[label={lst:init_arrays}]{[Index-dependent array initializer]An operator initializing an array with index-dependent values}
op init: (int -> 'x) -> int -> 'x array.

axiom init_length: forall (f:int -> 'x) l,
  0 <= l => length (init f l) = l.

axiom init_get: forall (f:int -> 'x) l i,
  0 <= l => 0 <= i => i < l =>
  (init f l).[i] = f i.

op first_ints n = init (lambda i, i) n.
\end{easycrypt}

The \rawec{lambda} notation works as expected. The argument types can be
specified where necessary. Equality on lambda terms is extensional.

\section{Describing Distributions\label{sec:distributions}}

\EC offers support for user-defined distributions. Distributions are defined by
formally defining their density function, represented by the built-in operator
\rawec{op mu: 'a distr -> ('a -> bool) -> real.} For example, the uniform
distribution on booleans is defined, in the standard library, as displayed in
Listing~\ref{lst:dbool}, where \rawec{caract P x} is \rawec{1\%r} if \rawec{P}
holds on \rawec{x} and \rawec{0\%r} otherwise.

\begin{easycrypt}[label={lst:dbool}]{[Uniform Boolean distribution]Defining the uniform distribution on booleans}
op dbool: bool distr.

axiom mu_def: forall (P:bool -> bool), 
  mu dbool P =
    (1%r/2%r) * caract P true +
    (1%r/2%r) * caract P false.
\end{easycrypt}

It may not always be this easy to define distributions. More examples of
user-provided definitions, and some distribution transformers that make it
possible to define new distributions from existing ones, can be found in the
standard library (Chapter~\ref{chap:libraries}).

\warningbox{Distributions are defined axiomatically, and no well-formedness checks are performed.}

\section{Modules and Functors\label{sec:modules}}

So far, we have only considered features whose goal is to extend the language of
expressions and the semantic domain of values. Specifications of schemes,
oracles, assumptions and game-based security properties all use modules (and
module signatures) and functors, which we now discuss, using a simple modular
definition of a random oracle from bitstrings to bitstrings as an example.

We first formally define the functionalities a random oracle is expected to
provide, as a \emph{module type}, or \emph{signature}
(Listing~\ref{lst:modulesig}). Any module implementation \rawec{M} that provides
\emph{at least} the functions from a module type \rawec{Mt} is said to be of
type \rawec{Mt} (denoted \rawec{M :> Mt}).

\begin{easycrypt}[label={lst:modulesig}]{[A signature for random oracles]A signature for random oracles from bitstrings to bitstrings}
module type RO = {
  fun init():unit;
  fun h(x:bitstring):bitstring; }.
\end{easycrypt}

Such a module signature can then be given various realizations. For example, Listing~\ref{lst:modules} shows two possible realizations of a random oracle, both of which assume a positive integer constant \rawec{qH}, used to bound the number of calls to the oracle, and two positive integer constants \rawec{inLen} and \rawec{outLen} representing the input and output lengths of the random oracles.

%% The following is probably the single most disgusting thing I've ever typeset in Latex... and I've done ugly things.
\begin{minipage}{\textwidth}
\hrule
\begin{multicols}{2}
\begin{easycrypt}[frame=none,xleftmargin=0pt,xrightmargin=0pt]{}
module RO_L: RO = {
  var cG: int
  var mG: (bitstring,bitstring) map

  fun init() = {
    mG = empty;
    cG = 0;
  }

  fun h(x:bitstring) = {
    var r:bitstring;
    var res:bitstring = empty;
    if (length x = inLen && cG < qG)
    {
      cG = cG + 1;
      r = $dbitstring(outLen);
      if (!mem(x,dom mG)) mG[x] = r;
      res = mG[x];
    }
    return res;
  }
}.
\end{easycrypt}
\columnbreak
\begin{easycrypt}[frame=none,xleftmargin=0pt,xrightmargin=0pt]{}
module RO_E: RO = {
  var tape: bitstring list
  var mG: (bitstring,bitstring) map

  fun init() = {
    var r:bitstring;
    tape = [];
    mG = empty;
    while (length tape < qG)
    {
      r = $dbitstring(outLen);
      tape = r :: tape;
    }
  }

  fun h(x:bitstring) = {
    var r:bitstring;
    var res:bitstring = empty;
    if (length x = inLen &&
       length tape <> 0)
    {
      r = hd tape;
      if (!mem(x,dom mG))
        mG[x] = r;
      res = mG[x];
      tape = tl tape;
    }
    return res;
  }
}.
\end{easycrypt}
\end{multicols}
\hrule
\begin{easycrypt}[frame=non,xleftmargin=0pt,xrightmargin=0pt,label={lst:modules}]{[Two random oracles]Two possible implementations of module type \rawec{RO}}
\end{easycrypt}
\end{minipage}

\section{Lemmas and Judgements}

\subsection{Ambient Lemmas}

\subsection{(Possibilistic) Hoare Judgements}

\subsection{Probabilistic Hoare Judgements}

\subsection{Relational Hoare Judgements}

\section{Requiring, Cloning and Realizing Theories\label{sec:cloning}}

%%% Local Variables: %%% mode: latex %%% TeX-master: "easycrypt" %%% End:
% Tactics: First-Order and pRHL Tactics
% --------------------------------------------------------------------
% --------------------------------------------------------------------
\begin{tactic}{admit}
\end{tactic}

% --------------------------------------------------------------------
\begin{tactic}{algebra}
  \begin{tsyntax}[empty]{algebra}
  \fix{Missing description of algebra}.
  \end{tsyntax}
\end{tactic}

% --------------------------------------------------------------------
\begin{tactic}{alias}
\end{tactic}

% --------------------------------------------------------------------
\begin{tactic}{apply}
\end{tactic}

% --------------------------------------------------------------------
\begin{tactic}{assumption}
  \begin{tsyntax}[empty]{assumption}
  Search in the context for a hypothesis that is convertible to the goal
  and apply. Fail if none can be found.
  \end{tsyntax}
\end{tactic}

% --------------------------------------------------------------------
\begin{tactic}{auto}
  \begin{tsyntax}[empty]{auto}
  \fix{Missing description of auto}.
  \end{tsyntax}
\end{tactic}

% --------------------------------------------------------------------
\begin{tactic}{beta}
\end{tactic}

% --------------------------------------------------------------------
\begin{tactic}{byequiv}

  \begin{tsyntax}{byequiv [option]? <specification>}
  Derives probability relation from \prhl judgements. 
  Only applies to judgments on procedures.
 
  \textbf{Examples:}
  \begin{mathpar}
    \inferrule*[left=(\prhl),rightskip=5em]%%
    { \pRHL{P}{f_1}{f_2}{Q} \\%
      P~\vec{a}_1~m_1~\vec{a}_2~m_2 \\%
      Q \Rightarrow E_1\{1\}  \Leftrightarrow E_2\{2\} }%%
    { \PR{f_1}{\vec{a}_1}{\mem{m_1}}{E_1} = \PR{f_2}{\vec{a}_2}{m_2}{E_2} }%%
    \quad\raisebox{.7em}{\tct{byequiv (: P ==> Q)} } \\
    \inferrule*[left=(\prhl),rightskip=5em]%%
    { \pRHL{P}{f_1}{f_2}{Q} \\% 
      P~\vec{a}_1~m_1~\vec{a}_2~m_2 \\%
      Q \Rightarrow E_1\{1\}  \Rightarrow E_2\{2\} }%%
    { \PR{f_1}{\vec{a}_1}{\mem{m_1}}{E_1} \leq \PR{f_2}{\vec{a}_2}{m_2}{E_2} }%%
    \quad\raisebox{.7em}{\tct{byequiv (: P ==> Q)} } \\
    \inferrule*[left=(\prhl),rightskip=5em]%%
    { \pRHL{P}{f_1}{f_2}{Q} \\%
      P~\vec{a}_1~m_1~\vec{a}_2~m_2 \\%
      Q \Rightarrow E_2\{2\}  \Rightarrow E_1\{1\} } %%
    { \PR{f_1}{\vec{a}_1}{\mem{m_1}}{E_1} \geq \PR{f_2}{\vec{a}_2}{m_2}{E_2} }%%
    \raisebox{.7em}{\tct{byequiv (: P ==> Q)} } 
  \end{mathpar}
 
 \end{tsyntax}

  Possible options are \tct{-eq} or \tct{eq}.
  Any one of the specification places can be filled
  with a wildcard \tct{_}. It that case the corresponding argument 
  is automatically inferred. Some time the infered postcondition  
  is stronger than necessary, in that case use the option \tct{-eq}.

  \fix{Missing description of byequiv for upto}.
  
  \begin{tsyntax}{byequiv <lemma>}
  Same as \tct{byequiv <specification>}, but the specification to use is 
  inferred from the lemma provided. Raises an error if the lemma does 
  not refer to the expected procedures. All variants of \tct{byequiv} 
  may take lemmas in place of explicit specifications with the same effect.
  \end{tsyntax}


\end{tactic}

% --------------------------------------------------------------------
\begin{tactic}{byphoare}
  \begin{tsyntax}{byphoare [option]? <spec>}
  Derives a probability relation from a \phl judgement on the
  procedure involved. \tct{<spec>} can include wildcards when the
  tactic should infer the pre or postcondition.

  \textbf{Options:} By default, (\tct{eq} option) specification
  inference attempts to infer a conjunction of equalities sufficient
  to imply the desired relation. Passing the \tct{-eq} option
  overrides this behaviour, instead using the trivial relation on
  events.

  \textbf{Examples:}
  \begin{mathpar}
    \inferrule%%
      {\pHL{P}{f}{Q}{=}{\delta} \\%
       \Pred{P}{m[\Arg\mapsto\vec{a}]} \\%
       \forall \mem{m'}.\,\Pred{Q}{m'} \Leftrightarrow \Pred{E}{m'}}%%
      {\PR{f}{\vec{a}}{\mem{m}}{E} = \delta}%%
      \quad\mbox{\parbox{200pt}{\tct{byphoare (_: P ==> Q)}}} \\
  \end{mathpar}
  \end{tsyntax}

  \begin{tsyntax}{byphoare <lemma>}
  Same as \tct{byphoare <spec>}, but the specification to use is
  inferred from the lemma provided. Raises an error if the lemma does
  not refer to the expected procedure. Inference options have no
  effect in this setting.
  \end{tsyntax}
\end{tactic}

% --------------------------------------------------------------------
\begin{tactic}{bypr}
  \begin{tsyntax}{bypr}
  Derives a program judgment from a probability relation or an exact
  probability. Only applies to judgments on procedures.

  \textbf{Examples:}
  \begin{mathpar}
    \inferrule*[left=(\prhl),rightskip=10em]%%
    {\forall m_1, m_2, a.\, E_1 = a \Rightarrow E_2 = a \Rightarrow Q~m_1~ m_2 \\%
     \forall \vec{a}_1, \vec{a}_2, m_1, m_2, a.\, P~\vec{a}_1~m_1~\vec{a}_2~m_2 \Rightarrow%
       \PR{f_1}{\vec{a}_1}{\mem{m_1}}{a = E_1} = \PR{f_2}{\vec{a}_2}{\mem{m_2}}{a = E_2}} %%
    {\pRHL{P}{f_1}{f_2}{Q}}%%
    \quad\raisebox{.7em}{\tct{bypr (E$_1$) (E$_2$)}} \\
  \inferrule*[left=(\phl),rightskip=10em]%%
    {\forall m, \vec{a}.\,P~\vec{a}~m \Rightarrow \PR{f}{\vec{a}}{m}{E} \mathrel{\diamond} \delta\{m\}}%%
    {\pHL{P}{f}{E}{\diamond}{\delta}}
    \quad\raisebox{.7em}{\tct{bypr}} \\
  \inferrule*[left=(\hl),rightskip=10em]%%
    {\forall m, \vec{a}.\,P~\vec{a}~m \Rightarrow \PR{f}{\vec{a}}{m}{\neg E} \mathop{=}0$\%$r}%%
    {\HL{P}{f}{E}}
    \quad\raisebox{.7em}{\tct{bypr}} \\
  \end{mathpar}
  \end{tsyntax}
\end{tactic}

% --------------------------------------------------------------------
\begin{tactic}{by}
\end{tactic}

% --------------------------------------------------------------------
\begin{tactic}{call}
  All variants of the \tct{call} tactic implicitly make use of a frame
  rule, based on a ``may modify'' analysis.

  \begin{tsyntax}{call (_: P ==> Q)}
  Compute the precondition of a procedure call using the given
  specification for the procedure. As a side-goal, prove that the
  procedure fulfills the given specification.

  As with other tactics, the specification \tct{(_: P ==> Q)} can be
  replaced with a lemma from which the specification is inferred.
  \end{tsyntax}

  \begin{tsyntax}{call (_: I)}
  Uses invariant \tct{I} to infer a specification for use with
  \tct{tactic}.
  %%
  In \prhl, equivalent to
  \tct{call (_: =$\{\Arg\}$ /\\ I ==> =$\{\Res\}$ /\\ I); first proc I.}
  %%
  In \phl and \hl, equivalent to
  \tct{call (_: I ==> I); first proc I.}
  \end{tsyntax}

  \begin{tsyntax}{call (_: B, I)}
  On \prhl abstract procedures only.
  Equivalent to \tct{call (_: $\neg$B /\\ =$\{\Arg\}$ /\\ I ==> $\neg$B => =$\{\Res\}$ /\\ I); first proc B I.}
  \end{tsyntax}

  \begin{tsyntax}{call (_: B, I, I')}
  On \prhl abstract procedures only.
  Equivalent to \tct{call (_: $\neg$B /\\ =$\{\Arg\}$ /\\ I ==> if $\neg$B then =$\{\Res\}$ /\\ I else I')); first proc B I I'.}
  \end{tsyntax}

  \textbf{Note:} When using the invariant-based variants of
  \tct{call}, error messages may be originating from the underlying
  application of \rtactic{proc}. In particular, when using them to
  deal with abstract procedure calls, the invariant \emph{should not}
  refer to memory locations the abstract procedure may modify.
\end{tactic}

% --------------------------------------------------------------------
\begin{tactic}[case $\;\phi$]{case}
  \begin{tsyntax}[empty]{case}
  Do an excluded-middle case analysis on $\phi$, substituting $\phi$
  in the goal.
  \end{tsyntax}

  \fixme{Describe the behaviour of \ec{case} on inductives.}
\end{tactic}

% --------------------------------------------------------------------
\begin{tactic}{cfold}
\end{tactic}

% --------------------------------------------------------------------
\begin{tactic}{change}
  \begin{tsyntax}[empty]{change}
  \fix{Missing description of change}.
  \end{tsyntax}
\end{tactic}

% --------------------------------------------------------------------
\begin{tactic}[clear $\;x_1 \cdots x_n$]{clear}
  \begin{tsyntax}[empty]{clear}
  Clear the local variables and hypotheses $x_1 \cdots x_n$ from the
  local context. Fail if any remaining hypotheses depend on any of the
  $x_i$.
  \end{tsyntax}
\end{tactic}

% --------------------------------------------------------------------
\begin{tactic}{congr}
  \begin{tsyntax}[empty]{congr}
  Replace a goal of the form \ec{f t$_1$ ... t$_n$ = f u$_1$ ... u$_n$}
  with the subgoals \ec{t$_i$ = u$_i$} for all \ec{$i$}. Subgoals solvable
  by \ec{reflexivity} are automatically closed.
  \end{tsyntax}
\end{tactic}

% --------------------------------------------------------------------
\begin{tactic}{conseq}
  \begin{tsyntax}{conseq <specification>}
  Rule of consequence. Proves a specification by weakening of a
  stronger result. Any one of the specification places can be filled
  with a wildcard \tct{_} to keep the value it contains in the current
  goal and trivially discharge the corresponding subgoal.

  \textbf{Examples:} In the following, $\leq^\uparrow$ (resp. $=^\uparrow$,
  $\geq^\uparrow$) is $\Leftarrow$ (resp. $\Leftrightarrow$ and
  $\Rightarrow$).
  \begin{mathpar}
  \inferrule*[left=(pRHL),rightskip=10em]%%
    {P' \Rightarrow P \\%
     Q \Rightarrow Q' \\%
     \pRHL{P}{c}{c'}{Q}}%%
    {\pRHL{P'}{c}{c'}{Q'}}%%
    \quad\raisebox{.7em}{\tct{conseq (_: P ==> Q)}} \\
  \inferrule*[left=(pHL),rightskip=10em]%%
    {P' \Rightarrow \delta \mathrel{\diamond} \delta' \\%
     P' \Rightarrow P \\%
     Q \mathrel{\diamond^\uparrow} Q' \\%
     \pHL{P}{c}{Q}{\diamond}{\delta}}%%
    {\pHL{P'}{c}{Q'}{\diamond}{\delta'}}%%
    \quad\raisebox{.7em}{\tct{conseq (_: P ==> Q: $\delta$)}} \\
  \inferrule*[left=(HL),rightskip=10em]%%
    {P' \Rightarrow P \\%
     Q \Rightarrow Q' \\%
     \HL{P}{c}{Q}}%%
    {\HL{P'}{c}{Q'}}%%
    \quad\raisebox{.7em}{\tct{conseq (_: P ==> Q)}} \\
  \end{mathpar}
  \end{tsyntax}

  \begin{tsyntax}{conseq* <specification>}
  Same as \tct{conseq <specification>}, but the subgoal corresponding
  to the postcondition is refined by a ``may modify'' analysis.
  \end{tsyntax}
\end{tactic}

% --------------------------------------------------------------------
\begin{tactic}[cut $\;\iota$: $\;\phi$]{cut}
  Same as \rtactic{have}.
\end{tactic}

% --------------------------------------------------------------------
\begin{tactic}{delta}
  \begin{tsyntax}[empty]{delta}
  \fix{Missing description of delta}.
  \end{tsyntax}
\end{tactic}

% --------------------------------------------------------------------
\begin{tactic}{done}
  \begin{tsyntax}[empty]{done}
  \fix{Missing description of done}.
  \end{tsyntax}
\end{tactic}

% --------------------------------------------------------------------
\begin{tactic}{eager}
  \begin{tsyntax}[empty]{eager}
  \fix{Missing description of eager}.
  \end{tsyntax}
\end{tactic}

% --------------------------------------------------------------------
\begin{tactic}{elim}
\end{tactic}

% --------------------------------------------------------------------
\begin{tactic}{exact}
  \begin{tsyntax}[empty]{exact}
  \fix{Missing description of exact}.
  \end{tsyntax}
\end{tactic}

% --------------------------------------------------------------------
\begin{tactic}{exfalso}
\end{tactic}

% --------------------------------------------------------------------
\begin{tactic}{fel}
  \begin{tsyntax}[empty]{fel}
  \fix{Missing description of fel}.
  \end{tsyntax}
\end{tactic}

% --------------------------------------------------------------------
\begin{tactic}{fieldeq}
\end{tactic}

% --------------------------------------------------------------------
\begin{tactic}{fission}
\end{tactic}

% --------------------------------------------------------------------
\begin{tactic}{fusion}
\end{tactic}

% --------------------------------------------------------------------
\begin{tactic}{generalize}
\end{tactic}

% --------------------------------------------------------------------
\begin{tactic}{idtac}
\end{tactic}

% --------------------------------------------------------------------
\begin{tactic}{inline}
\end{tactic}

% --------------------------------------------------------------------
\begin{tactic}{intros}
\end{tactic}

% --------------------------------------------------------------------
\begin{tactic}{iota}
  \begin{tsyntax}[empty]{iota}
  \fix{Missing description of iota}.
  \end{tsyntax}
\end{tactic}

% --------------------------------------------------------------------
\begin{tactic}{kill}
  \begin{tsyntax}[empty]{kill}
  \fix{Missing description of kill}.
  \end{tsyntax}
\end{tactic}

% --------------------------------------------------------------------
\begin{tactic}{left}
  \begin{tsyntax}[empty]{left}
  Reduce a disjunctive goal to its left member.
  \end{tsyntax}
\end{tactic}

% --------------------------------------------------------------------
\begin{tactic}{logic}
\end{tactic}

% --------------------------------------------------------------------
\begin{tactic}{modpath}
  \begin{tsyntax}[empty]{modpath}
  \fix{Missing description of modpath}.
  \end{tsyntax}
\end{tactic}

% --------------------------------------------------------------------
\begin{tactic}[move | move: $\;\pi_1 \cdots \pi_n$]{move}
  \begin{tsyntax}{move}
     Does nothing, equivalent to \rtactic{idtac}. This form is mainly
     used in conjonction with an introduction pattern (see
     Section~\ref{s:intro-pattern}), e.g. \ls!move=> $\iota_1 \cdots \iota_n$!.
  \end{tsyntax}

  \begin{tsyntax}{move: $\;\pi_1 \cdots \pi_n$}
    Generalize the patterns $\pi_1, \cdots, \pi_n$, starting from
    $\pi_n$ and going back.
    %See Section~\ref{s:gen-pattern} for more
    %information on the generalization mechanism.
  \end{tsyntax}
\end{tactic}

% --------------------------------------------------------------------
\begin{tactic}{pose}
  \begin{tsyntax}[empty]{pose}
  \fix{Missing description of pose}.
  \end{tsyntax}
\end{tactic}

% --------------------------------------------------------------------
\begin{tactic}{pr\_bounded}
  \begin{tsyntax}[empty]{pr\_bounded}
  \fix{Missing description of pr\_bounded}.
  \end{tsyntax}
\end{tactic}

% --------------------------------------------------------------------
\begin{tactic}{progress}
\end{tactic}

% --------------------------------------------------------------------
\begin{tactic}{rcondf}
  \begin{tsyntax}[empty]{rcondf}
  \fix{Missing description of rcondf}.
  \end{tsyntax}
\end{tactic}

% --------------------------------------------------------------------
\begin{tactic}{rcondt}
  \begin{tsyntax}[empty]{rcondt}
  \fix{Missing description of rcondt}.
  \end{tsyntax}
\end{tactic}

% --------------------------------------------------------------------
\begin{tactic}{reflexivity}
\end{tactic}

% --------------------------------------------------------------------
\begin{tactic}{rewrite}
\end{tactic}

% --------------------------------------------------------------------
\begin{tactic}{right}
\end{tactic}

% --------------------------------------------------------------------
\begin{tactic}{ringeq}
  \begin{tsyntax}[empty]{ringeq}
  \fix{Missing description of ringeq}.
  \end{tsyntax}
\end{tactic}

% --------------------------------------------------------------------
\begin{tactic}{rnd}
\end{tactic}

% --------------------------------------------------------------------
\begin{tactic}{rwnormal}
  \begin{tsyntax}[empty]{rwnormal}
  \fix{Missing description of rwnormal}.
  \end{tsyntax}
\end{tactic}

% --------------------------------------------------------------------
\begin{tactic}{seq}
  Rules for sequences:

  \begin{tsyntax}{seq p1 p2 : R}
  \begin{mathpar}
  \inferrule*[left=(\prhl),rightskip=10em]%%
    {|c_1| = \tct{p1}\\%%
     |c_2| = \tct{p2}\\%%
     \pRHL{P}{c_1}{c_2}{R}\\%%
     \pRHL{R}{c_1'}{c_2'}{Q}}%%
    {\pRHL{P}{c_1;c_1'}{c_2;c_2'}{Q}}%%
    \quad\raisebox{.7em}{\tct{seq p1 p2 : R}}\\%%
  \end{mathpar}
  \end{tsyntax}

  \begin{tsyntax}{seq p : R}
  \begin{mathpar}
  \inferrule*[left=(\hl),rightskip=10em]%%
    { |c| = \tct{p}\\%%
     \HL{P}{c}{R}\\%%
     \HL{R}{c'}{Q} }%%
    {\HL{P}{c;c'}{Q}}%%
    \quad\raisebox{.7em}{\tct{seq p : R}}\\%%
  \end{mathpar}
  \end{tsyntax}


  \fix{Missing description of seq for phl}.

\end{tactic}

% --------------------------------------------------------------------
\begin{tactic}{simplify}
  \begin{tsyntax}[empty]{simplify}
  \fix{Missing description of simplify}.
  \end{tsyntax}
\end{tactic}

% --------------------------------------------------------------------
\begin{tactic}{sim}
\end{tactic}

% --------------------------------------------------------------------
\begin{tactic}{skip}
  \begin{tsyntax}[empty]{skip}
  \fix{Missing description of skip}.
  \end{tsyntax}
\end{tactic}

% --------------------------------------------------------------------
\begin{tactic}[smt $\textit{ smt-options}$]{smt}
  \begin{tsyntax}[empty]{smt}
  Try to solve the goal using SMT solvers. The goal is sent along with 
  the local hypotheses plus a selected number of axioms/lemmas.
  \end{tsyntax}
  Generic options are:
  \begin{itemize}
    \item \ec{timeout=}$n$: set the timeout for provers to $n$ (in seconds).
    \item \ec{maxprovers=}$n$: set the maximun number of prover runing in 
          parallele to $n$ 
    \item \ec{prover=[}\textit{prover-selector}\ec{]} : select the provers. \\
          Variant [\textit{prover-selector}]. \\
          \textit{prover-selector} can be:
          \begin{itemize}
            \item \ec{``}\textit{prover-name}\ec{''}: use this particular prover
            \item \ec{+``}\textit{prover-name}\ec{''}: add \textit{prover-name} 
                    to the current list of provers
            \item \ec{-``}\textit{prover-name}\ec{''}: 
                    remove \textit{prover-name} 
                    from the current list of provers 
          \end{itemize}
          Examples:
          \begin{itemize}
          \item \ec{[``Z3'' ``Alt-Ergo'']}: use only Z3 and Alt-Ergo 
          \item \ec{[``Z3'' ``Alt-Ergo'' -''Z3'']}: use only Alt-Ergo 
          \item \ec{[-''CVC4'']}: remove CVC4 form the current list of prover,
                so assumming the current list is Z3 and CVC4 this is equivalent
                to \ec{[``Z3'']}
          \item \ec{[+''CVC4'']}: add CVC4 to the current list of prover,
                so assumming the current list is Z3 and Alt-Ergo this is
                equivalent
                to \ec{[``Z3'' ``Alt-Ergo'' ``CVC4'' ]}
          \end{itemize}          
  \end{itemize}
  Axioms and lemmas are not all send to smt provers, 
  \EasyCrypt use a strategy to automatically select them.
  Lemmas and axioms marked with ``nosmt'' are not selected.
  This strategy can be parametrized using different options:
  \begin{itemize}    
    \item \ec{unwantedlemmas=}\textit{dbhint}: 
          do not send axiom/lemma selected by \textit{dbhint}
    \item \ec{wantedlemmas=}\textit{dbhint}: 
          send axiom/lemma selected by \textit{dbhint} 
    \item \ec{all}: 
          select all available axioms/lemmas execpted those specified by 
          \ec{unwantedlemmas} (if any).
    \item \ec{maxlemmas=}$n$: 
          set the maximun number of selected axioms/lemmas to $n$.
          Keep this number small is generally more effienciant.
          Variant: $n$
    \item \ec{iterate}: try to incrementally augment the number of selected
          axioms/lemmas. Last call will be equivalent to all.
  \end{itemize}

  \fixme{Describe \textit{dbhint} options.}

  Options can also be specified by short name, for example:
  \begin{center} \ec{smt 100 [+''Z3] tmo=4 mp=2}\end{center}
  is equivalent to 
  \begin{center}
  \ec{smt maxlemmas=100 prover=[+''Z3] timeout=4 maxprovers=2}
  \end{center}

  Smt option can be set globally using the following syntax:\\
  \ec{prover} \texit{smt-options}


\end{tactic}

% --------------------------------------------------------------------
\begin{tactic}{split}
  \begin{tsyntax}[empty]{split}
  Break an intrinsically conjunctive goal into its component subgoals.
  For instance, it can:
  \begin{itemize}
    \item close any goal that is convertible to \tct{true} or provable by \tct{reflexivity},
    \item replace a logical equivalence by the direct and indirect implication,
    \item replace a goal of the form \tct{f1 /\\ f2} by the two subgoals for \tct{f1} an
          \tct{f2}. The same applies for a goal of the form \tct{f1 && f2},
    \item replace an equality between $n$-tuples by $n$ equalities
          on their components.
  \end{itemize}
  \end{tsyntax}
\end{tactic}

% --------------------------------------------------------------------
\begin{tactic}{splitwhile}
  \begin{tsyntax}[empty]{splitwhile}
  \fix{Missing description of splitwhile}.
  \end{tsyntax}
\end{tactic}

% --------------------------------------------------------------------
\begin{tactic}{sp}
  \begin{tsyntax}[empty]{sp}
  \fix{Missing description of sp}.
  \end{tsyntax}
\end{tactic}

% --------------------------------------------------------------------
\begin{tactic}[subst | subst x]{subst}
  \begin{tsyntax}[empty]{subst}
  Search for the first equation of the form \ec{x = f} or \ec{f = x} in the context
  and replace all the occurrences of \ec{x} by \ec{f} everywhere in the context and the
  goal before clearing it. If no identifier is given, repeatedly apply the tactic to
  all identifiers for which such an equation exists.
  \end{tsyntax}
\end{tactic}

% --------------------------------------------------------------------
\begin{tactic}{swap}
\end{tactic}

% --------------------------------------------------------------------
\begin{tactic}{symmetry}
  \begin{tsyntax}{symmetry}
  In \prhl, swaps the two programs, transforming the pre and
  postconditions by swapping the memories they refer to.

  \textbf{Examples:} In the following, $\invrel{\cdot}$ inverses its
  argument relation. (That is, for any relation $R$ and any $m_1$,
  $m_2$, we have
  $m_1 \mathrel{R} m_2\Leftrightarrow m_2 \mathrel{\invrel{R}} m_1$.)
  \begin{mathpar}
  \inferrule%%
    {\pRHL{\invrel{P}}{c_2}{c_1}{\invrel{Q}}}%%
    {\pRHL{P}{c_1}{c_2}{Q}}%%
    \quad\mbox{(\prhl)\quad\parbox{50pt}{\tct{symmetry}}}
  \end{mathpar}
  \end{tsyntax}
\end{tactic}

% --------------------------------------------------------------------
\begin{tactic}{transitivity}
\end{tactic}

% --------------------------------------------------------------------
\begin{tactic}{trivial}
  \begin{tsyntax}[empty]{trivial}
  \fix{Missing description of trivial}.
  \end{tsyntax}
\end{tactic}

% --------------------------------------------------------------------
\begin{tactic}{unroll}
  \begin{tsyntax}[empty]{unroll}
  \fix{Missing description of unroll}.
  \end{tsyntax}
\end{tactic}

% --------------------------------------------------------------------
\begin{tactic}{wp}
  \begin{tsyntax}{wp}
  Computes the weakest precondition of a straightline deterministic
  suffix of the program(s) that implies the current
  postcondition. \tct{wp} also consumes deterministic \tct{if}
  statements (when both branches are deterministic straightline code
  without procedure calls).
  \end{tsyntax}

  \begin{tsyntax}{wp $\ n_1$ $\ n_2$}
  In \prhl, let \tct{wp} consume \emph{exactly} $n_1$ statements of
  the left program and $n_2$ statements of the right program.
  \end{tsyntax}

  \begin{tsyntax}{wp $\ n$}
  In \phl and \hl, let \tct{wp} consume \emph{exactly} $n$ statements
  of the program.
  \end{tsyntax}
\end{tactic}

% --------------------------------------------------------------------
\begin{tactic}{zeta}
  \begin{tsyntax}[empty]{zeta}
  \fix{Missing description of zeta}.
  \end{tsyntax}
\end{tactic}


% Libraries
\chapter{Standard Library}
\section{Theories}

\section{Cryptographic Assumptions and Properties}

%%% Local Variables: 
%%% mode: latex
%%% TeX-master: "easycrypt"
%%% End: 

% Examples
% !TeX root = easycrypt.tex

\chapter{Advanced Examples}

%%% Local Variables: 
%%% mode: latex
%%% TeX-master: "easycrypt"
%%% End: 


\addcontentsline{toc}{chapter}{References}
\bibliographystyle{plainnat}
\bibliography{references}

%\part{Language Reference}

\cleardoublepage
\addcontentsline{toc}{part}{Index}
\printindex{easycrypt}{General Index}
\printindex{ambient}{Index of Ambient Logic Tactics}
\printindex{pL}{Index of Program Logics Tactics}

\end{document}

\lstset{language=easycrypt}

\def\ls{\lstinline}
\def\ec#1{\lstinline[language=easycrypt]"#1"}

\def\Arg{\ensuretext{\ec{arg}}}

% --------------------------------------------------------------------
% Typesetting judgments
\newcommand{\pRHL}[4]{\{#1\}\; #2 \mathrel{\sim} #3\; \{#4\}}
\newcommand{\pHL}[5]{\{#1\}\; #2\; \{#3\} \mathrel{#4} #5}
\newcommand{\HL}[3]{\{#1\}\; #2\; \{#3\}}
\newcommand{\PR}[4]{\mathbf{Pr} [#3, #1(#2) : #4]}

\newcommand{\pRHLs}[4]{\ec{equiv [#2 ~ #3: #1 ==> #4]}}
\newcommand{\pHLs}[5]{\ec{phoare [#2: #1 ==> #3] #4 #5}}
\newcommand{\HLs}[3]{\ec{hoare [#2: #1 ==> #3]}}
\newcommand{\PRs}[4]{\ec{Pr[#1(#2) @ &#3: #4]}}

% --------------------------------------------------------------------
\newcommand{\mem}[1]{#1}
\newcommand{\inmem}[2]{#1\langle{#2}\rangle}
