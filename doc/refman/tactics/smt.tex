% --------------------------------------------------------------------
\begin{tactic}[smt $\textit{ smt-options}$]{smt}
  \begin{tsyntax}[empty]{smt}
  Try to solve the goal using SMT solvers. The goal is sent along with 
  the local hypotheses plus a selected number of axioms/lemmas.
  \end{tsyntax}
  Generic options are:
  \begin{itemize}
    \item \ec{timeout=}$n$: set the timeout for provers to $n$ (in seconds).
    \item \ec{maxprovers=}$n$: set the maximun number of prover runing in 
          parallele to $n$ 
    \item \ec{prover=[}\textit{prover-selector}\ec{]} : select the provers. \\
          Variant [\textit{prover-selector}]. \\
          \textit{prover-selector} can be:
          \begin{itemize}
            \item \ec{``}\textit{prover-name}\ec{''}: use this particular prover
            \item \ec{+``}\textit{prover-name}\ec{''}: add \textit{prover-name} 
                    to the current list of provers
            \item \ec{-``}\textit{prover-name}\ec{''}: 
                    remove \textit{prover-name} 
                    from the current list of provers 
          \end{itemize}
          Examples:
          \begin{itemize}
          \item \ec{[``Z3'' ``Alt-Ergo'']}: use only Z3 and Alt-Ergo 
          \item \ec{[``Z3'' ``Alt-Ergo'' -''Z3'']}: use only Alt-Ergo 
          \item \ec{[-''CVC4'']}: remove CVC4 form the current list of prover,
                so assumming the current list is Z3 and CVC4 this is equivalent
                to \ec{[``Z3'']}
          \item \ec{[+''CVC4'']}: add CVC4 to the current list of prover,
                so assumming the current list is Z3 and Alt-Ergo this is
                equivalent
                to \ec{[``Z3'' ``Alt-Ergo'' ``CVC4'' ]}
          \end{itemize}          
  \end{itemize}
  Axioms and lemmas are not all send to smt provers, 
  \EasyCrypt use a strategy to automatically select them.
  Lemmas and axioms marked with ``nosmt'' are not selected.
  This strategy can be parametrized using different options:
  \begin{itemize}    
    \item \ec{unwantedlemmas=}\textit{dbhint}: 
          do not send axiom/lemma selected by \textit{dbhint}
    \item \ec{wantedlemmas=}\textit{dbhint}: 
          send axiom/lemma selected by \textit{dbhint} 
    \item \ec{all}: 
          select all available axioms/lemmas execpted those specified by 
          \ec{unwantedlemmas} (if any).
    \item \ec{maxlemmas=}$n$: 
          set the maximun number of selected axioms/lemmas to $n$.
          Keep this number small is generally more effienciant.
          Variant: $n$
    \item \ec{iterate}: try to incrementally augment the number of selected
          axioms/lemmas. Last call will be equivalent to all.
  \end{itemize}

  \fixme{Describe \textit{dbhint} options.}

  Options can also be specified by short name, for example:
  \begin{center} \ec{smt 100 [+''Z3] tmo=4 mp=2}\end{center}
  is equivalent to 
  \begin{center}
  \ec{smt maxlemmas=100 prover=[+''Z3] timeout=4 maxprovers=2}
  \end{center}

  Smt option can be set globally using the following syntax:\\
  \ec{prover} \texit{smt-options}


\end{tactic}
