% --------------------------------------------------------------------
\section{Tactics}

\subsection{Ambient logic}

% --------------------------------------------------------------------
\begin{tactic}{idtac}
\end{tactic}

% --------------------------------------------------------------------
\begin{tactic}[move | move: $\;\pi_1 \cdots \pi_n$]{move}
  \begin{tsyntax}{move}
     Does nothing, equivalent to \rtactic{idtac}. This form is mainly
     used in conjonction with an introduction pattern (see
     Section~\ref{s:intro-pattern}), e.g. \ls!move=> $\iota_1 \cdots \iota_n$!.
  \end{tsyntax}

  \begin{tsyntax}{move: $\;\pi_1 \cdots \pi_n$}
    Generalize the patterns $\pi_1, \cdots, \pi_n$, starting from
    $\pi_n$ and going back.
    %See Section~\ref{s:gen-pattern} for more
    %information on the generalization mechanism.
  \end{tsyntax}
\end{tactic}

% --------------------------------------------------------------------
\begin{tactic}[clear $\;x_1 \cdots x_n$]{clear}
  \begin{tsyntax}[empty]{clear}
  Clear the local variables and hypotheses $x_1 \cdots x_n$ from the
  local context. Fail if any remaining hypotheses depend on any of the
  $x_i$.
  \end{tsyntax}
\end{tactic}

% --------------------------------------------------------------------
\begin{tactic}{done}
  \begin{tsyntax}[empty]{done}
  \fix{Missing description of done}.
  \end{tsyntax}
\end{tactic}

% --------------------------------------------------------------------
\begin{tactic}{apply}
\end{tactic}

% --------------------------------------------------------------------
\begin{tactic}{exact}
  \begin{tsyntax}[empty]{exact}
  \fix{Missing description of exact}.
  \end{tsyntax}
\end{tactic}

% --------------------------------------------------------------------
\begin{tactic}{assumption}
  \begin{tsyntax}[empty]{assumption}
  Search in the context for a hypothesis that is convertible to the goal
  and apply. Fail if none can be found.
  \end{tsyntax}
\end{tactic}

% --------------------------------------------------------------------
\begin{tactic}{pose}
  \begin{tsyntax}[empty]{pose}
  \fix{Missing description of pose}.
  \end{tsyntax}
\end{tactic}

% --------------------------------------------------------------------
\begin{tactic}[cut $\;\iota$: $\;\phi$]{cut}
  Same as \rtactic{have}.
\end{tactic}


% --------------------------------------------------------------------
\begin{tactic}{rewrite}
\end{tactic}

% --------------------------------------------------------------------
\begin{tactic}{rwnormal}
  \begin{tsyntax}[empty]{rwnormal}
  \fix{Missing description of rwnormal}.
  \end{tsyntax}
\end{tactic}

% --------------------------------------------------------------------
\begin{tactic}[subst | subst x]{subst}
  \begin{tsyntax}[empty]{subst}
  Search for the first equation of the form \ec{x = f} or \ec{f = x} in the context
  and replace all the occurrences of \ec{x} by \ec{f} everywhere in the context and the
  goal before clearing it. If no identifier is given, repeatedly apply the tactic to
  all identifiers for which such an equation exists.
  \end{tsyntax}
\end{tactic}


% --------------------------------------------------------------------
\begin{tactic}{split}
  \begin{tsyntax}[empty]{split}
  Break an intrinsically conjunctive goal into its component subgoals.
  For instance, it can:
  \begin{itemize}
    \item close any goal that is convertible to \tct{true} or provable by \tct{reflexivity},
    \item replace a logical equivalence by the direct and indirect implication,
    \item replace a goal of the form \tct{f1 /\\ f2} by the two subgoals for \tct{f1} an
          \tct{f2}. The same applies for a goal of the form \tct{f1 && f2},
    \item replace an equality between $n$-tuples by $n$ equalities
          on their components.
  \end{itemize}
  \end{tsyntax}
\end{tactic}

% --------------------------------------------------------------------
\begin{tactic}{left}
  \begin{tsyntax}[empty]{left}
  Reduce a disjunctive goal to its left member.
  \end{tsyntax}
\end{tactic}

% --------------------------------------------------------------------
\begin{tactic}{right}
\end{tactic}


% --------------------------------------------------------------------
\begin{tactic}[case $\;\phi$]{case}
  \begin{tsyntax}[empty]{case}
  Do an excluded-middle case analysis on $\phi$, substituting $\phi$
  in the goal.
  \end{tsyntax}

  \fixme{Describe the behaviour of \ec{case} on inductives.}
\end{tactic}

% --------------------------------------------------------------------
\begin{tactic}{elim}
\end{tactic}


% --------------------------------------------------------------------
\begin{tactic}{change}
  \begin{tsyntax}[empty]{change}
  \fix{Missing description of change}.
  \end{tsyntax}
\end{tactic}

% --------------------------------------------------------------------
\begin{tactic}{simplify}
  \begin{tsyntax}[empty]{simplify}
  \fix{Missing description of simplify}.
  \end{tsyntax}
\end{tactic}

% --------------------------------------------------------------------
\begin{tactic}{progress}
\end{tactic}


% --------------------------------------------------------------------
\begin{tactic}{beta}
\end{tactic}

% --------------------------------------------------------------------
\begin{tactic}{delta}
  \begin{tsyntax}[empty]{delta}
  \fix{Missing description of delta}.
  \end{tsyntax}
\end{tactic}

% --------------------------------------------------------------------
\begin{tactic}{zeta}
  \begin{tsyntax}[empty]{zeta}
  \fix{Missing description of zeta}.
  \end{tsyntax}
\end{tactic}

% --------------------------------------------------------------------
\begin{tactic}{iota}
  \begin{tsyntax}[empty]{iota}
  \fix{Missing description of iota}.
  \end{tsyntax}
\end{tactic}

% --------------------------------------------------------------------
\begin{tactic}{logic}
\end{tactic}


% --------------------------------------------------------------------
\begin{tactic}{reflexivity}
\end{tactic}


% --------------------------------------------------------------------
\begin{tactic}{ringeq}
  \begin{tsyntax}[empty]{ringeq}
  \fix{Missing description of ringeq}.
  \end{tsyntax}
\end{tactic}

% --------------------------------------------------------------------
\begin{tactic}{fieldeq}
\end{tactic}

% --------------------------------------------------------------------
\begin{tactic}{algebra}
  \begin{tsyntax}[empty]{algebra}
  \fix{Missing description of algebra}.
  \end{tsyntax}
\end{tactic}

% --------------------------------------------------------------------
\begin{tactic}{congr}
  \begin{tsyntax}[empty]{congr}
  Replace a goal of the form \ec{f t$_1$ ... t$_n$ = f u$_1$ ... u$_n$}
  with the subgoals \ec{t$_i$ = u$_i$} for all \ec{$i$}. Subgoals solvable
  by \ec{reflexivity} are automatically closed.
  \end{tsyntax}
\end{tactic}


% --------------------------------------------------------------------
\begin{tactic}{trivial}
  \begin{tsyntax}[empty]{trivial}
  \fix{Missing description of trivial}.
  \end{tsyntax}
\end{tactic}

% --------------------------------------------------------------------
\begin{tactic}[smt $\textit{ smt-options}$]{smt}
  \begin{tsyntax}[empty]{smt}
  Try to solve the goal using SMT solvers. The goal is sent along with 
  the local hypotheses plus a selected number of axioms/lemmas.
  \end{tsyntax}
  Generic options are:
  \begin{itemize}
    \item \ec{timeout=}$n$: set the timeout for provers to $n$ (in seconds).
    \item \ec{maxprovers=}$n$: set the maximun number of prover runing in 
          parallele to $n$ 
    \item \ec{prover=[}\textit{prover-selector}\ec{]} : select the provers. \\
          Variant [\textit{prover-selector}]. \\
          \textit{prover-selector} can be:
          \begin{itemize}
            \item \ec{``}\textit{prover-name}\ec{''}: use this particular prover
            \item \ec{+``}\textit{prover-name}\ec{''}: add \textit{prover-name} 
                    to the current list of provers
            \item \ec{-``}\textit{prover-name}\ec{''}: 
                    remove \textit{prover-name} 
                    from the current list of provers 
          \end{itemize}
          Examples:
          \begin{itemize}
          \item \ec{[``Z3'' ``Alt-Ergo'']}: use only Z3 and Alt-Ergo 
          \item \ec{[``Z3'' ``Alt-Ergo'' -''Z3'']}: use only Alt-Ergo 
          \item \ec{[-''CVC4'']}: remove CVC4 form the current list of prover,
                so assumming the current list is Z3 and CVC4 this is equivalent
                to \ec{[``Z3'']}
          \item \ec{[+''CVC4'']}: add CVC4 to the current list of prover,
                so assumming the current list is Z3 and Alt-Ergo this is
                equivalent
                to \ec{[``Z3'' ``Alt-Ergo'' ``CVC4'' ]}
          \end{itemize}          
  \end{itemize}
  Axioms and lemmas are not all send to smt provers, 
  \EasyCrypt use a strategy to automatically select them.
  Lemmas and axioms marked with ``nosmt'' are not selected.
  This strategy can be parametrized using different options:
  \begin{itemize}    
    \item \ec{unwantedlemmas=}\textit{dbhint}: 
          do not send axiom/lemma selected by \textit{dbhint}
    \item \ec{wantedlemmas=}\textit{dbhint}: 
          send axiom/lemma selected by \textit{dbhint} 
    \item \ec{all}: 
          select all available axioms/lemmas execpted those specified by 
          \ec{unwantedlemmas} (if any).
    \item \ec{maxlemmas=}$n$: 
          set the maximun number of selected axioms/lemmas to $n$.
          Keep this number small is generally more effienciant.
          Variant: $n$
    \item \ec{iterate}: try to incrementally augment the number of selected
          axioms/lemmas. Last call will be equivalent to all.
  \end{itemize}

  \fixme{Describe \textit{dbhint} options.}

  Options can also be specified by short name, for example:
  \begin{center} \ec{smt 100 [+''Z3] tmo=4 mp=2}\end{center}
  is equivalent to 
  \begin{center}
  \ec{smt maxlemmas=100 prover=[+''Z3] timeout=4 maxprovers=2}
  \end{center}

  Smt option can be set globally using the following syntax:\\
  \ec{prover} \texit{smt-options}


\end{tactic}


% --------------------------------------------------------------------
\begin{tactic}{admit}
\end{tactic}


\subsection{Program Logics}

Judgments in the program logics may refer to procedures or
statements. Whenever the context allows both, we use $c$ (or \tct{c})
to denote programs, using $f$ (or \tct{f}) when only judgments on
procedures are allowed by the context, and $s$ (or \tct{s}) when only
judgments on statements are allowed.

\EasyCrypt includes three different program logics:
\begin{itemize}
\item \prhl, or probabilistic relational Hoare logic, with judgments of the form
%%
$$\pRHL{P}{c_1}{c_2}{Q}$$
%%
where $c_1$ and $c_2$ are programs, and $P$ and $Q$
are relations on memories.
\item \phl, or probabilistic Hoare logic, with judgments of the form
%%
$$\pHL{P}{c}{Q}{\diamond}{\delta}$$
%%
where $c$ is a program, $P$ and $Q$ are predicates on memories,
$\diamond\in\{\leq,\geq,=\}$ is a comparison relation and $\delta$ is
a real-valued expression, evaluated in the initial memory.
\item \hl, or (possibilistic) Hoare logic, with judgments of the form
%%
$$\HL{P}{c}{Q}$$
%%
where $c$ is a program, and $P$ and $Q$ are predicates on memories.
\end{itemize}

When $c$ is a procedure, preconditions ($P$ above) operate on memories
extended with a special $\Arg$ location that refers to the procedure's
arguments, and postcondition ($Q$ above) operate on memories extended
with a special $\Res$ location that refers to the procedure's return
value.

In the following, given a relation $R$, we denote with $\invrel{R}$
its inverse relation (that is,
%%
$\Rel{R}{m_1}{m_2} \Leftrightarrow \Rel{\invrel{R}}{m_2}{m_1}$).

We denote with $\diamond^{\uparrow}$ the function defined by
$$
\cdot^{\uparrow} =
\left\{\begin{array}{l @{\quad\mapsto\quad} r}
=    & \Leftrightarrow \\
\leq & \Leftarrow      \\
\geq & \Rightarrow
\end{array}\right.
$$

Given a predicate $P$, we denote with $\inmem{P}{1}$
(resp. $\inmem{P}{2}$) the relation defined by
%%
$\Rel{\inmem{P}{1}}{m_1}{m_2} \Leftrightarrow \Pred{P}{m_1}$
(resp. $\Rel{\inmem{P}{2}}{m_1}{m_2} \Leftrightarrow \Pred{P}{m_2}$).
We lift logical connectors to predicates and relations over memories
in the natural way.

TODO: define \tct{<spec>}, \tct{<lemma>}, \tct{<prhl>}, \tct{<phl>},
\tct{<hl>}.

\paragraph{Reasoning on Specifications}
% --------------------------------------------------------------------
\begin{tactic}{symmetry}
  \begin{tsyntax}{symmetry}
  In \prhl, swaps the two programs, transforming the pre and
  postconditions by swapping the memories they refer to.

  \textbf{Examples:} In the following, $\invrel{\cdot}$ inverses its
  argument relation. (That is, for any relation $R$ and any $m_1$,
  $m_2$, we have
  $m_1 \mathrel{R} m_2\Leftrightarrow m_2 \mathrel{\invrel{R}} m_1$.)
  \begin{mathpar}
  \inferrule%%
    {\pRHL{\invrel{P}}{c_2}{c_1}{\invrel{Q}}}%%
    {\pRHL{P}{c_1}{c_2}{Q}}%%
    \quad\mbox{(\prhl)\quad\parbox{50pt}{\tct{symmetry}}}
  \end{mathpar}
  \end{tsyntax}
\end{tactic}

% --------------------------------------------------------------------
\begin{tactic}{transitivity}
\end{tactic}

% --------------------------------------------------------------------
\begin{tactic}{conseq}
  \begin{tsyntax}{conseq <specification>}
  Rule of consequence. Proves a specification by weakening of a
  stronger result. Any one of the specification places can be filled
  with a wildcard \tct{_} to keep the value it contains in the current
  goal and trivially discharge the corresponding subgoal.

  \textbf{Examples:} In the following, $\leq^\uparrow$ (resp. $=^\uparrow$,
  $\geq^\uparrow$) is $\Leftarrow$ (resp. $\Leftrightarrow$ and
  $\Rightarrow$).
  \begin{mathpar}
  \inferrule*[left=(pRHL),rightskip=10em]%%
    {P' \Rightarrow P \\%
     Q \Rightarrow Q' \\%
     \pRHL{P}{c}{c'}{Q}}%%
    {\pRHL{P'}{c}{c'}{Q'}}%%
    \quad\raisebox{.7em}{\tct{conseq (_: P ==> Q)}} \\
  \inferrule*[left=(pHL),rightskip=10em]%%
    {P' \Rightarrow \delta \mathrel{\diamond} \delta' \\%
     P' \Rightarrow P \\%
     Q \mathrel{\diamond^\uparrow} Q' \\%
     \pHL{P}{c}{Q}{\diamond}{\delta}}%%
    {\pHL{P'}{c}{Q'}{\diamond}{\delta'}}%%
    \quad\raisebox{.7em}{\tct{conseq (_: P ==> Q: $\delta$)}} \\
  \inferrule*[left=(HL),rightskip=10em]%%
    {P' \Rightarrow P \\%
     Q \Rightarrow Q' \\%
     \HL{P}{c}{Q}}%%
    {\HL{P'}{c}{Q'}}%%
    \quad\raisebox{.7em}{\tct{conseq (_: P ==> Q)}} \\
  \end{mathpar}
  \end{tsyntax}

  \begin{tsyntax}{conseq* <specification>}
  Same as \tct{conseq <specification>}, but the subgoal corresponding
  to the postcondition is refined by a ``may modify'' analysis.
  \end{tsyntax}
\end{tactic}

% --------------------------------------------------------------------
\begin{tactic}{phoare split}
  \begin{tsyntax}{phoare split $\delta_{A}$ $\delta_{B}$ $\delta_{AB}$}
  Splits a \phl judgment whose postcondition is a conjunction or
  disjunction into three \phl judgments following the definition of
  the probability of a disjunction of events.

  \paragraph{Examples:}\strut

  \begin{cmathpar}
  \texample[\phl{}]
    {\ec{phoare split $\ \delta_{A}\ \delta_{B}\ \delta_{AB}$}}
    {\delta_{A} + \delta_{B} - \delta_{AB} \diamond \delta \\
     \pHL{P}{c}{A}{\diamond}{\delta_{A}} \\
     \pHL{P}{c}{B}{\diamond}{\delta_{B}} \\
     \pHL{P}{c}{A \wedge B}{\invrel{\diamond}}{\delta_{AB}}}
    {\pHL{P}{c}{A \vee B}{\diamond}{\delta}}

  \texample[\phl{}]
    {\ec{phoare split $\ \delta_{A}\ \delta_{B}\ \delta_{AB}$}}
    {\delta_{A} + \delta_{B} - \delta_{AB} \diamond \delta \\
     \pHL{P}{c}{A}{\diamond}{\delta_{A}} \\
     \pHL{P}{c}{B}{\diamond}{\delta_{B}} \\
     \pHL{P}{c}{A \vee B}{\invrel{\diamond}}{\delta_{AB}}}
    {\pHL{P}{c}{A \wedge B}{\diamond}{\delta}}
  \end{cmathpar}
  \end{tsyntax}

  \begin{tsyntax}{phoare split $\ {!}$ $\ \delta_{\top}$ $\ \delta_{!}$}
  Splits a \phl judgment into two judgments whose postcondition are
  true and the negation of the original postcondition, respectively.

  \paragraph{Examples:}\strut

  \begin{cmathpar}
  \texample[\phl{}]
    {\ec{phoare split ! $\ \delta_{\top}\ \delta_{!}$}}
    {\delta_{\top} - \delta_{!} \diamond \delta \\
     \pHL{P}{c}{\mathsf{true}}{\diamond}{\delta_{\top}} \\
     \pHL{P}{c}{!Q}{\invrel{\diamond}}{\delta_{!}}}
    {\pHL{P}{c}{Q}{\diamond}{\delta}}
  \end{cmathpar}
  \end{tsyntax}

  \begin{tsyntax}{phoare split $\delta_{A}$ $\delta_{!A}$: A}
  Splits a \phl judgment following an event $A$.

  \paragraph{Examples:}\strut

  \begin{cmathpar}
  \texample[\phl{}]
    {\ec{phoare split $\ \delta_{A}\ \delta_{!A}$: A}}
    {\delta_{A} + \delta_{!A} \diamond \delta \\
     \pHL{P}{c}{Q \wedge A}{\diamond}{\delta_{A}} \\
     \pHL{P}{c}{Q \wedge \neg A}{\diamond}{\delta_{!A}}}
    {\pHL{P}{c}{Q}{\diamond}{\delta}}
  \end{cmathpar}
  \end{tsyntax}  
\end{tactic}

% --------------------------------------------------------------------
\begin{tactic}{byequiv}

  \begin{tsyntax}{byequiv [option]? <specification>}
  Derives probability relation from \prhl judgements. 
  Only applies to judgments on procedures.
 
  \textbf{Examples:}
  \begin{mathpar}
    \inferrule*[left=(\prhl),rightskip=5em]%%
    { \pRHL{P}{f_1}{f_2}{Q} \\%
      P~\vec{a}_1~m_1~\vec{a}_2~m_2 \\%
      Q \Rightarrow E_1\{1\}  \Leftrightarrow E_2\{2\} }%%
    { \PR{f_1}{\vec{a}_1}{\mem{m_1}}{E_1} = \PR{f_2}{\vec{a}_2}{m_2}{E_2} }%%
    \quad\raisebox{.7em}{\tct{byequiv (: P ==> Q)} } \\
    \inferrule*[left=(\prhl),rightskip=5em]%%
    { \pRHL{P}{f_1}{f_2}{Q} \\% 
      P~\vec{a}_1~m_1~\vec{a}_2~m_2 \\%
      Q \Rightarrow E_1\{1\}  \Rightarrow E_2\{2\} }%%
    { \PR{f_1}{\vec{a}_1}{\mem{m_1}}{E_1} \leq \PR{f_2}{\vec{a}_2}{m_2}{E_2} }%%
    \quad\raisebox{.7em}{\tct{byequiv (: P ==> Q)} } \\
    \inferrule*[left=(\prhl),rightskip=5em]%%
    { \pRHL{P}{f_1}{f_2}{Q} \\%
      P~\vec{a}_1~m_1~\vec{a}_2~m_2 \\%
      Q \Rightarrow E_2\{2\}  \Rightarrow E_1\{1\} } %%
    { \PR{f_1}{\vec{a}_1}{\mem{m_1}}{E_1} \geq \PR{f_2}{\vec{a}_2}{m_2}{E_2} }%%
    \raisebox{.7em}{\tct{byequiv (: P ==> Q)} } 
  \end{mathpar}
 
 \end{tsyntax}

  Possible options are \tct{-eq} or \tct{eq}.
  Any one of the specification places can be filled
  with a wildcard \tct{_}. It that case the corresponding argument 
  is automatically inferred. Some time the infered postcondition  
  is stronger than necessary, in that case use the option \tct{-eq}.

  \fix{Missing description of byequiv for upto}.
  
  \begin{tsyntax}{byequiv <lemma>}
  Same as \tct{byequiv <specification>}, but the specification to use is 
  inferred from the lemma provided. Raises an error if the lemma does 
  not refer to the expected procedures. All variants of \tct{byequiv} 
  may take lemmas in place of explicit specifications with the same effect.
  \end{tsyntax}


\end{tactic}

% --------------------------------------------------------------------
\begin{tactic}{byphoare}
  \begin{tsyntax}{byphoare [option]? <spec>}
  Derives a probability relation from a \phl judgement on the
  procedure involved. \tct{<spec>} can include wildcards when the
  tactic should infer the pre or postcondition.

  \textbf{Options:} By default, (\tct{eq} option) specification
  inference attempts to infer a conjunction of equalities sufficient
  to imply the desired relation. Passing the \tct{-eq} option
  overrides this behaviour, instead using the trivial relation on
  events.

  \textbf{Examples:}
  \begin{mathpar}
    \inferrule%%
      {\pHL{P}{f}{Q}{=}{\delta} \\%
       \Pred{P}{m[\Arg\mapsto\vec{a}]} \\%
       \forall \mem{m'}.\,\Pred{Q}{m'} \Leftrightarrow \Pred{E}{m'}}%%
      {\PR{f}{\vec{a}}{\mem{m}}{E} = \delta}%%
      \quad\mbox{\parbox{200pt}{\tct{byphoare (_: P ==> Q)}}} \\
  \end{mathpar}
  \end{tsyntax}

  \begin{tsyntax}{byphoare <lemma>}
  Same as \tct{byphoare <spec>}, but the specification to use is
  inferred from the lemma provided. Raises an error if the lemma does
  not refer to the expected procedure. Inference options have no
  effect in this setting.
  \end{tsyntax}
\end{tactic}

% --------------------------------------------------------------------
\begin{tactic}{hoare}
  \begin{tsyntax}{hoare <spec>}
  Derives a null probability from a \hl judgement on the procedure
  involved. \ec{hoare} can also be used to derive \phl judgments and
  certain probability inequalities by automatically applying
  \rtactic{conseq}.

  \paragraph{Examples:}\strut

  \begin{cmathpar}
    \texample
      {\ec{hoare}}
      {\HL{\mathsf{true}}{f}{\neg Q}}
      {\PR{f}{\vec{a}}{\mem{m}}{Q} = 0}

    \texample
      {\ec{hoare}}
      {\HL{\mathsf{P}}{f}{\neg Q}}
      {\pHL{P}{f}{Q}{\leq}{0}}
  \end{cmathpar}
  \end{tsyntax}
\end{tactic}

% --------------------------------------------------------------------
\begin{tactic}{bypr}
  \begin{tsyntax}{bypr}
  Derives a program judgment from a probability relation or an exact
  probability. Only applies to judgments on procedures.

  \textbf{Examples:}
  \begin{mathpar}
    \inferrule*[left=(\prhl),rightskip=10em]%%
    {\forall m_1, m_2, a.\, E_1 = a \Rightarrow E_2 = a \Rightarrow Q~m_1~ m_2 \\%
     \forall \vec{a}_1, \vec{a}_2, m_1, m_2, a.\, P~\vec{a}_1~m_1~\vec{a}_2~m_2 \Rightarrow%
       \PR{f_1}{\vec{a}_1}{\mem{m_1}}{a = E_1} = \PR{f_2}{\vec{a}_2}{\mem{m_2}}{a = E_2}} %%
    {\pRHL{P}{f_1}{f_2}{Q}}%%
    \quad\raisebox{.7em}{\tct{bypr (E$_1$) (E$_2$)}} \\
  \inferrule*[left=(\phl),rightskip=10em]%%
    {\forall m, \vec{a}.\,P~\vec{a}~m \Rightarrow \PR{f}{\vec{a}}{m}{E} \mathrel{\diamond} \delta\{m\}}%%
    {\pHL{P}{f}{E}{\diamond}{\delta}}
    \quad\raisebox{.7em}{\tct{bypr}} \\
  \inferrule*[left=(\hl),rightskip=10em]%%
    {\forall m, \vec{a}.\,P~\vec{a}~m \Rightarrow \PR{f}{\vec{a}}{m}{\neg E} \mathop{=}0$\%$r}%%
    {\HL{P}{f}{E}}
    \quad\raisebox{.7em}{\tct{bypr}} \\
  \end{mathpar}
  \end{tsyntax}
\end{tactic}

% --------------------------------------------------------------------
\begin{tactic}{exfalso}
\end{tactic}

%% --------------------------------------------------------------------
\begin{tactic}{pr\_bounded}
  \begin{tsyntax}[empty]{pr\_bounded}
  \fix{Missing description of pr\_bounded}.
  \end{tsyntax}
\end{tactic}


\paragraph{Reasoning on Programs}
Unless specified, the following program logic tactics operate on a
program's last instruction. Although we describe these tactics as if
they operated on single instructions, their practical implementation
automatically and implicitly applies tactic \rtactic{seq} to deal with
context when necessary.

For simple proofs, it is often enough to simply apply the program
tactic corresponding to the last instruction in the program and let
\tct{smt} deal with the verification condition once the program has
been exhausted.

\medskip

% --------------------------------------------------------------------
\begin{tactic}{skip}
  \begin{tsyntax}[empty]{skip}
  \fix{Missing description of skip}.
  \end{tsyntax}
\end{tactic}

% --------------------------------------------------------------------
\begin{tactic}{seq}
  Rules for sequences:

  \begin{tsyntax}{seq p1 p2 : R}
  \begin{mathpar}
  \inferrule*[left=(\prhl),rightskip=10em]%%
    {|c_1| = \tct{p1}\\%%
     |c_2| = \tct{p2}\\%%
     \pRHL{P}{c_1}{c_2}{R}\\%%
     \pRHL{R}{c_1'}{c_2'}{Q}}%%
    {\pRHL{P}{c_1;c_1'}{c_2;c_2'}{Q}}%%
    \quad\raisebox{.7em}{\tct{seq p1 p2 : R}}\\%%
  \end{mathpar}
  \end{tsyntax}

  \begin{tsyntax}{seq p : R}
  \begin{mathpar}
  \inferrule*[left=(\hl),rightskip=10em]%%
    { |c| = \tct{p}\\%%
     \HL{P}{c}{R}\\%%
     \HL{R}{c'}{Q} }%%
    {\HL{P}{c;c'}{Q}}%%
    \quad\raisebox{.7em}{\tct{seq p : R}}\\%%
  \end{mathpar}
  \end{tsyntax}


  \fix{Missing description of seq for phl}.

\end{tactic}

% --------------------------------------------------------------------
\begin{tactic}{sp}
  \begin{tsyntax}[empty]{sp}
  \fix{Missing description of sp}.
  \end{tsyntax}
\end{tactic}

% --------------------------------------------------------------------
\begin{tactic}{wp}
  \begin{tsyntax}{wp}
  Computes the weakest precondition of a straightline deterministic
  suffix of the program(s) that implies the current
  postcondition. \tct{wp} also consumes deterministic \tct{if}
  statements (when both branches are deterministic straightline code
  without procedure calls).
  \end{tsyntax}

  \begin{tsyntax}{wp $\ n_1$ $\ n_2$}
  In \prhl, let \tct{wp} consume \emph{exactly} $n_1$ statements of
  the left program and $n_2$ statements of the right program.
  \end{tsyntax}

  \begin{tsyntax}{wp $\ n$}
  In \phl and \hl, let \tct{wp} consume \emph{exactly} $n$ statements
  of the program.
  \end{tsyntax}
\end{tactic}

% --------------------------------------------------------------------
\begin{tactic}{rnd}
\end{tactic}

% --------------------------------------------------------------------
\begin{tactic}{if}
  \begin{tsyntax}{if}
  Operates on the \emph{first} statement (instead of the last one). If
  the first statement in the program is
  %%
  \ec{if (b) \{ $c_1$ \} else \{ $c_2$ \}},
  %%
  applying tactic \ec{if} is equivalent to
  \ec{case b; [rcondt 1=> //= | rcondf 1=> //=]}
  (see \rtactic[case-pl]{case}, \rtactic{rcondt}, \rtactic{rcondf}).
  \end{tsyntax}
\end{tactic}

% --------------------------------------------------------------------
\begin{tactic}{while}
  \begin{tsyntax}[empty]{while}
  \fix{Missing description of while}.
  \end{tsyntax}
\end{tactic}

% --------------------------------------------------------------------
\begin{tactic}{call}
  All variants of the \tct{call} tactic implicitly make use of a frame
  rule, based on a ``may modify'' analysis.

  \begin{tsyntax}{call (_: P ==> Q)}
  Compute the precondition of a procedure call using the given
  specification for the procedure. As a side-goal, prove that the
  procedure fulfills the given specification.

  As with other tactics, the specification \tct{(_: P ==> Q)} can be
  replaced with a lemma from which the specification is inferred.
  \end{tsyntax}

  \begin{tsyntax}{call (_: I)}
  Uses invariant \tct{I} to infer a specification for use with
  \tct{tactic}.
  %%
  In \prhl, equivalent to
  \tct{call (_: =$\{\Arg\}$ /\\ I ==> =$\{\Res\}$ /\\ I); first proc I.}
  %%
  In \phl and \hl, equivalent to
  \tct{call (_: I ==> I); first proc I.}
  \end{tsyntax}

  \begin{tsyntax}{call (_: B, I)}
  On \prhl abstract procedures only.
  Equivalent to \tct{call (_: $\neg$B /\\ =$\{\Arg\}$ /\\ I ==> $\neg$B => =$\{\Res\}$ /\\ I); first proc B I.}
  \end{tsyntax}

  \begin{tsyntax}{call (_: B, I, I')}
  On \prhl abstract procedures only.
  Equivalent to \tct{call (_: $\neg$B /\\ =$\{\Arg\}$ /\\ I ==> if $\neg$B then =$\{\Res\}$ /\\ I else I')); first proc B I I'.}
  \end{tsyntax}

  \textbf{Note:} When using the invariant-based variants of
  \tct{call}, error messages may be originating from the underlying
  application of \rtactic{proc}. In particular, when using them to
  deal with abstract procedure calls, the invariant \emph{should not}
  refer to memory locations the abstract procedure may modify.
\end{tactic}

% --------------------------------------------------------------------
\begin{tactic}{proc}
  \begin{tsyntax}[empty]{proc}
  \fix{Missing description of proc}.
  \end{tsyntax}
\end{tactic}



\paragraph{Transforming Programs}
% --------------------------------------------------------------------
\begin{tactic}{swap}
\end{tactic}

% --------------------------------------------------------------------
\begin{tactic}{inline}
\end{tactic}


% --------------------------------------------------------------------
\begin{tactic}{rcondf}
  \begin{tsyntax}[empty]{rcondf}
  \fix{Missing description of rcondf}.
  \end{tsyntax}
\end{tactic}

% --------------------------------------------------------------------
\begin{tactic}{rcondt}
  \begin{tsyntax}[empty]{rcondt}
  \fix{Missing description of rcondt}.
  \end{tsyntax}
\end{tactic}


% --------------------------------------------------------------------
\begin{tactic}{splitwhile}
  \begin{tsyntax}[empty]{splitwhile}
  \fix{Missing description of splitwhile}.
  \end{tsyntax}
\end{tactic}

% --------------------------------------------------------------------
\begin{tactic}{unroll}
  \begin{tsyntax}[empty]{unroll}
  \fix{Missing description of unroll}.
  \end{tsyntax}
\end{tactic}

% --------------------------------------------------------------------
\begin{tactic}{fission}
\end{tactic}

% --------------------------------------------------------------------
\begin{tactic}{fusion}
\end{tactic}


% --------------------------------------------------------------------
\begin{tactic}{alias}
\end{tactic}

% --------------------------------------------------------------------
\begin{tactic}{cfold}
\end{tactic}

% --------------------------------------------------------------------
\begin{tactic}{kill}
  \begin{tsyntax}[empty]{kill}
  \fix{Missing description of kill}.
  \end{tsyntax}
\end{tactic}

% --------------------------------------------------------------------
\begin{tactic}{modpath}
  \begin{tsyntax}[empty]{modpath}
  \fix{Missing description of modpath}.
  \end{tsyntax}
\end{tactic}


\paragraph{Automated Tactics}
% --------------------------------------------------------------------
\begin{tactic}{auto}
  \begin{tsyntax}[empty]{auto}
  \fix{Missing description of auto}.
  \end{tsyntax}
\end{tactic}

% --------------------------------------------------------------------
\begin{tactic}{sim}
\end{tactic}


\paragraph{Advanced Tactics}
% --------------------------------------------------------------------
\begin{tactic}{eager}
  \begin{tsyntax}[empty]{eager}
  \fix{Missing description of eager}.
  \end{tsyntax}
\end{tactic}

% --------------------------------------------------------------------
\begin{tactic}{fel}
  \begin{tsyntax}[empty]{fel}
  \fix{Missing description of fel}.
  \end{tsyntax}
\end{tactic}

