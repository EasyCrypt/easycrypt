\documentclass[a4paper,notitlepage]{book}
\usepackage{anysize}
\marginsize{3cm}{2cm}{1cm}{1cm}

\usepackage{todonotes}
\usepackage{url}
\usepackage{xspace}
\usepackage{syntax}
\usepackage{framed}
\usepackage{stmaryrd}
\usepackage{infer}
\usepackage{makeidx}
\usepackage{verbatim}
% \usepackage{fancyhdr}
% \pagestyle{fancy}

% !TeX root = easycrypt.tex
%% Misc
\newcommand{\DONE}{}% {{\color{red}DONE}}
\newcommand{\Example}{\paragraph*{Example}}
\newcommand{\Syntax}{\paragraph*{Syntax}}
\newcommand{\Description}{\paragraph*{Description}}
\setcounter{secnumdepth}{3}
\renewcommand{\thesubsubsection}{\arabic{chapter}.\arabic{section}.\arabic{subsection}.\arabic{subsubsection}}
\newbox\minicodebox
\newenvironment{minicode}[1]{%
\minipage[t]{#1\linewidth} %
\centering %
\verbatim %
}{%
\endverbatim %
\endminipage% 
}




% \newcounter{todos}\setcounter{todos}{1}
% \newcommand{\TODO}[1]{%
%   \textsf{\textcolor{red}{$^{[\thetodos]}$}}%
%   \marginpar{%
%     \framebox[1.2\marginparwidth][t]{%
%       \parbox[t]{\marginparwidth}{%
%         \raggedright\scriptsize{
%           \textcolor{red}{\thetodos: #1}%
%         }}}}%
%   \stepcounter{todos}%
% }
\newcounter{alarmcounter}
\setcounter{alarmcounter}{1}
\newcommand{\alarm}[1]
            {\begingroup
              \def\thefootnote{{\normalsize\color{red}(\arabic{footnote})}}
              \footnote{\textsf{\textbf{{\color{red}\sc \bf ALARM:}
                  #1}}}\endgroup}

% \providecommand{\note}[1]{
%  \noindent{\color{red}
%  \framebox[\textwidth][t]{%
%   \parbox[t]{0.98\textwidth}{\textcolor{red}{#1}}
% }}}

\newcommand{\infr}[2]{
{\renewcommand{\arraystretch}{1.1}
\begin{array}{c}
{#1}\\
\hline
{#2}
\end{array}}}

\providecommand{\eqref}[1]{\textup{(\ref{#1})}}
\providecommand{\eqdef}{\raisebox{-.2ex}[.2ex]{$\stackrel{\textrm{\tiny def}}{~=~}$}}
\newcommand{\gl}{\boldsymbol}
\newcommand{\game}[1]{\ensuremath{\mathsf{#1}}}
\newcommand{\tactic}{\texttt}
\newcommand{\fail}{F}
\newcommand{\bad}{\mathbf{bad}}
\newcommand{\result}{\mathsf{res}}
\newcommand{\guess}[1]{\tilde{#1}}
\newcommand{\challenge}[1]{\gl{#1^*}}
\newcommand{\uptobad}[2]{\mathsf{uptobad}({#1},{#2})}
\providecommand{\implies}{\Longrightarrow}
\newcommand{\cnt}{\mathsf{cntr}}
\newcommand{\qG}{q_\mathsf{G}}
\newcommand{\qH}{q_\mathsf{H}}
\newcommand{\qS}{q_\mathsf{S}}
\newcommand{\qD}{q_\Dec}

%% Names

\newcommand{\CertiCrypt}{\textsf{CertiCrypt}\xspace}
\newcommand{\CertiPriv}{\textsf{CertiPriv}\xspace}
\newcommand{\ProVerif}{\textsf{ProVerif}\xspace}
\newcommand{\CryptoVerif}{\textsf{CryptoVerif}\xspace}
\newcommand{\Coq}{\textsf{Coq}\xspace}
\newcommand{\CIL}{\textsf{CIL}\xspace}
\newcommand{\pWHILE}{\textsf{p}\textsc{While}\xspace}
\newcommand{\Why}{\textsf{Why}\xspace}
\newcommand{\altergo}{\textsf{alt-ergo}\xspace}
\newcommand{\simplify}{\textsf{Simplify}\xspace}

%% Security properties and schemes

\newcommand{\INDCPA}{\textsf{IND-CPA}\xspace}
\newcommand{\INDCCAone}{\textsf{IND-CCA1}\xspace}
\newcommand{\INDCCA}{\textsf{IND-CCA}\xspace}
\newcommand{\EFCMA}{\textsf{EF-CMA}\xspace}
\newcommand{\LCDH}{\ensuremath{\mathsf{LCDH}}\xspace}
\newcommand{\CDH}{\textsf{CDH}\xspace}
\newcommand{\DDH}{\textsf{DDH}\xspace}
\newcommand{\ElGamal}{\textsf{ElGamal}\xspace}
\newcommand{\HElGamal}{\textsf{HElGamal}\xspace}
\newcommand{\RSA}{\textsf{RSA}\xspace}
\newcommand{\OAEP}{\textsf{OAEP}\xspace}
\newcommand{\FDH}{\textsf{FDH}\xspace}
\newcommand{\HMAC}{\textsf{HMAC}\xspace}
\newcommand{\CS}{\textsf{CS}\xspace}
\newcommand{\Skein}{\textsf{Skein}\xspace}
\newcommand{\TCR}{\textsf{TCR}\xspace}

\newcommand{\A}{\mathcal{A}}
\newcommand{\B}{\mathcal{B}}
\newcommand{\C}{\mathcal{C}}
\newcommand{\D}{\mathcal{D}}
\newcommand{\Oracle}{\mathcal{O}}
\newcommand{\KG}{\mathcal{KG}}
\newcommand{\Enc}{\mathcal{E}}
\newcommand{\Dec}{\mathcal{D}}
\newcommand{\Sign}{\mathsf{Sign}}
\newcommand{\Verify}{\mathsf{Verify}}

%% Complexity and termination

\newcommand{\lossless}{\mathsf{lossless}}
\newcommand{\PPT}[1]{\mathsf{PPT}(#1)}
\newcommand{\PPTX}{\mathsf{PPT}}

\newcommand{\bounded}[1]{\mathsf{bounded}(#1)}

%% Sets

\newcommand{\zeroone}{[0,1]}
\newcommand{\bit}{\{0,1\}}
\newcommand{\bitstring}[1]{\ensuremath{\bit^{#1}}}
\newcommand{\zq}{\mathbb{Z}_q}
\newcommand{\bool}{\mathbb{B}}
\newcommand{\nat}{\mathbb{N}}
\newcommand{\real}{\mathbb{R}}
\newcommand{\option}[1]{#1_\bot}

%% Mathematics

\renewcommand{\Pr}[2]{\mathrm{Pr}\left[#1 : #2\right]}
\newcommand{\Prm}[3]{\mathrm{Pr}\left[#1,#3 : #2\right]}
\newcommand{\labs}{\left\lvert}
\newcommand{\rabs}{\right\rvert}
\newcommand{\charfun}{\mathds{1}}
\newcommand{\indicator}[1]{\mathbb{I}_{#1}}
\newcommand{\lub}{\sup}
\newcommand{\negl}{\mathsf{negl}}
\newcommand{\Adv}[2]{\mathbf{Adv}_{\mbox{\scriptsize $#1$}}^{\mbox{\scriptsize $#2$}}}
\newcommand{\Succ}[2]{\mathbf{Succ}_{\mbox{\scriptsize $#1$}}^{\mbox{\scriptsize $#2$}}}
\newcommand{\PER}{\mathsf{PER}}
\newcommand{\sym}{\mathsf{SYM}}

%% Distribution monad

\newcommand{\distr}{\mathcal{D}}
\newcommand{\unit}{\mathsf{unit}}
\newcommand{\bind}{\mathsf{bind}}
\newcommand{\supp}{\mathsf{support}}
\newcommand{\range}[2]{\mathsf{range}~{#1}~{#2}}
\newcommand{\lift}[3]{{#2}\,\mathcal{L}(#1)\,{#3}}

%% Semantics 

\newcommand{\sem}[1]{\llbracket #1 \rrbracket}
\newcommand{\subst}[2]{\left[{}^{#2}/{}_{#1}\right]}
\newcommand{\evalexpr}[1]{\sem{#1}_{\Expr}}
\newcommand{\evaldistr}[1]{\sem{#1}_{\DExpr}}
\newcommand{\eval}[1]{\sem{#1}}
\newcommand{\Proc}{\mathcal{P}}
\newcommand{\Var}{\mathcal{V}}
\newcommand{\Expr}{\mathcal{E}}
\newcommand{\DExpr}{\mathcal{DE}}
\newcommand{\Instr}{\mathcal{I}}
\newcommand{\Cmd}{\mathcal{C}}
\newcommand{\Mem}{\mathcal{M}}
\newcommand{\Decl}{\mathsf{decl}}
\newcommand{\Params}{\mathsf{params}}
\newcommand{\Body}{\mathsf{body}}
\newcommand{\RE}{\mathsf{re}}
\newcommand{\Global}{\mathsf{global}}
\newcommand{\Local}{\mathsf{local}}
\newcommand{\List}[1]{{#1}^*}
\newcommand{\fv}{\mathsf{fv}}
\newcommand{\modifies}{\mathsf{mod}}
\newcommand{\depends}{\mathsf{dep}}
\newcommand{\valid}{\mathsf{valid}}
\newcommand{\glob}{\mathsf{glob}}
\newcommand{\loc}{\mathsf{loc}}
\newcommand{\col}{\textsf{col}}

%% Well-formed adversaries

\newcommand{\Gadv}{\mathcal{RW}}
\newcommand{\Gcomm}{\mathcal{R}}
\newcommand{\Gorcl}{\mathcal{O}}
\newcommand{\GoodWrite}[1]{\mathsf{writable}(#1)}
\newcommand{\GoodRead}[2]{\fv(#2) \subseteq #1}
\newcommand{\GoodAdvc}[3]{#1 \vdash #2\!:\!#3}
\newcommand{\GoodAdv}[2]{#1 \vdash_{\mathrm{wf}} #2}

%% Program equivalence statements 

\newcommand{\Equiv}[4]{\models {#1} \sim {#2} : {#3} \Longrightarrow {#4}}
\newcommand{\AEquiv}[6]{\models {#2} \sim_{#5,#6} {#3} : {#1} \Longrightarrow {#4}}
\newcommand{\JAEquiv}[6]{{#2} \sim_{#5,#6} {#3} : {#1} \Longrightarrow {#4}}
\newcommand{\EquivMem}[2]{\models {#1} \equiv {#2}}
\newcommand{\EqObs}[4]{\models {#1} \simeq^{#3}_{#4} {#2}}
\newcommand{\AEqObs}[5]{\models {#1} \simeq^{#3}_{{#4}} {#2} \preceq {#5}} 
\newcommand{\ACEqObs}[7]
        {\AEqObs{\left[ #1 \right]_{#6}}{\left[ #2 \right]_{#7}}{#3}{#4}{#5}}
\newcommand{\Triple}[3]{\sem{#2} {#3} \preceq {#1}}
\newcommand{\DTriple}[3]{\sem{#2} {#3} \succeq {#1}}
\newcommand{\dequiv}[3]{{#1} \simeq_{#3} {#2}}
\newcommand{\fequiv}[3]{{#1} =_{#3} {#2}}
\newcommand{\Pre}{\Psi}
\newcommand{\Post}{\Phi}
\newcommand{\Inv}{\Phi}
\newcommand{\side}[1]{\langle #1 \rangle}
\newcommand{\sidel}{\side{1}}
\newcommand{\sider}{\side{2}}
\newcommand{\eqobsin}{\mathsf{eqobs\_in}}
\newcommand{\eqobsout}{\mathsf{eqobs\_out}}
\newcommand{\pre}{\Psi}
\newcommand{\post}{\Phi}

%% Variables

\newcommand{\LH}{\gl{L}_H}
\newcommand{\LD}{\gl{L}_\Dec}
\newcommand{\cdef}{\gl{\gamma_\mathsf{def}}}

%% Constants and operators
% TODO: Update!
\newcommand{\true}{\mathsf{true}}
\newcommand{\false}{\mathsf{false}}
\newcommand{\nil}{\mathsf{nil}}
\newcommand{\hd}{\mathsf{hd}}
\newcommand{\tl}{\mathsf{tl}}
\newcommand{\app}{\mathbin{+\mkern-7mu+}}
\newcommand{\concat}{\parallel}
\newcommand{\xor}{\oplus}
\newcommand{\msb}[2]{[#1]^{#2}}
\newcommand{\lsb}[2]{[#1]_{#2}}
\newcommand{\dom}{\mathsf{dom}}
\newcommand{\ran}{\mathsf{ran}}
\newcommand{\fst}{\mathsf{fst}}
\newcommand{\snd}{\mathsf{snd}}
\newcommand{\some}[1]{#1}
\newcommand{\none}{\bot}

%% Language
% TODO: Update!
\newcommand{\Skip}{\mathsf{skip}}
\newcommand{\Seq}[2]{#1;\ #2}
\newcommand{\Ass}[2]{#1 \leftarrow #2}
\newcommand{\Rand}[2]{#1 \stackrel{\raisebox{-.25ex}[.25ex]%
{\tiny $\mathdollar$}}{\raisebox{-.2ex}[.2ex]{$\leftarrow$}} #2}
\newcommand{\Randi}[2]{\Rand{#1}{[0..#2]}}
\newcommand{\Randb}[1]{\Rand{#1}{\bit}}
\newcommand{\Randbs}[2]{\Rand{#1}{\bitstring{#2}}}
\newcommand{\Cond}[3]{\mathsf{if}\ #1\ \mathsf{then}\ #2\ \mathsf{else}\ #3}
\newcommand{\Condt}[2]{\mathsf{if}\ #1\ \mathsf{then}\ #2}
\newcommand{\Else}{\mathsf{else}\ }
\newcommand{\Elsif}{\mathsf{elsif}\ }
\newcommand{\nWhile}[3]{\mathsf{while}_{#1}\ #2\ \mathsf{do}\ #3}
\newcommand{\While}[2]{\mathsf{while}\ #1\ \mathsf{do}\ #2}
\newcommand{\Call}[3]{#1 \leftarrow #2\mathsf{(}#3\mathsf{)}}
\newcommand{\Return}{\mathsf{return}}
\newcommand{\Assert}[1]{\mathsf{assert}~#1}

%% Language definition
\lstnewenvironment{easycrypt}[2][]%
  {\lstset{language=easycrypt,caption=#2,#1}}%
  {}

\newcommand{\rawec}[2][]{\lstinline[language=easycrypt,#1]{#2}}
\newcommand{\ec}[2][]{\lstinline[language=easycrypt,style=easycrypt-pretty,#1]{#2}}

\lstdefinelanguage{easycrypt}{
  style=easycrypt-default,
  procnamekeys={op,pred,fun},
  procnamestyle={\sffamily\itshape},
  morekeywords=[1]{theory,end},
  morekeywords=[2]{type,op,axiom,lemma,module,pred,lambda},
  morekeywords=[3]{var,fun},
  morekeywords=[4]{while,if},
  morekeywords=[5]{bool,int,real,bitstring,array,list,matrix,word},
  morekeywords=[6]{forall,exists},
%  moredirectives={prover,print}, % Incomplete
  morecomment=[n][\itseries]{(*}{*)},
  morecomment=[n][\bfseries]{(**}{*)}
}

\lstdefinestyle{easycrypt-default}{
  columns=fullflexible,
  captionpos=b,
  frame=tb,
  xleftmargin=.1\textwidth,
  xrightmargin=.1\textwidth,
  basicstyle=\small\sffamily,
  identifierstyle={},
  keywordstyle=[1]{\bfseries},
  keywordstyle=[2]{\bfseries},
  keywordstyle=[3]{\bfseries},
  keywordstyle=[4]{\bfseries},
  keywordstyle=[5]{\itshape},
  keywordstyle=[6]{\bfseries}
}

\lstdefinestyle{easycrypt-pretty}{
    basicstyle=\small\sffamily,
    literate={:=}{{$\mathrel{\gets}$}}1
              {<=}{{$\binop{\leq}$}}1
              {>=}{{$\geq$}}1
              {=\$}{{$\stackrel{\$}{\gets}$}}1
              {forall}{{$\forall$}}1
              {exists}{{$\exists$}}1
              {->}{{$\rightarrow\;$}}1
              {=>}{{$\Rightarrow\;$}}1
              {==>}{{$\Rrightarrow\;$}}1
              {\/\\}{{$\wedge$}}1
              {\\\/}{{$\vee$}}1
              {.\[}{{[}}1
              {'a}{{$\alpha\,$}}1
              {'b}{{$\beta\,$}}1
              {'c}{{$\gamma\,$}}1
              {'x}{{$\chi\,$}}1
              {lambda}{{$\lambda\,$}}1
}

%% Typesetting
\newcommand{\titledbox}[4]{{\color{#1}\fbox{\begin{minipage}{#2}{\textbf{#3:} \color{black}#4}\end{minipage}}}}
\newcommand{\warningbox}[1]{\titledbox{red}{.9\textwidth}{Warning}{#1}}

%%% Local Variables: 
%%% mode: latex
%%% TeX-master: "easycrypt"
%%% End: 

\newcommand{\rwp}{\textsc{wp}\xspace}
\makeindex

\begin{document}

\thispagestyle{empty}

\begin{center}

% \rule\textwidth{0.8mm}

\vfill

{\fontsize{40}{80pt}\selectfont\bfseries\sffamily\sc The \EasyCrypt tool}

\vfill

% \rule\textwidth{0.8mm}

\vfill

{\fontsize{20}{20pt}\selectfont\sffamily Documentation and User's Manual}

\vfill

\begin{LARGE}
  Version 0.2,  \today 
\end{LARGE}

\vfill

\begin{Large}
  \begin{tabular}{c}
  Gilles Barthe$^{1}$ \\
  Benjamin Gregoire$^{2}$  \\
  Juan Manuel Crespo$^{1}$ \\
  Cesar Kunz$^{1,4}$\\
  Santiago Zanella Beguelin$^{3}$
\end{tabular}
\end{Large}
\vfill

\begin{flushleft}

\begin{tabular}{l}
$^1$ IMDEA Software Institute, Spain \\
$^2$ INRIA Sophia Antipolis, France \\
$^3$ Microsoft Research, United Kingdom \\
$^4$ Universidad Polit\'ecnica de Madrid, Spain \\
\end{tabular}

\bigskip

  % \textcopyright ....

  % This work has been partly supported by ....

\end{flushleft}
\end{center}

\chapter*{Foreword}

This is the manual for the \EasyCrypt framework for computer-aided
cryptographic proofs. \EasyCrypt is an automated tool that supports
the machine-checked construction and verification of security proofs
of cryptographic systems, and that can be used to verify public-key
encryption schemes, digital signature schemes, hash function designs,
and block cipher modes of operation.


\subsection*{Availability}

\EasyCrypt web page can be found at
\url{http://http://easycrypt.gforge.inria.fr/}. Instructions for
accessing the source code, documentation, and examples can be found
there, together with contact information and recent publications.

See the file \texttt{README} for installation instructions.


\subsection*{Contact}

There is a public mailing list for users' discussions: 
\begin{quote}
\url{http://lists.gforge.inria.fr/mailman/listinfo/easycrypt-club}.
\end{quote}

\noindent
Report any bug to the \EasyCrypt Bug Tracking System:
\begin{quote}
\url{https://gforge.inria.fr/tracker/?atid=8938&group_id=2622&func=browse}
\end{quote}

\subsection*{Acknowledgements}

We gratelly thank the people who contributed to \EasyCrypt: 
Guido Genzone, 
Daniel Hedin, 
Sylvain Heraud, 
Anne Pacalet.



%%% Local Variables: 
%%% mode: latex
%%% TeX-master: "easycrypt"
%%% End: 

\tableofcontents

% \section{Requirements}

We only list the versions of required software that are known to
work. Note that EasyCrypt may still compile and work as expected using
versions of packages other than those listed.

To compile EasyCrypt and run the examples you will need:

\begin{itemize}
\item GNU Automake

\item GNU Make 3.81

  Available at \url{http://www.gnu.org/software/make/}
  Version 3.82 will most probably work

\item Objective Caml >= 3.11
 
  Available at \url{http://caml.inria.fr/download.en.html}
  Older versions >= 3.08 will most probably work

\item Why3 0.71
 
  Install the version provided in the repository at trunk/why3-0.71
  Patched from the version available at \url{http://why3.lri.fr/}

\item CVC3 2.4.1

  Available at \url{http://www.cs.nyu.edu/acsys/cvc3/}

\item Alt-Ergo 0.94
 
  Available at \url{http://alt-ergo.lri.fr/}

The following automated theorem provers are supported by EasyCrypt,
but are not needed to reproduce the case studies:

\item Z3 

  Available at
  \url{http://research.microsoft.com/en-us/um/redmond/projects/z3/download.html}

\item Simplify

  Pre-compiled binaries for various architectures are available at
  \url{http://krakatoa.lri.fr/ws/Simplify-1.5.5-13-06-07-binary.zip}

\item Yices

  Available at \url{http://yices.csl.sri.com/}

\item Eprover

  Available at \url{http://www4.informatik.tu-muenchen.de/~schulz/E/}

\item Vampire

  Available at \url{http://www.vprover.org/}

\end{itemize}

To install the ProofGeneral front-end for EasyCrypt you will
additionally need:

\begin{itemize}
\item GNU Emacs 23.2
 
  Available at \url{http://www.gnu.org/software/emacs/}

\item  ProofGeneral 4.1

  Available at \url{http://proofgeneral.inf.ed.ac.uk/}
\end{itemize}


\section{Installing Why3}

To compile EasyCrypt you need to install the byte-compiled version of
the Why3 library. Follow first the standard installation instructions 
in the corresponding Why3 README file. If you do not plan to use
Why3 as a standalone tool it is recommended to invoke the Why3 configure
script with the --disable-ide option to avoid unnecessary library 
dependences:
\begin{verbatim}
  ./configure --disable-ide
  make
  make install
\end{verbatim}
After installing Why3 from source code, you must type

\begin{verbatim}
 make byte
 make install-lib
\end{verbatim}

to install the library.

Once you have installed Why3 and the automated provers of your choice,
please make sure that Why is correctly configured to use the provers
by running the command

\begin{verbatim}
 why3config --detect
\end{verbatim}

If everything is correct, you should see a table detailing the provers
that Why3 detected---you can safely ignore any "not know to be
supported" warnings. Remember that you need at least CVC3 and Alt-Ergo
to reproduce the case studies.


\section{Copmpilation}

When the contents of the package were extracted, you should have ended
up with a directory containing this README file and a sub-directory
"easycrypt". ("easycrypt/trunk" when installing from the SVN repository). 

To compile EasyCrypt, simply change to the sub-directory "easycrypt" and type
 
\begin{verbatim}
 ./configure --with-proof-general=PATH_TO_PROOFGENERAL
\end{verbatim}
The argument above is optional but recommended. For a list of additional 
options type

\begin{verbatim}
 ./configure --help
\end{verbatim}

Then type

\begin{verbatim}
 make
\end{verbatim}

and then 

\begin{verbatim}
 make install
\end{verbatim} 

with the appropriate access permissions.

If everything goes well, a binary named "easycrypt.top" will be
generated. To test the setup you may then run

\begin{verbatim}
 make test
\end{verbatim}

and verify that all tests pass.


\section{Running the examples}

Several examples are available under the directory
"easycrypt/examples". To compile them from the directory "easycrypt",
simply type

\begin{verbatim}
 ./easycrypt examples/elgamal.ec
 ./easycrypt examples/helgamal.ec
 ./easycrypt examples/fdh.ec
\end{verbatim}

\section{Installing the ProofGeneral front-end}

\subsection{ProofGeneral - Requirements}

\begin{itemize}
\item Proof general >=  4.1

  Available at http://proofgeneral.inf.ed.ac.uk/download

\item CertiCrypt

\end{itemize}

\subsection{ProofGeneral - Manual Installation}

Add the following line to <proof-general-home>/generic/proof-site.el
in the definition of `proof-assistant-table-default':

\begin{verbatim}
   (certicrypt "CertiCrypt" "ec" nil (".v" ".vo" ".glob" ".ml"))
\end{verbatim}

Copy the directory "certicrypt" and its contents to the directory
where ProofGeneral was installed, typically

\begin{verbatim}
   /usr/local/share/emacs/site-elisp/ProofGeneral/
\end{verbatim}

and check that the final directory has the appropriate access permissions.

\subsection{ProofGeneral - Automatic Installation}

The provided Makefile will install everything in default
locations (or the location specified to the ./configure script). Simply type

\begin{verbatim}
   make install_proofgeneral
\end{verbatim}

or 

\begin{verbatim}
   sudo make install_proofgeneral
\end{verbatim}

as appropriate.


\subsection{ProofGeneral - Configuration}

Add the following line to your emacs configuration file (typically ~/.emacs):

\begin{verbatim}
 (load-file "/usr/share/emacs/site-lisp/proofgeneral/generic/proof-site.el")
\end{verbatim}

Set the path to the EasyCrypt executable and the prelude file in your
Emacs configuration. This can be achieved either by modifying the
variable certicrypt-prog-name inside Emacs:

\begin{verbatim}
 Proof-General
   -> Advanced 
     -> Customize 
       -> Certicrypt 
         -> CertiCrypt prog name
\end{verbatim}

You should set its value to (modifying paths as appropriate):

\begin{verbatim}
 "<path-to-easycrypt>/easycrypt -emacs -prelude <path-to-prelude>/easycrypt_base.ec"
\end{verbatim}

The prelude file "easycrypt-base.ec" can be found at "easycrypt/src/".

Alternatively, you can modify your Emacs local configuration file
(typically ~/.emacs):

\begin{verbatim}
 (custom-set-variables
 ...
  '(certicrypt-prog-name 
    "<path-to-easycrypt>/easycrypt -emacs
        			   -prelude <path-to-prelude>/easycrypt_base.ec)
 ...)
\end{verbatim}





%%% Local Variables: 
%%% mode: latex
%%% TeX-master: "easycrypt"
%%% End: 


\part{An introduction to \EasyCrypt}
  
\chapter{\EasyCrypt language}


\section{Basic declarations}
\paragraph*{Types, constants, operators.}
\index{types}\index{constants}\index{operators}

\EasyCrypt provides native basic types such as \verb|unit|,
\verb|bool|, \verb|int|, \verb|real|, \verb|bitstring| as well as
polymorphic lists \verb|list|, polymorphic maps \verb|map|, product
types \verb!*! (infix notation), and \verb|option| types.
%
Abstract types can be declared with statements of the form 
\verb+type+~\textit{type_ident}, as in the following example:
\begin{verbatim} 
type secret_key.
type group.
\end{verbatim} 
Parametric type declarations are also supported. Type variables start
with a \verb|'| symbol:
\begin{verbatim}
type 'a list.
\end{verbatim}
%
Types synonyms can be declared with declaration of the form 
\verb+type+~\textit{type_ident}~\verb+=+~\textit{type_exp},
where \textit{type_exp} is built from basic types, type instantiation,
and other user-declared types, as in the following example:
\begin{verbatim} 
type secret_key = int.
type pkey = group.
type ciphertext = group * group. 
\end{verbatim} 

Constants are introduced with declarations of the form
\verb+cnst+~\textit{ident}\verb+:+~\textit{type_exp}~[\textit{exp}],
where \textit{exp} is an optional expression defining the constant.
For example, the following declarations introduce constants with
identifiers {\tt g} and {\tt empty_map} of types {\tt group} and {\tt
  ('a, 'b) map}, respectively:
\begin{verbatim}
cnst g : group.
cnst empty_map : ('a, 'b) map.
\end{verbatim}

Operators are introduced with declarations of the form
\verb+op+~\textit{op_ident}~\verb+:+~\textit{fun_type}~[\verb+as+~\textit{id}]
where the operator \textit{op_ident} can be either an alpha-numerical
identifier or a binary operator ---which may include extra symbols
such as \verb'=', \verb'<', \verb'~', \verb'+', \verb'%', and \verb'^'
for example--- enclosed in square backets.  The identifier
\textit{gt_int} is required when defining a binary operator enclosed
in brackets, and is used as an internal identifier following the
syntactic conventions of the tools in which \EasyCrypt relies.  The
signature \textit{fun_type} is defined with the syntax
\textit{type_exp}~\verb+->+~\textit{type_exp}, or
\verb+(+\textit{type_exp}${}_1$\verb+,+...\verb+,+\textit{type_exp}${}_k$\verb+)+~\verb+->+~\textit{type_exp},
where \textit{type_exp} stands for type expressions and 
\textit{type_exp}${}_1$\verb+,+...\verb+,+\textit{type_exp}${}_k$ is a
possibly empty list of type expressions.
%
For example:
\begin{verbatim}
op exp : real -> real
\end{verbatim}
The first operator is declared as infix and denoted by the symbol
\verb|>|. The operator \verb|exp| is a prefix operator. 
%
The definition of polymorphic operators is also allowed by the use of
type variables, e.g., the \verb+hd+ operator defined in the
\EasyCrypt prelude:
\begin{verbatim}
op hd : 'a list -> 'a.
\end{verbatim}
%
As well as constants, operators can be defined by an expression using
the following syntax:
\\
\verb+op+~\textit{op_ident}\verb+(+\textit{params}\verb+) = +\textit{exp}~[\verb+as+~\textit{id}]
\\
\noindent
notice that the result type is not required in this case.
The following are examples of operators defined in the \EasyCrypt prelude:
\begin{verbatim}
op fst(c : 'a * 'b) = let a,b = c in a.
op [>] (x,y:int) = y < x as gt_int.
\end{verbatim}


\paragraph*{Probabilistic operators.}\index{probabilistic operators}
Probability distributions (see random samplings in the definition of
probabilistic statements) can be defined by declaring operators with
the syntax \verb+pop+~\textit{ident}~\verb+:+~\textit{fun_type} where,
as well as in the definition of deterministic operators, the function
signature \textit{fun_type} is defined with the syntax
\textit{type_exp}~\verb+->+~\textit{type_exp}, or
\verb+(+\textit{type_exp}${}_1$\verb+,+...\verb+,+\textit{type_exp}${}_k$\verb+)+~\verb+->+~\textit{type_exp},
where \textit{type_exp} stands for type expressions and
\textit{type_exp}${}_1$\verb+,+...\verb+,+\textit{type_exp}${}_k$ is a
possibly empty list of type expressions.  For example:
\begin{verbatim}
pop gen_secret_key : int -> secret_key.
\end{verbatim}

\paragraph*{Logical formulae.}
Formulae are built from boolean expressions, standard logical
connectives, defined predicates, and logical variable
quantification. Boolean expressions are built by the application of
native or user-defined operators.

Logical formulae must be closed with respect to logical variables. The
syntax for universal quantification is of the form:
\begin{verbatim}
 forall (x,y:int,z:real), p(x,y,z)
\end{verbatim}
where \verb|p| is a first-order formula and \verb|x,y,z| are logical
variables, and similarly with existential quantification (\verb+exists+).

In addition to logical variables, in some contexts, predicates may
contain program variables tagged with a \verb|{1}| or \verb|{2}|
flag. A formula defining an axiom must contain only logical variables,
whereas formulae describing pre and postconditions on a relational
judgment (discussed below) usually refers to tagged program variables.


The special notation to specify that the states on the left and right
are equal over a subset of variables. For example, one can write
\verb|={x,y,z}| to denote the equivalent relational predicate
\begin{verbatim}
x{1}=x{2} && y{1}=y{2} && z{1}=z{2}
\end{verbatim}
% (the special keyword \verb|res| refers to function return value) 



\paragraph*{Predicates.}\index{predicates}
Predicates are introduced with the syntax
\verb+pred+~\textit{ident}\verb+(+\textit{params}\verb+)=+~\textit{p}
where \textit{params} is a list of formal argument declarations and
\textit{p} is a first-order non-relational formula. For example:
\begin{verbatim}
pred injective(T:('a, 'b) map) = 
  forall (x,y:'a), in_dom(x,T) => in_dom(y,T) => T[x] = T[y] => x = y.
\end{verbatim}



\paragraph*{Axioms and Lemmas.}\index{axioms}\index{lemmas}

Axioms are used to describe properties of abstract operators and
types, or to introduce hypotheses over declared constants. Axioms are
defined by a declaration of the form
\verb+lemma+~\textit{ident}~\verb+:+~\textit{p}, where \textit{ident}
is a valid identifier and $p$ is a first-order non-relational formula.
For example:
\begin{verbatim}
axiom head_def : forall (a: 'a, l: 'a list),  hd(a::l) = a.

axiom empty_in_dom : forall (a:'a), !in_dom(a, empty_map).
\end{verbatim}
%
The axiom \verb+head_def+ defines the list operator \verb+hd+. 
The axiom \verb+empty_in_dom+ characterizes \verb+empty_map+ as a
map with an empty domain.

Lemmas can also be introduced to facilitate the verification of later
goals. The syntax is similar to the one of axioms:
\verb+lemma+~\textit{ident}~\verb+:+~\textit{p}, where $p$ is a
first-order non-relational formula. 
%
When a lemma statement is found, \EasyCrypt proves it by calling the
available provers/SMT provers through the Why3 tool.


\section{Game declarations}
Games are defined by three components: variables describing the global
state, defined procedures and abstract adversary declarations.


\subsection{Probabilistic statements.}

Statements are defined as a list, possible empty, of basic
instructions (assignments and function calls) ending on a semicolon,
or composed instructions (conditional and while loops). No semicolon
is accepted after a conditional or loop
statement. Conditional statements follow the syntax 
%
\verb+if (+\textit{b}\verb+) {+ \textit{stmt} \verb+}+ where \textit{stmt} is
a probabilistic statement and \textit{b} is a boolean guard.  While
loop statements follow the syntax \verb+while (+\textit{b}\verb+) {+
  \textit{stmt} \verb+}+. Curly brackets are not required when
\textit{stmt} contains a single instruction.

Probabilistic assignments are of the form 
%
\verb+ident+ \verb+=+ \textit{d_exp}
%
where \textit{d_exp} is a probability expression, such as uniform
distributions over booleans (\verb+{0,1}+), integer intervals
\verb+[i..j]+, and bitstrings of arbitrary length
(\verb+{0,1}^k+), or distributions defined in terms of probabilistic
operators.  Assume \verb|gen_secret_key : int -> secret_key| is a
defined probabilistic operator, the following are valid probabilistic
assignments:
\begin{verbatim}
x = {0,1}
x = [0..q-1]
x = {0,1}^k
x = gen_secret_key(0)
\end{verbatim}


\subsection{Function Definition.}

Functions are defined either by a function body containing variable
declarations and probabilistic statements or as synonyms of functions
of already defined games.

\begin{itemize}
\item 
\verb+fun+ \textit{fun_ident} \verb+(+%
\textit{typed_args}\verb+) : +\textit{ret_type} \verb+ = { +%
\textit{fun_body} \verb+}+

\textit{fun_ident} is a valid function identifier, a list of typed
formal parameters \textit{typed_args}, the return type
\textit{ret_type} and its body \textit{fun_body}. The function body is
defined as a list of local variable declarations of the form
%
\verb+var+ \textit{ident} \verb+:+ \textit{type}\verb+;+, a
%
probabilistic statement, and a return instruction of the form
\verb+return+ \textit{exp}, where \textit{exp} is a deterministic
expression.

\item
\verb+fun+ \textit{fun_ident} \verb+=+ \textit{game_ident}\verb+.+\textit{fun_ident}

The resulting function has the same formal parameters and function
body than the function on the right.
\end{itemize}


\subsection{Adversary Signature and Declaration.}
Adversary signatures are defined outside a game declaration with a
syntax of the form:

\verb+adversary+
\textit{adv_sign_ident}\verb+(+\textit{typed_args}\verb+) :+
\textit{res_type} \verb+{+\textit{o_sign}${}_1$\verb+,+...\verb+,+\textit{o_sign}${}_k$\verb+}.+

\noindent
where \textit{res_type} is a type expression specifying the return
type and \textit{o_sign}${}_1$\verb+,+...\verb+,+\textit{o_sign}${}_k$
is a list (possibly empty), of oracle signatures.
In the following example
\begin{verbatim}
adversary A1_sign(pk:pkey)  : message * message { group -> message}.
adversary A2_sign(c:cipher) : bool              { group -> message}.
\end{verbatim}
the type expressions \verb|message*message| and \verb|bool| indicate
the return type. A list of signatures in square brackets indicates the
signature of the oracles that can be invoked by adversaries with these
signatures. In this particular example both signatures belong to
adversaries that can invoke a single oracle with type
\verb|group -> message|.

As well as function definition, adversaries are either declared abstractly
or as adversary synonyms.
Abstract declarations follow the syntax:\\
\verb+abs+ \textit{adv_ident} \verb+=+ \textit{adv_sign_ident}
\verb+{+ \textit{ident}${}_1$\verb+,+...\verb+,+\textit{ident}${}_k$\verb+}+

\noindent
For the adversary signature above we can write for example:
\begin{verbatim}
  abs A1 = A1_sign {H_A}
  abs A2 = A2_sign {H_A}
\end{verbatim}
where \verb|H_A| is a defined function representing an
oracle. Clearly, \EasyCrypt requires the function \verb|H_A| to have
the signature \verb|group -> message|.

Adversary synonyms follow a similar syntax to function synonyms:

\verb+fun+ \textit{adv_ident} \verb+=+ \textit{game_ident}\verb+.+\textit{adv_ident}

\noindent
The result of this declaration is, however, not necessarily an
abstract adversary.

\subsection{Game definition}


\begin{itemize}
\item A game can be defined by the following syntax:  
  \Syntax
  \verb+game+ \textit{ident} \verb+=+ \verb+{+\textit{game_body}\verb+}+
  % 
  The body of a game \textit{game_body} is composed of a global
  variable declaration, function definitions and abstract adversary
  declarations. The declaration of global variables consists of a list
  of statements of the form
  % 
  \verb+var+ \textit{ident} \verb+:+ \textit{type} 
  % 
  as in the definition of function local variables, except that they
  are not separated by a semicolon.
\item Alternatively, one can redefine a game by removing or adding
  variables, and redefining functions from an already defined game.
  \Syntax \verb+game+ \textit{ident} \verb+=+
  \textit{g_ident} \textit{var_modifs} \\
  \verb+           where+ \textit{ident${}_1$} \verb+= {+ \textit{fun_body} \verb+} and+ ...
  \verb+and+ \textit{ident${}_k$} \verb+= {+ \textit{fun_body} \verb+}+.

  The \textit{g_ident} identifier refers to an existing game,
  \textit{var_modifs} consists of an optional statement of the form
  \verb+remove+
  \textit{ident${}_1$}\verb+,+..\verb+,+\textit{ident${}_k$} and a
  possible empty list of new variable declarations. Finally, a list of
  function redefinitions is given separated by the \verb+and+ keyword.
\end{itemize}





\chapter{Probabilistic Relational Hoare Logic}

\section{Foundations}
Probabilistic Relational Hoare Logic (pRHL) judgments are quadruples
of the form:
%
$$ \Equiv{c_1}{c_2}{\Pre}{\Post} $$
%
where $c_1, c_2$ are programs and $\Pre, \Post$ are first-order
relational formulae. Relational formulae are first-order formulae over
logical variables and program variables tagged with either \verb|{1}|
or \verb|{2}| to denote their interpretation in the left or right-hand
side program. The special keyword \verb|res| denotes the return value
of a procedure and can be used in the place of a program variable. One
can also write \verb|e{i}| for the expression |e| in which all program
variables are tagged with \verb|{i}|. A relational formula is
interpreted as a relation on program memories.  See the related
articles~\cite{Barthe:2009} for more information on this logic.

\section{Judgements}
In \EasyCrypt, pRHL judgments are introduced with judgments
of the form
\begin{verbatim}
equiv Fact : Game1.f1 ~ Game2.f2 : Pre ==> Post.
\end{verbatim}
where \verb|Fact| is a judgment identifier, \verb!Game1! and
\verb!Game2! are games, \verb!f1! and \verb!f2! are identifiers for
procedures in \verb!Game1! and \verb!Game2! respectively. The
procedures \verb!f1! and \verb!f2! may be abstract or concrete;
however, judgments between two abstract procedures can only be defined
only if the two abstract procedures correspond to the same adversary.

The pre-condition \verb!Pre! and post-condition \verb!Post! are
relational formulae, and define relations between the parameters and
the global variables of the two procedures, the post-condition is a
relation between the global variables and a special variable named
\verb+res+, representing the return value of the procedures.  More
precisely \verb+res{1}+ stands for return value of the left procedure
and \verb+res{2}+ stands for the return value of the right
procedure. For convenience, \EasyCrypt also allows pre-conditions and
post-conditions to include sub-formulae of the form
\verb!={x1, ..., xn}! stating that the values of \verb!x1 ... xn!
  coincide in the left and right memories. That is,
  \verb!={x1, ..., xn}! is a shorthand for
  \verb!x1{1}=x1{2} && ... && xn{1}=xn{2}!.

\EasyCrypt also supports judgments of the form:
\begin{verbatim}
equiv Fact : Game1.f1 ~ Game2.f2 : (Inv).
\end{verbatim}
as a shorthand for 
\begin{verbatim}
equiv Fact : Game1.f1 ~ Game2.f2 : ={params} && Inv  ==>  ={res} && Inv.
\end{verbatim}
where \verb!params! is the list of parameters of \verb!f1! and
\verb!f2!. Note that in order for the judgment to be meaningful, the
procedures must have the same return type and the same signature type.






\section{Proof process}
A statement of the form 
\begin{verbatim}
equiv Fact : G1.f1 ~ G2.f2 : Pre ==> Post.
\end{verbatim}
opens a verification process, provided \verb!f1! and \verb!f2! are
both abstract procedures, or both concrete procedures. 


In case \verb!f1! and \verb!f2! are both abstract procedures, the only
available tactic is \verb!auto!. Note that, since abstract procedures
are allowed to call concrete procedures, it is sometimes useful to
prove invariants on the latter prior to proving equivalence properties
on \verb!f1! and \verb!f2!.


In case both procedures \verb!f1!  and \verb!f2! are concrete,
\EasyCrypt automatically transforms the judgment into a judgment on
their bodies. The pre-condition remains unchanged, but the
post-condition is modified by replacing the variables \verb+res{1}+
and \verb+res{2}+ by the return expressions of \verb!f1! and \verb!f2!
respectively.

For example, in the file \verb+examples/elgamal.ec+ after the
definition of the game \verb+DDH0+ we can start a new judgment,
stating that the two procedures \verb!INDCPA.Main! and
\verb! DDH0.Main! are equivalent if we observe their results
(\verb+={res}+ stands for \verb+res{1} = res{2}+):
\begin{verbatim}
equiv CPA_DDH0 : INDCPA.Main ~ DDH0.Main : true ==> ={res}.
\end{verbatim}
The judgment is automatically transformed into the following goal:
\begin{verbatim}
pre   = true
stmt1 =   1 : (sk, pk) = KG ();
          2 : (m0, m1) = A1 (pk);
          3 : b = {0,1};
          4 : mb = if b then m0 else m1;
          5 : c = Enc (pk, mb);
          6 : b' = A2 (pk, c);
stmt2 =   1 : x = [0..q - 1];
          2 : y = [0..q - 1];
          3 : d = B (g ^ x, g ^ y, g ^ (x * y));
post  = (b{1} = b'{1}) = d{2}
\end{verbatim}
At this point, the \EasyCrypt interpreter expects the user to provide
tactics to guide the verification of the judgment. Each tactic may
generate both logical verification goals (first-order formulae) that
are sent to SMT solvers and new verification subgoals that are stacked
for later verification by the user. The interactive verification task
concludes when there are no more goals in the stack and the result is
\emph{saved} (by typing \verb|save|) or when the verification goal is
\emph{aborted}.
%


Note that we have not implemented support to reason about the case
where one procedure is abstract, and another concrete. One possible
workaround is to wrap the abstract procedure, say \verb!f1!, into
a concrete procedure \verb!f1c! that simply calls \verb!f1!.  



 




\section{Tactics}
% --------------------------------------------------------------------
% --------------------------------------------------------------------
\begin{tactic}{admit}
\end{tactic}

% --------------------------------------------------------------------
\begin{tactic}{algebra}
  \begin{tsyntax}[empty]{algebra}
  \fix{Missing description of algebra}.
  \end{tsyntax}
\end{tactic}

% --------------------------------------------------------------------
\begin{tactic}{alias}
\end{tactic}

% --------------------------------------------------------------------
\begin{tactic}{apply}
\end{tactic}

% --------------------------------------------------------------------
\begin{tactic}{assumption}
  \begin{tsyntax}[empty]{assumption}
  Search in the context for a hypothesis that is convertible to the goal
  and apply. Fail if none can be found.
  \end{tsyntax}
\end{tactic}

% --------------------------------------------------------------------
\begin{tactic}{auto}
  \begin{tsyntax}[empty]{auto}
  \fix{Missing description of auto}.
  \end{tsyntax}
\end{tactic}

% --------------------------------------------------------------------
\begin{tactic}{beta}
\end{tactic}

% --------------------------------------------------------------------
\begin{tactic}{byequiv}

  \begin{tsyntax}{byequiv [option]? <specification>}
  Derives probability relation from \prhl judgements. 
  Only applies to judgments on procedures.
 
  \textbf{Examples:}
  \begin{mathpar}
    \inferrule*[left=(\prhl),rightskip=5em]%%
    { \pRHL{P}{f_1}{f_2}{Q} \\%
      P~\vec{a}_1~m_1~\vec{a}_2~m_2 \\%
      Q \Rightarrow E_1\{1\}  \Leftrightarrow E_2\{2\} }%%
    { \PR{f_1}{\vec{a}_1}{\mem{m_1}}{E_1} = \PR{f_2}{\vec{a}_2}{m_2}{E_2} }%%
    \quad\raisebox{.7em}{\tct{byequiv (: P ==> Q)} } \\
    \inferrule*[left=(\prhl),rightskip=5em]%%
    { \pRHL{P}{f_1}{f_2}{Q} \\% 
      P~\vec{a}_1~m_1~\vec{a}_2~m_2 \\%
      Q \Rightarrow E_1\{1\}  \Rightarrow E_2\{2\} }%%
    { \PR{f_1}{\vec{a}_1}{\mem{m_1}}{E_1} \leq \PR{f_2}{\vec{a}_2}{m_2}{E_2} }%%
    \quad\raisebox{.7em}{\tct{byequiv (: P ==> Q)} } \\
    \inferrule*[left=(\prhl),rightskip=5em]%%
    { \pRHL{P}{f_1}{f_2}{Q} \\%
      P~\vec{a}_1~m_1~\vec{a}_2~m_2 \\%
      Q \Rightarrow E_2\{2\}  \Rightarrow E_1\{1\} } %%
    { \PR{f_1}{\vec{a}_1}{\mem{m_1}}{E_1} \geq \PR{f_2}{\vec{a}_2}{m_2}{E_2} }%%
    \raisebox{.7em}{\tct{byequiv (: P ==> Q)} } 
  \end{mathpar}
 
 \end{tsyntax}

  Possible options are \tct{-eq} or \tct{eq}.
  Any one of the specification places can be filled
  with a wildcard \tct{_}. It that case the corresponding argument 
  is automatically inferred. Some time the infered postcondition  
  is stronger than necessary, in that case use the option \tct{-eq}.

  \fix{Missing description of byequiv for upto}.
  
  \begin{tsyntax}{byequiv <lemma>}
  Same as \tct{byequiv <specification>}, but the specification to use is 
  inferred from the lemma provided. Raises an error if the lemma does 
  not refer to the expected procedures. All variants of \tct{byequiv} 
  may take lemmas in place of explicit specifications with the same effect.
  \end{tsyntax}


\end{tactic}

% --------------------------------------------------------------------
\begin{tactic}{byphoare}
  \begin{tsyntax}{byphoare [option]? <spec>}
  Derives a probability relation from a \phl judgement on the
  procedure involved. \tct{<spec>} can include wildcards when the
  tactic should infer the pre or postcondition.

  \textbf{Options:} By default, (\tct{eq} option) specification
  inference attempts to infer a conjunction of equalities sufficient
  to imply the desired relation. Passing the \tct{-eq} option
  overrides this behaviour, instead using the trivial relation on
  events.

  \textbf{Examples:}
  \begin{mathpar}
    \inferrule%%
      {\pHL{P}{f}{Q}{=}{\delta} \\%
       \Pred{P}{m[\Arg\mapsto\vec{a}]} \\%
       \forall \mem{m'}.\,\Pred{Q}{m'} \Leftrightarrow \Pred{E}{m'}}%%
      {\PR{f}{\vec{a}}{\mem{m}}{E} = \delta}%%
      \quad\mbox{\parbox{200pt}{\tct{byphoare (_: P ==> Q)}}} \\
  \end{mathpar}
  \end{tsyntax}

  \begin{tsyntax}{byphoare <lemma>}
  Same as \tct{byphoare <spec>}, but the specification to use is
  inferred from the lemma provided. Raises an error if the lemma does
  not refer to the expected procedure. Inference options have no
  effect in this setting.
  \end{tsyntax}
\end{tactic}

% --------------------------------------------------------------------
\begin{tactic}{bypr}
  \begin{tsyntax}{bypr}
  Derives a program judgment from a probability relation or an exact
  probability. Only applies to judgments on procedures.

  \textbf{Examples:}
  \begin{mathpar}
    \inferrule*[left=(\prhl),rightskip=10em]%%
    {\forall m_1, m_2, a.\, E_1 = a \Rightarrow E_2 = a \Rightarrow Q~m_1~ m_2 \\%
     \forall \vec{a}_1, \vec{a}_2, m_1, m_2, a.\, P~\vec{a}_1~m_1~\vec{a}_2~m_2 \Rightarrow%
       \PR{f_1}{\vec{a}_1}{\mem{m_1}}{a = E_1} = \PR{f_2}{\vec{a}_2}{\mem{m_2}}{a = E_2}} %%
    {\pRHL{P}{f_1}{f_2}{Q}}%%
    \quad\raisebox{.7em}{\tct{bypr (E$_1$) (E$_2$)}} \\
  \inferrule*[left=(\phl),rightskip=10em]%%
    {\forall m, \vec{a}.\,P~\vec{a}~m \Rightarrow \PR{f}{\vec{a}}{m}{E} \mathrel{\diamond} \delta\{m\}}%%
    {\pHL{P}{f}{E}{\diamond}{\delta}}
    \quad\raisebox{.7em}{\tct{bypr}} \\
  \inferrule*[left=(\hl),rightskip=10em]%%
    {\forall m, \vec{a}.\,P~\vec{a}~m \Rightarrow \PR{f}{\vec{a}}{m}{\neg E} \mathop{=}0$\%$r}%%
    {\HL{P}{f}{E}}
    \quad\raisebox{.7em}{\tct{bypr}} \\
  \end{mathpar}
  \end{tsyntax}
\end{tactic}

% --------------------------------------------------------------------
\begin{tactic}{by}
\end{tactic}

% --------------------------------------------------------------------
\begin{tactic}{call}
  All variants of the \tct{call} tactic implicitly make use of a frame
  rule, based on a ``may modify'' analysis.

  \begin{tsyntax}{call (_: P ==> Q)}
  Compute the precondition of a procedure call using the given
  specification for the procedure. As a side-goal, prove that the
  procedure fulfills the given specification.

  As with other tactics, the specification \tct{(_: P ==> Q)} can be
  replaced with a lemma from which the specification is inferred.
  \end{tsyntax}

  \begin{tsyntax}{call (_: I)}
  Uses invariant \tct{I} to infer a specification for use with
  \tct{tactic}.
  %%
  In \prhl, equivalent to
  \tct{call (_: =$\{\Arg\}$ /\\ I ==> =$\{\Res\}$ /\\ I); first proc I.}
  %%
  In \phl and \hl, equivalent to
  \tct{call (_: I ==> I); first proc I.}
  \end{tsyntax}

  \begin{tsyntax}{call (_: B, I)}
  On \prhl abstract procedures only.
  Equivalent to \tct{call (_: $\neg$B /\\ =$\{\Arg\}$ /\\ I ==> $\neg$B => =$\{\Res\}$ /\\ I); first proc B I.}
  \end{tsyntax}

  \begin{tsyntax}{call (_: B, I, I')}
  On \prhl abstract procedures only.
  Equivalent to \tct{call (_: $\neg$B /\\ =$\{\Arg\}$ /\\ I ==> if $\neg$B then =$\{\Res\}$ /\\ I else I')); first proc B I I'.}
  \end{tsyntax}

  \textbf{Note:} When using the invariant-based variants of
  \tct{call}, error messages may be originating from the underlying
  application of \rtactic{proc}. In particular, when using them to
  deal with abstract procedure calls, the invariant \emph{should not}
  refer to memory locations the abstract procedure may modify.
\end{tactic}

% --------------------------------------------------------------------
\begin{tactic}[case $\;\phi$]{case}
  \begin{tsyntax}[empty]{case}
  Do an excluded-middle case analysis on $\phi$, substituting $\phi$
  in the goal.
  \end{tsyntax}

  \fixme{Describe the behaviour of \ec{case} on inductives.}
\end{tactic}

% --------------------------------------------------------------------
\begin{tactic}{cfold}
\end{tactic}

% --------------------------------------------------------------------
\begin{tactic}{change}
  \begin{tsyntax}[empty]{change}
  \fix{Missing description of change}.
  \end{tsyntax}
\end{tactic}

% --------------------------------------------------------------------
\begin{tactic}[clear $\;x_1 \cdots x_n$]{clear}
  \begin{tsyntax}[empty]{clear}
  Clear the local variables and hypotheses $x_1 \cdots x_n$ from the
  local context. Fail if any remaining hypotheses depend on any of the
  $x_i$.
  \end{tsyntax}
\end{tactic}

% --------------------------------------------------------------------
\begin{tactic}{congr}
  \begin{tsyntax}[empty]{congr}
  Replace a goal of the form \ec{f t$_1$ ... t$_n$ = f u$_1$ ... u$_n$}
  with the subgoals \ec{t$_i$ = u$_i$} for all \ec{$i$}. Subgoals solvable
  by \ec{reflexivity} are automatically closed.
  \end{tsyntax}
\end{tactic}

% --------------------------------------------------------------------
\begin{tactic}{conseq}
  \begin{tsyntax}{conseq <specification>}
  Rule of consequence. Proves a specification by weakening of a
  stronger result. Any one of the specification places can be filled
  with a wildcard \tct{_} to keep the value it contains in the current
  goal and trivially discharge the corresponding subgoal.

  \textbf{Examples:} In the following, $\leq^\uparrow$ (resp. $=^\uparrow$,
  $\geq^\uparrow$) is $\Leftarrow$ (resp. $\Leftrightarrow$ and
  $\Rightarrow$).
  \begin{mathpar}
  \inferrule*[left=(pRHL),rightskip=10em]%%
    {P' \Rightarrow P \\%
     Q \Rightarrow Q' \\%
     \pRHL{P}{c}{c'}{Q}}%%
    {\pRHL{P'}{c}{c'}{Q'}}%%
    \quad\raisebox{.7em}{\tct{conseq (_: P ==> Q)}} \\
  \inferrule*[left=(pHL),rightskip=10em]%%
    {P' \Rightarrow \delta \mathrel{\diamond} \delta' \\%
     P' \Rightarrow P \\%
     Q \mathrel{\diamond^\uparrow} Q' \\%
     \pHL{P}{c}{Q}{\diamond}{\delta}}%%
    {\pHL{P'}{c}{Q'}{\diamond}{\delta'}}%%
    \quad\raisebox{.7em}{\tct{conseq (_: P ==> Q: $\delta$)}} \\
  \inferrule*[left=(HL),rightskip=10em]%%
    {P' \Rightarrow P \\%
     Q \Rightarrow Q' \\%
     \HL{P}{c}{Q}}%%
    {\HL{P'}{c}{Q'}}%%
    \quad\raisebox{.7em}{\tct{conseq (_: P ==> Q)}} \\
  \end{mathpar}
  \end{tsyntax}

  \begin{tsyntax}{conseq* <specification>}
  Same as \tct{conseq <specification>}, but the subgoal corresponding
  to the postcondition is refined by a ``may modify'' analysis.
  \end{tsyntax}
\end{tactic}

% --------------------------------------------------------------------
\begin{tactic}[cut $\;\iota$: $\;\phi$]{cut}
  Same as \rtactic{have}.
\end{tactic}

% --------------------------------------------------------------------
\begin{tactic}{delta}
  \begin{tsyntax}[empty]{delta}
  \fix{Missing description of delta}.
  \end{tsyntax}
\end{tactic}

% --------------------------------------------------------------------
\begin{tactic}{done}
  \begin{tsyntax}[empty]{done}
  \fix{Missing description of done}.
  \end{tsyntax}
\end{tactic}

% --------------------------------------------------------------------
\begin{tactic}{eager}
  \begin{tsyntax}[empty]{eager}
  \fix{Missing description of eager}.
  \end{tsyntax}
\end{tactic}

% --------------------------------------------------------------------
\begin{tactic}{elim}
\end{tactic}

% --------------------------------------------------------------------
\begin{tactic}{exact}
  \begin{tsyntax}[empty]{exact}
  \fix{Missing description of exact}.
  \end{tsyntax}
\end{tactic}

% --------------------------------------------------------------------
\begin{tactic}{exfalso}
\end{tactic}

% --------------------------------------------------------------------
\begin{tactic}{fel}
  \begin{tsyntax}[empty]{fel}
  \fix{Missing description of fel}.
  \end{tsyntax}
\end{tactic}

% --------------------------------------------------------------------
\begin{tactic}{fieldeq}
\end{tactic}

% --------------------------------------------------------------------
\begin{tactic}{fission}
\end{tactic}

% --------------------------------------------------------------------
\begin{tactic}{fusion}
\end{tactic}

% --------------------------------------------------------------------
\begin{tactic}{generalize}
\end{tactic}

% --------------------------------------------------------------------
\begin{tactic}{idtac}
\end{tactic}

% --------------------------------------------------------------------
\begin{tactic}{inline}
\end{tactic}

% --------------------------------------------------------------------
\begin{tactic}{intros}
\end{tactic}

% --------------------------------------------------------------------
\begin{tactic}{iota}
  \begin{tsyntax}[empty]{iota}
  \fix{Missing description of iota}.
  \end{tsyntax}
\end{tactic}

% --------------------------------------------------------------------
\begin{tactic}{kill}
  \begin{tsyntax}[empty]{kill}
  \fix{Missing description of kill}.
  \end{tsyntax}
\end{tactic}

% --------------------------------------------------------------------
\begin{tactic}{left}
  \begin{tsyntax}[empty]{left}
  Reduce a disjunctive goal to its left member.
  \end{tsyntax}
\end{tactic}

% --------------------------------------------------------------------
\begin{tactic}{logic}
\end{tactic}

% --------------------------------------------------------------------
\begin{tactic}{modpath}
  \begin{tsyntax}[empty]{modpath}
  \fix{Missing description of modpath}.
  \end{tsyntax}
\end{tactic}

% --------------------------------------------------------------------
\begin{tactic}[move | move: $\;\pi_1 \cdots \pi_n$]{move}
  \begin{tsyntax}{move}
     Does nothing, equivalent to \rtactic{idtac}. This form is mainly
     used in conjonction with an introduction pattern (see
     Section~\ref{s:intro-pattern}), e.g. \ls!move=> $\iota_1 \cdots \iota_n$!.
  \end{tsyntax}

  \begin{tsyntax}{move: $\;\pi_1 \cdots \pi_n$}
    Generalize the patterns $\pi_1, \cdots, \pi_n$, starting from
    $\pi_n$ and going back.
    %See Section~\ref{s:gen-pattern} for more
    %information on the generalization mechanism.
  \end{tsyntax}
\end{tactic}

% --------------------------------------------------------------------
\begin{tactic}{pose}
  \begin{tsyntax}[empty]{pose}
  \fix{Missing description of pose}.
  \end{tsyntax}
\end{tactic}

% --------------------------------------------------------------------
\begin{tactic}{pr\_bounded}
  \begin{tsyntax}[empty]{pr\_bounded}
  \fix{Missing description of pr\_bounded}.
  \end{tsyntax}
\end{tactic}

% --------------------------------------------------------------------
\begin{tactic}{progress}
\end{tactic}

% --------------------------------------------------------------------
\begin{tactic}{rcondf}
  \begin{tsyntax}[empty]{rcondf}
  \fix{Missing description of rcondf}.
  \end{tsyntax}
\end{tactic}

% --------------------------------------------------------------------
\begin{tactic}{rcondt}
  \begin{tsyntax}[empty]{rcondt}
  \fix{Missing description of rcondt}.
  \end{tsyntax}
\end{tactic}

% --------------------------------------------------------------------
\begin{tactic}{reflexivity}
\end{tactic}

% --------------------------------------------------------------------
\begin{tactic}{rewrite}
\end{tactic}

% --------------------------------------------------------------------
\begin{tactic}{right}
\end{tactic}

% --------------------------------------------------------------------
\begin{tactic}{ringeq}
  \begin{tsyntax}[empty]{ringeq}
  \fix{Missing description of ringeq}.
  \end{tsyntax}
\end{tactic}

% --------------------------------------------------------------------
\begin{tactic}{rnd}
\end{tactic}

% --------------------------------------------------------------------
\begin{tactic}{rwnormal}
  \begin{tsyntax}[empty]{rwnormal}
  \fix{Missing description of rwnormal}.
  \end{tsyntax}
\end{tactic}

% --------------------------------------------------------------------
\begin{tactic}{seq}
  Rules for sequences:

  \begin{tsyntax}{seq p1 p2 : R}
  \begin{mathpar}
  \inferrule*[left=(\prhl),rightskip=10em]%%
    {|c_1| = \tct{p1}\\%%
     |c_2| = \tct{p2}\\%%
     \pRHL{P}{c_1}{c_2}{R}\\%%
     \pRHL{R}{c_1'}{c_2'}{Q}}%%
    {\pRHL{P}{c_1;c_1'}{c_2;c_2'}{Q}}%%
    \quad\raisebox{.7em}{\tct{seq p1 p2 : R}}\\%%
  \end{mathpar}
  \end{tsyntax}

  \begin{tsyntax}{seq p : R}
  \begin{mathpar}
  \inferrule*[left=(\hl),rightskip=10em]%%
    { |c| = \tct{p}\\%%
     \HL{P}{c}{R}\\%%
     \HL{R}{c'}{Q} }%%
    {\HL{P}{c;c'}{Q}}%%
    \quad\raisebox{.7em}{\tct{seq p : R}}\\%%
  \end{mathpar}
  \end{tsyntax}


  \fix{Missing description of seq for phl}.

\end{tactic}

% --------------------------------------------------------------------
\begin{tactic}{simplify}
  \begin{tsyntax}[empty]{simplify}
  \fix{Missing description of simplify}.
  \end{tsyntax}
\end{tactic}

% --------------------------------------------------------------------
\begin{tactic}{sim}
\end{tactic}

% --------------------------------------------------------------------
\begin{tactic}{skip}
  \begin{tsyntax}[empty]{skip}
  \fix{Missing description of skip}.
  \end{tsyntax}
\end{tactic}

% --------------------------------------------------------------------
\begin{tactic}[smt $\textit{ smt-options}$]{smt}
  \begin{tsyntax}[empty]{smt}
  Try to solve the goal using SMT solvers. The goal is sent along with 
  the local hypotheses plus a selected number of axioms/lemmas.
  \end{tsyntax}
  Generic options are:
  \begin{itemize}
    \item \ec{timeout=}$n$: set the timeout for provers to $n$ (in seconds).
    \item \ec{maxprovers=}$n$: set the maximun number of prover runing in 
          parallele to $n$ 
    \item \ec{prover=[}\textit{prover-selector}\ec{]} : select the provers. \\
          Variant [\textit{prover-selector}]. \\
          \textit{prover-selector} can be:
          \begin{itemize}
            \item \ec{``}\textit{prover-name}\ec{''}: use this particular prover
            \item \ec{+``}\textit{prover-name}\ec{''}: add \textit{prover-name} 
                    to the current list of provers
            \item \ec{-``}\textit{prover-name}\ec{''}: 
                    remove \textit{prover-name} 
                    from the current list of provers 
          \end{itemize}
          Examples:
          \begin{itemize}
          \item \ec{[``Z3'' ``Alt-Ergo'']}: use only Z3 and Alt-Ergo 
          \item \ec{[``Z3'' ``Alt-Ergo'' -''Z3'']}: use only Alt-Ergo 
          \item \ec{[-''CVC4'']}: remove CVC4 form the current list of prover,
                so assumming the current list is Z3 and CVC4 this is equivalent
                to \ec{[``Z3'']}
          \item \ec{[+''CVC4'']}: add CVC4 to the current list of prover,
                so assumming the current list is Z3 and Alt-Ergo this is
                equivalent
                to \ec{[``Z3'' ``Alt-Ergo'' ``CVC4'' ]}
          \end{itemize}          
  \end{itemize}
  Axioms and lemmas are not all send to smt provers, 
  \EasyCrypt use a strategy to automatically select them.
  Lemmas and axioms marked with ``nosmt'' are not selected.
  This strategy can be parametrized using different options:
  \begin{itemize}    
    \item \ec{unwantedlemmas=}\textit{dbhint}: 
          do not send axiom/lemma selected by \textit{dbhint}
    \item \ec{wantedlemmas=}\textit{dbhint}: 
          send axiom/lemma selected by \textit{dbhint} 
    \item \ec{all}: 
          select all available axioms/lemmas execpted those specified by 
          \ec{unwantedlemmas} (if any).
    \item \ec{maxlemmas=}$n$: 
          set the maximun number of selected axioms/lemmas to $n$.
          Keep this number small is generally more effienciant.
          Variant: $n$
    \item \ec{iterate}: try to incrementally augment the number of selected
          axioms/lemmas. Last call will be equivalent to all.
  \end{itemize}

  \fixme{Describe \textit{dbhint} options.}

  Options can also be specified by short name, for example:
  \begin{center} \ec{smt 100 [+''Z3] tmo=4 mp=2}\end{center}
  is equivalent to 
  \begin{center}
  \ec{smt maxlemmas=100 prover=[+''Z3] timeout=4 maxprovers=2}
  \end{center}

  Smt option can be set globally using the following syntax:\\
  \ec{prover} \texit{smt-options}


\end{tactic}

% --------------------------------------------------------------------
\begin{tactic}{split}
  \begin{tsyntax}[empty]{split}
  Break an intrinsically conjunctive goal into its component subgoals.
  For instance, it can:
  \begin{itemize}
    \item close any goal that is convertible to \tct{true} or provable by \tct{reflexivity},
    \item replace a logical equivalence by the direct and indirect implication,
    \item replace a goal of the form \tct{f1 /\\ f2} by the two subgoals for \tct{f1} an
          \tct{f2}. The same applies for a goal of the form \tct{f1 && f2},
    \item replace an equality between $n$-tuples by $n$ equalities
          on their components.
  \end{itemize}
  \end{tsyntax}
\end{tactic}

% --------------------------------------------------------------------
\begin{tactic}{splitwhile}
  \begin{tsyntax}[empty]{splitwhile}
  \fix{Missing description of splitwhile}.
  \end{tsyntax}
\end{tactic}

% --------------------------------------------------------------------
\begin{tactic}{sp}
  \begin{tsyntax}[empty]{sp}
  \fix{Missing description of sp}.
  \end{tsyntax}
\end{tactic}

% --------------------------------------------------------------------
\begin{tactic}[subst | subst x]{subst}
  \begin{tsyntax}[empty]{subst}
  Search for the first equation of the form \ec{x = f} or \ec{f = x} in the context
  and replace all the occurrences of \ec{x} by \ec{f} everywhere in the context and the
  goal before clearing it. If no identifier is given, repeatedly apply the tactic to
  all identifiers for which such an equation exists.
  \end{tsyntax}
\end{tactic}

% --------------------------------------------------------------------
\begin{tactic}{swap}
\end{tactic}

% --------------------------------------------------------------------
\begin{tactic}{symmetry}
  \begin{tsyntax}{symmetry}
  In \prhl, swaps the two programs, transforming the pre and
  postconditions by swapping the memories they refer to.

  \textbf{Examples:} In the following, $\invrel{\cdot}$ inverses its
  argument relation. (That is, for any relation $R$ and any $m_1$,
  $m_2$, we have
  $m_1 \mathrel{R} m_2\Leftrightarrow m_2 \mathrel{\invrel{R}} m_1$.)
  \begin{mathpar}
  \inferrule%%
    {\pRHL{\invrel{P}}{c_2}{c_1}{\invrel{Q}}}%%
    {\pRHL{P}{c_1}{c_2}{Q}}%%
    \quad\mbox{(\prhl)\quad\parbox{50pt}{\tct{symmetry}}}
  \end{mathpar}
  \end{tsyntax}
\end{tactic}

% --------------------------------------------------------------------
\begin{tactic}{transitivity}
\end{tactic}

% --------------------------------------------------------------------
\begin{tactic}{trivial}
  \begin{tsyntax}[empty]{trivial}
  \fix{Missing description of trivial}.
  \end{tsyntax}
\end{tactic}

% --------------------------------------------------------------------
\begin{tactic}{unroll}
  \begin{tsyntax}[empty]{unroll}
  \fix{Missing description of unroll}.
  \end{tsyntax}
\end{tactic}

% --------------------------------------------------------------------
\begin{tactic}{wp}
  \begin{tsyntax}{wp}
  Computes the weakest precondition of a straightline deterministic
  suffix of the program(s) that implies the current
  postcondition. \tct{wp} also consumes deterministic \tct{if}
  statements (when both branches are deterministic straightline code
  without procedure calls).
  \end{tsyntax}

  \begin{tsyntax}{wp $\ n_1$ $\ n_2$}
  In \prhl, let \tct{wp} consume \emph{exactly} $n_1$ statements of
  the left program and $n_2$ statements of the right program.
  \end{tsyntax}

  \begin{tsyntax}{wp $\ n$}
  In \phl and \hl, let \tct{wp} consume \emph{exactly} $n$ statements
  of the program.
  \end{tsyntax}
\end{tactic}

% --------------------------------------------------------------------
\begin{tactic}{zeta}
  \begin{tsyntax}[empty]{zeta}
  \fix{Missing description of zeta}.
  \end{tsyntax}
\end{tactic}



\section{Miscellaneous tool directives}
\begin{itemize}
\item {\verb+include+~\textit{filename}}: Loads and processes the
  contents of the \EasyCrypt file \textit{filename}.

\item {\verb+timeout+ \textit{secs}:} Sets the current timeout given
  to SMT solvers to the value \textit{secs}. Used to increase the
  default timeout value when no SMT solver manage to prove the
  required logical goals.

\item %
  {\verb+prover+~\textit{prover${}_1$}\verb+,+..\verb+,+\textit{prover${}_k$}:}
  Sets the list of provers (separated by '\verb+,+') that are
  available to discharge the logical verification conditions. By
  default, \EasyCrypt tries with all provers recognized when invoking
  \verb|why3config --detect|. A prover name can be given either as an
  identifier or a string.

\item {\verb+check+~\textit{name}/ \verb+print+~\textit{name}} Show
  information about the object associated to the name \textit{name}. 
  

\item {\verb+checkproof+:} Enables and disables the verification of
  logical verification conditions. 

\item {\verb+set+~\textit{name}/\verb+unset+~\textit{name}}: Make the
  axiom or lemma with name \textit{name} available/unavailable as
  hypothesis for the verification of logical formulae.

\item {\verb+transparent+~\textit{name}/ \verb+opaque+~\textit{name}}:
  Set the definition of the predicate with name \textit{name} as
  transparent or opaque. If a predicate is opaque then its definition
  is not unfolded during the verification of logical formulae.


\end{itemize}

\chapter{Probability Claims and Computation}
Security properties are expressed in terms of probability of events,
rather than as pRHL judgments. Pleasingly, one can derive inequalities
(resp. equality) about probability quantities from valid judgments. In
particular, assume that the postcondition $\post$ implies $A\sidel
\Rightarrow B\sider$. Then for any programs $c_1$, $c_2$ and
precondition $\pre$ such that $\Equiv{c_1}{c_2}{\pre}{\post}$ is valid
and for any initial memories $m_1$, $m_2$ satisfying the precondition
$\pre$, we have
$$\Prm{c_1}{A}{m_1} \leq \Prm{c_2}{B}{m_2}$$
Up to now, \easycrypt assume that the two games start in the same initial
memory (i.e. $m_1 = m_2$), thus the equality of initial memories should
imply the validity of the precondition.

\section{Claims using equiv}

The natural way to obtain new claims is to deduce it from a pRHL judgment.
Assume we have proved a pRHL judgment of the form: 
\begin{verbatim}
equiv Fact1 : Game1.Main ~ Game2.Main : true ==> ={res}.
\end{verbatim} 
Then we can deduce:
\begin{verbatim}
claim c1 : Game1.Main[res] = Game2.Main[res] using Fact1.
\end{verbatim}
\easycrypt will check that the equality of the initial memories implies
the validity of the precondition (here \verb+true+) and that the
postcondition implies the logical equivalence of the two events 
(here \verb+ ={res} => (res{1} <=> res{2})+).

pRHL judgments also allow proving inequality relations between probability
expressions.  Assume we have proved a pRHL judgment of the form:
\begin{verbatim}
equiv Fact2 : Game1.Main ~ Game2.Main : 
    true ==> ={res} && (bad{1} => bad{2}).
\end{verbatim} 
Then we can deduce:
\begin{verbatim}
claim c2 : Game1.Main[res] = Game2.Main[res] using Fact2.
\end{verbatim}
but also:
\begin{verbatim}
claim c3 : Game1.Main[res && bad] <= Game2.Main[bad] using Fact2.
\end{verbatim}
For the last claim, \EasyCrypt checks that the postcondition of 
the pRHL judgment (\verb+={res} && (bad{1} => bad{2})+)
and the event associated to the first game (\verb+res{1} && bad{1}+)
imply the event associated to the second game (\verb+bad{2}+).

There is a third kind of claim which can be deduced from a pRHL judgment.
This kind of judgment is closely related to the fundamental lemma 
(also named difference lemma).
\paragraph{Fundamental lemma}{\it Let $F_1$ and $F_2$ be to distribution,
 and $A_1, A_2, B_1, B_2$ some events. Assume that
 \begin{itemize}
    \item $\Pr{F_1}{B_1} = \Pr{F_2}{B_2} $
    \item $\Pr{F_1}{A_1 \land \neg B_1} = \Pr{F_2}{A_2 \land \neg B_2}$
 \end{itemize}
then we have 
  $$ | \Pr{F_1}{B_1} - \Pr{F_2}{B_2} | \leq \Pr{F_i}{B_i}$$}

Now assume we have proved a specification of the form:
\begin{verbatim}
equiv Fact3 : Game1.Main ~ Game2.Main : 
    true ==> B1{1} <=> B2{2} && (!B1{1} => A1{1} <=> A2{2}).
\end{verbatim}
Then we can derive the following claims:
\begin{verbatim}
claim c4_1 : Game1.Main[B1] = Game2.Main[B2]
using Fact3.
claim c4_2 : Game1.Main[!B1 && A1] = Game2.Main[!B2 && A2]
using Fact3.
\end{verbatim}
So the two hypotheses of the fundamental lemma are satisfied. 
\EasyCrypt allows deriving directly the conclusion of the fundamental
lemma from \verb+Fact3+:
\begin{verbatim}
claim c4 : |Game1.Main[A1] - Game2.Main[A2] | <= Game2.Main[B2] 
using Fact3.
\end{verbatim} 
For this kind of claim, \EasyCrypt checks that the postcondition of
the pRHL judgment implies the equivalence of the bad events 
(here \verb+B1+ and \verb+B2+) in the two games. 
Furthermore if the postcondition is valid and the bad event (here \verb+B2+) 
is not set then the two events 
(here \verb+A1{1}+ and \verb+A2{2}+) should be equivalent.



\section{Claim using same and split}

There is some particular case of claim which can be deduced 
automatically without using pRHL judgments.
More precisely, the judgment $\Equiv{c}{c}{=}{=}$ is always valid 
(where $=$ means the equality of the memories).
Thus, we can derive some simple properties from it.
\begin{verbatim}
claim c_1 : G1.Main[res && (b || !b)] = G1.Main[res] 
same.
claim c_2 : G1.Main[res && b ] <= G1.Main[res]
same.
\end{verbatim}
Claim defined using \verb+same+ argument should relates the probability
of two events $A_1$ and $A_2$ in the same game.
If the comparison operator is the equality then we should have 
$A_1 \Leftrightarrow A_2$ (as in the claim \verb+c_1+).
If the comparison operator is the less or equal operator 
then we should have $A_1 \Rightarrow A_2$ (as in the claim \verb+c_2+).

Another way to simply derive claim is to use the \verb+split+ argument.
\begin{verbatim}
claim c_3 : G1.Main[res] = G1.Main[res && bad] + G1.Main[res && !bad]
split.
\end{verbatim}
If the comparison operator is the equality the claim should match the
generic form \verb?G.F[A] <= G.F[A&&B] + G.F[A&&!B]?.
If the comparison operator is the less or equal operator then
the claim should have the generic form \verb?G.F[A] <= G.F[B] + G.F[C]?.
Furthermore \easycrypt check that $A \Rightarrow (B \lor C)$.

An exemple of use of the \verb+split+ and \verb+same+ is the proof of the
fundamental lemma, assume we have proved the specification:
\begin{verbatim}
equiv Fact3 : Game1.Main ~ Game2.Main : 
    true ==> B1{1} <=> B2{2} && (!B1{1} => A1{1} <=> A2{2}).
\end{verbatim}
Then we can derive the following claims:
\begin{verbatim}
claim c4_1 : Game1.Main[B1] = Game2.Main[B2]
using Fact3.
claim c4_2 : Game1.Main[!B1 && A1] = Game2.Main[!B2 && A2]
using Fact3.
\end{verbatim}
but also:
\begin{verbatim}
claim c4_split1 : Game1.Main[A1] = Game1.Main[B1 && A1] + Game1.Main[!B1 && A1]
split.
claim c4_split2 : Game2.Main[A2] = Game2.Main[B2 && A2] + Game2.Main[!B2 && A2]
split.
claim c4_same1 : Game1.Main[B1 && A1] <= Game1.Main[A1]
same.
claim c4_same2 : Game2.Main[B2 && A2] <= Game1.Main[A2]
same. 
\end{verbatim}
Using the claims \verb+c4_1+, \verb+c4_2+, \verb+c4_split1+, \verb+c4_split2+,
\verb+c4_same1+, \verb+c4_same2+ the automatic provers (like \verb+alt-ergo+)
are able to derive the following claim:
\begin{verbatim}
claim c4 : |Game1.Main[A1] - Game2.Main[A2] | <= Game2.Main[B2].
\end{verbatim}





\section{Deducing claim from other claims}
Claim can be derived as a consequence of other claims.
When no argument is given after the statement of the claim \easycrypt
try to prove it using the previously proved claims.

Assume we have already proved the following claims:
\begin{verbatim}
claim c_1 : G1.Main[res] = G2.Main[res].
claim c_2 : | G2.Main[res] - G3.Main[res] | <= G3.Main[bad].
claim c_3 : G3.Main[res] = 1%r/2%r.
claim c_4 : G3.Main[bad] <= 1%r/(2^n)%r.
\end{verbatim}
Then the following claim is automatically deduced from the previous one:
\begin{verbatim}
claim c_5 : | G1.Main[res] - 1%r/2%r | <= 1%r/(2^n)%r.
\end{verbatim}

\section{Claims by compute}

During a reduction proof, we sometime need to compute or to bound
the probability of an event in a given game. This can be done using
the \verb+compute+ argument. Assume we have the following game:
\begin{verbatim}
game G = { 
   ...
   fun Main() : bool = {
     (pk,sk) = KG();
     (m0,m1) = A_1(pk);
     c       = {0,1}^k;
     b'      = A_2(c);
     b       = {0,1};
     return b = b';
  }
} 
\end{verbatim}
Then \easycrypt is able to compute the probability of \verb+res=true+
in the function \verb+G.Main+:
\begin{verbatim}
claim c : G.Main[res] = 1%r/2%r 
compute.
\end{verbatim} 

The \verb+compute+ argument is also able to prove the claim that can be
derive using \verb+split+ and \verb+same+, but it is less efficient.
On the other side it is also more powerful, for example we can prove:
\begin{verbatim}
claim c : G.Main[A || B || C] <= G.Main[A] + G.Main[B] + G.Main[C]
compute.  
\end{verbatim}
This claim can also be obtained using the \verb+split+ argument, using
the following sequence:
\begin{verbatim}
claim c_1 :  G.Main[A || B || C] <= G.Main[A || B] + G.Main[C]
split.
claim c_2 :  G.Main[A || B] <= G.Main[A] + G.Main[B]
split.
claim c : G.Main[A || B || C] <= G.Main[A] + G.Main[B] + G.Main[C]. 
\end{verbatim}
The claim \verb+c+ is a direct consequence of the claims \verb+c_1+ and 
\verb+c_2+.

A last example of use for \verb+compute+ is the following, assume
we have a game of the form:
\begin{verbatim}
game G = { 
   ...
   fun Main () : bool = {
     x = init();
     d = A(x);
     z = {0,1}^k;
     return d;
   }
}     
\end{verbatim}
Then \verb+compute+ is able to prove the following claim:
\begin{verbatim}
claim c :
  G.Main[res && mem(z,L) && length(L) <= q] <= q%r/(2^k)%r * G.Main[res]
compute.
\end{verbatim}

Sometime the event we want to bound is not set in the main function
but in an oracle, furthermore we known that the oracle can be call at
most $q$ time. Assume that the probability that the event is set during
one call to the oracle is bounded by $u$, we would like to conclude 
that the probability event is set in the main function is bounded by
$u*q$. This is possible using the failure event lemma.
\begin{verbatim}
game G = {
  var C: int
  var bad : bool
  fun O(x:int) : bitstring{k} = {
    var r = {0,1}^k;
    C = C + 1;
    if (r = 0) bad = true;
    return r;
  }
  abs A = A {O}
  fun Main() : bool = {
    var d : bool;
    C = 0;
    bad = false;
    d = A();
    return d;
  }    
\end{verbatim}
In the example the probability that \verb+bad+ is set in during a
call to the oracle \verb+O+ is $1/2^k$, furthermore the counter \verb+C+
count the number of call to \verb+O+.
We can use the following to bound the probability of \verb+bad+ in the main:
\begin{verbatim} 
claim pr_bad : G.Main[bad && C <= q] <= q%r * (1%r/(2^k)%r)
compute 2 (bad), (C).
\end{verbatim}

The second argument indicate the bad event (of the oracle) 
we consider and the third should be a expression representing the counter. 
The first argument is an integer indicating the number of instructions
in the main needed to initialize the failure event lemma. 
After those instructions the value associated to the counter should be 0
and the bad event should evaluate to false.
Then \easycrypt should be able to prove that the probability that the bad
event is set during an oracle call is bounded by $1/2^k$. Furthermore,
if the bad event is set during a call to the oracle then the counter
increase, and do not decrease in the other case, and that the bad event
is never reset.

\section{Claims by admit}
The last possibility to define a claim is to use the admit argument.
\begin{verbatim}
claim c : G.Main[res] = G'.Main[res] 
admit.
\end{verbatim}
It that case the validity of the claim is admitted without any check.

\section{Claims by auto}

It is also possible to directly define claim which normally should be 
defined using an \textit{equiv} specification directly:
\begin{verbatim}
claim c12 : G1.Main[res] = G2.Main[res]
auto.
\end{verbatim}
This is a shortcut for: 
\begin{verbatim}
equiv c12_aux : G1.Main ~ G2.Main : true ==> ={res}
by auto.
claim c12 : G1.Main[res] = G2.Main[res]
using c12_aux.
\end{verbatim}

%%% Local Variables: 
%%% mode: latex
%%% TeX-master: "easycrypt"
%%% End: 


\chapter{Example: elgamal}
\begin{flushright}
\it (The syntax used in this section may be outdated.)
\end{flushright}

We illustrate the key ingredients presented in prevous chapters with a
simple example: a game-based proof of the \INDCPA-security of the
ElGamal public-key encryption scheme.

The ElGamal encryption scheme is based on any cyclic group $G$ of
order $q$ with generator $g$ and is defined by the following triple of
algorithms

\begin{itemize} 
\item The key generation algorithm $\KG()$ selects uniformly a random
      number $x$ from $\{0,\ldots,q-1\}$; the secret (private) key is
      $x$, the public key is $g^x$.

\item Given a public key $pk$ and a plaintext $m$ (an element of the
      group $G$), the encryption algorithm $\Enc(pk, m)$ chooses
      uniformly a random element $y$ from $\{0,\ldots,q-1\}$ and
      returns the ciphertext $(g^y, pk^y * m)$.

\item Given a secret key $sk$ and a ciphertext $c$, 
      the decryption algorithm $\Dec(sk, c)$, parses $c$ as
      $(\beta,\zeta)$ and returns a plaintext computed as
      $\zeta * \beta^{-x}$.
\end{itemize}

We start by declaring a type for elements of the group $G$, and
defining type synonyms for the type of public and secret keys,
plaintexts and ciphertexts:
%
\begin{verbatim} 
type group 
type skey = int 
type pkey = group 
type plaintext = group 
type ciphertext = group * group 
\end{verbatim} 
%
The order of the group $q$ and its generator $g$ are declared as
constants:
%

\begin{verbatim}
cnst q : int
cnst g : group
\end{verbatim}

%
We then declare operators that will denote the group law in $G$,
exponentiation and discrete logarithm (in base $g$).
%
\begin{verbatim}
op (*) : group, group -> group = group_mult
op (^) : group, int -> group   = group_pow
op log : group-> int           = group_log
\end{verbatim}
%
% The first two operators are declared as infix, and denoted by the
% symbols \verb|*| and \verb|^|, respectively. The operator
% corresponding to the discrete logarithm is a normal prefix
% operator. The names appearing on the right of the declaration are
% identifiers that will be used as internal names for the operators when
% generating proof obligations that are sent to SMT solvers (this is
% needed because fancy identifiers like \verb|*| are not valid
% identifiers, and useful to avoid name clashes with predefined
% operators).

At this point the operators and constants that we declared above are
completely abstract, nothing is known about them besides their
type. To specify

At that point nothing say that the type \verb+group+ is a cyclic
group, we only known that the type come with three
operators \verb+*+, \verb+^+ and \verb+log+. We should specify the
behavior of the operators this is done using axioms:

\begin{verbatim}
axiom q_pos : {0 < q}

axiom group_pow_add : 
 forall (x:int, y:int). { g ^ (x + y) == g ^ x * g ^ y }

axiom group_pow_mult :
 forall (x:int, y:int). { (g ^ x) ^ y == g ^ (x * y) } 

axiom log_pow : 
 forall (g':group). { g ^ log(g') == g' }

axiom pow_mod : 
 forall (z:int). { g ^ (z%q) == g ^ z }
\end{verbatim}
      The first axiom \verb+q_pos+ expresses that the integer \verb+q+ 
      representing the order of the group is positive. 
      The next \verb+group_pow_add+ and \verb+group_pow_mult+ specify the 
      behavior of the multiplication and the exponentiation,  
      \verb+log_pow+ partially specify the behavior of the logarithm operator.
      The \verb|+| operator used in \verb+group_pow_add+ is the predefined 
      additive operator over integer. Note that the \verb+*+ operator in 
      the axiom \verb+group_pow_mult+ represent the multiplication over 
      integer and not the multiplication law of the group 
      (\easycrypt{} allows to overloading of operator). 
      The last axiom expresses the fact that the group is a cyclic group of 
      order \verb+q+, \verb+%+ stand for the modulus operator over integer.

      To be able to perform the proof we also add axioms on the modulus operator:
\begin{verbatim}
axiom mod_add : 
 forall (x:int, y:int). { (x%q + y)%q == (x + y)%q }

axiom mod_small : 
 forall (x:int). { 0 <= x } => { x < q } => { x%q == x}

axiom mod_sub : 
 forall (x:int, y:int). { (x%q - y)%q == (x - y)%q } 
\end{verbatim}

The IND-CPA semantic security is expressed as a game parameterized by an pair 
of adversaries, let us declare this two adversaries:
\begin{verbatim}
adversary A1(pk:pkey)               : plaintext * plaintext {}
adversary A2(pk:pkey, c:ciphertext) : bool {}
\end{verbatim}
The first one \verb+A1+ expect a public key \verb+pk+ and return a pair 
of plaintext, the second one expect a public key and a cyphertext and return
a boolean. The semi-bracket contains the declaration of the oracles that can
be used by the adversaries, here there is no oracles.

We can now define the game representing the IND-CPA semantic security of ElGamal:
\begin{verbatim}
game INDCPA = {
  fun KG() : keys = {
    var x : int = [0..q-1];
    return (x, g^x);
  }

  fun Enc(pk:pkey, m:plaintext): ciphertext = {
    var y : int = [0..q-1];
    return (g^y, (pk^y) * m);
  }

  abs A1 = A1 {}
  abs A2 = A2 {}
  
  fun Main() : bool = {
    var sk : skey;
    var pk : pkey;
    var m0, m1, mb : plaintext;
    var c: ciphertext;
    var b, b' : bool;

    (sk,pk) = KG();
    (m0,m1) = A1(pk);
    b = {0,1};
    mb = b ? m0 : m1;
    c = Enc(pk, mb);
    b' = A2(pk, c);
    return (b == b');
  } 
}      
\end{verbatim}
The game start by the declaration of two functions the key generation
algorithm \verb+KG+ and the encryption algorithm \verb+Enc+. Then come
the definition of the two adversary \verb+A1+ and \verb+A2+, they are
defined to be equal to the abstract functions previously defined.
The main function, at the end of the game, represent the IND-CPA experiment.
First the key generation algorithm is used to generate the secret and public
keys, then the public key is given to \verb+A1+ which generate two plaintext
\verb+m0+ and \verb+m1+. The instruction \verb+b = {0,1}+ uniformly sample a
boolean which is stored in \verb+b+. Depending on this bit \verb+b+
either the plaintext \verb+m0+ or \verb+m1+ is encrypted with the
public key \verb+pk+, generating the ciphertext \verb+c+. The public key and 
ciphertext are then give back to the adversary \verb+A2+. The goal of the
adversary is to discover which plaintext as been encrypted. It win if
\verb+b+ is equal to \verb+b'+.

The IND-CPA semantic security of ElGamal express that there exists a 
adversary \verb+B+ build on top of \verb+A1+ and \verb+A2+ which as a higher 
probability of breaking the Decisional Diffie Hellman problem (DDH) than
\verb+A1+ and \verb+A2+ of winning the IND-CPA game. 
% The DDH hypothesis 
% say that it is hard to distinguish ....\todo{finish this}.
The first thing to do is to define the two games and the 
adversary \verb+B+ involved in DDH problem:
\begin{verbatim}

game DDH0 = {
  abs A1 = A1 {}
  abs A2 = A2 {}
  
  fun B(gx:group, gy:group, gz:group) : bool = {
    var m0, m1, mb : plaintext;
    var c : ciphertext;
    var b, b' : bool;
 
    (m0, m1) = A1(gx);
    b = {0,1};
    mb = b ? m0 : m1;
    c = (gy, gz * mb);
    b' = A2(gx,c);
    return (b == b');
  }

  fun Main() : bool = {
    var x, y : int;
    var d : bool;

    x = [0..q-1];
    y = [0..q-1];
    d = B(g^x, g^y, g^(x*y));
    return d;
  }     
}

game DDH1 = DDH0 where 
  Main = {
    var x, y, z : int;
    var d : bool;

    x = [0..q-1];
    y = [0..q-1];
    z = [0..q-1];
    d = B(g^x, g^y, g^z);
    return d;
  } 

\end{verbatim}
The main experiment in the game \verb+DDH0+ start by uniformly sample
two values \verb+x+ and \verb+y+ between 0 and $q-1$ and then 
send $g^x, g^y, g^{xy}$ to the adversary \verb+B+. The game \verb+DDH1+
is defined to be equal to the game \verb+DDH0+ where only the main function
changes: a new variable \verb+z+ is uniformly sample and $g^z$ is send
to the adversary instead of $g^{xy}$. The goal of the adversary is to discover
if its last argument correspond to $g^{xy}$ or $g^z$, i.e. if it play
between \verb+DDH0+ or \verb+DDH1+.

We can know start our proof:
\begin{verbatim}
prover alt-ergo

equiv auto Fact1 : INDCPA.Main ~ DDH0.Main : {true} ==> ={res};;

claim Pr1 : INDCPA.Main[res] == DDH0.Main[res] 
using Fact1;;
\end{verbatim}
The first line select the prover to be used, here \verb+alt-ergo+ (the
default one is \verb+simplify+). The second line is the main component of
\easycrypt. We demonstrate using the probabilistic Relational Hoare Logic (pRHL)
that the two functions \verb+INDCPA.Main+ and \verb+DDH0.Main+ are 
indistinguishable if we observe only their results.
This allows to proving the claim \verb+Pr1+ which state that the probability
that \verb+res+ is true after running the two programs is equal.
% \todo{rewrite this ...}

\begin{verbatim}
game G1 = INDCPA where 
  Main = {
    var x, y, z : int;
    var gx, gy, gz : group;
    var d, b, b' : bool;
    var m0, m1, mb : plaintext;
    var c : ciphertext;
 
    x = [0..(q - 1)];
    y = [0..(q - 1)];
    gx = g^x;
    gy = g^y;
    (m0, m1) = A1 (gx);
    b = {0,1};
    mb = b ? m0 : m1; 
    z = [0..(q - 1)];
    gz = g^z;
    c = (gy, gz * mb);
    b' = A2 (gx, c);
    d = (b == b');
    return d;
  }

equiv auto Fact2 : G1.Main ~ DDH1.Main : {true} ==> ={res};;
 
claim Pr2 : G1.Main[res] == DDH1.Main[res] 
using Fact2;;
\end{verbatim}

\begin{verbatim}
game G2 = G1 where 
  Main = {
    var x, y, z : int;
    var gx, gy, gz : group;
    var d, b, b' : bool;
    var m0, m1, mb : plaintext;
    var c : ciphertext;
 
    x = [0..(q - 1)];
    y = [0..(q - 1)];
    gx = g^x;
    gy = g^y;
    (m0, m1) = A1(gx);
    z = [0..(q - 1)];
    gz = g^z;
    c = (gy, gz); 
    b' = A2 (gx, c);
    b = {0,1};
    d = (b == b');
    return d;
  }

equiv Fact3 : G1.Main ~ G2.Main : {true} ==> ={res} 
 swap{2} [10-10] -4; auto;
 rnd (z + log(b?m0:m1)) % q, (z - log(b?m0:m1)) % q; wp; rnd; 
 auto; repeat rnd;
 trivial;;
save;;

claim Pr3 : G1.Main[res] == G2.Main[res]
using Fact3;;
\end{verbatim}

\begin{verbatim}
claim Pr4 : G2.Main[res] == 1%r / 2%r
compute;;

claim Conclusion : 
 | INDCPA.Main[res] - 1%r / 2%r | <= | DDH0.Main[res] - DDH1.Main[res] | 
\end{verbatim}

%%% Local Variables: 
%%% mode: latex
%%% TeX-master: "easycrypt"
%%% End: 


%%% Local Variables: 
%%% mode: latex
%%% TeX-master: "easycrypt"
%%% End: 


\part{Language Reference}
  
\section{Lexical conventions}


%\section{Syntax}
\newenvironment{ecgrammar}{\bgroup\framed\grammar}{\endgrammar\endframed\egroup}

%\setlength{\grammarparsep}{20pt plus 1pt minus 1pt} % increase separation between rules
\setlength{\grammarindent}{8em} % increase separation between LHS/RHS 

% \small


\subsubsection*{Comments.}
Comments are enclosed by $(*$ and $*)$.


\subsubsection*{Strings.}


\subsubsection*{Identifiers.}

\begin{ecgrammar}
<letter> := `a' - `z' | `A' - `Z' | `_'

<digit> ::= `0' - `9'

<other_letter> ::= <letter> | <digit> | `\''

<ident> ::= <letter> <other_letter>$^*$

<ident_list> ::=  <ident> | <ident> `,'  <ident_list>

<ident_list0> ::= <empty> | <ident_list>

<prim_ident> ::= `\'' <ident>

<prim_ident_list> ::= <prim_ident> | <prim_ident_list> `,' <prim_ident_list>

<number_list> ::= <number> | <number> `,' <number_list>

<qualif_fct_name>  ::= <ident>`.'<ident>

<number> ::= <digit>$^+$

<znumber> ::= <number> | `-'<number>
\end{ecgrammar}

\subsubsection*{Keywords.} The following literals are reserved and must
not be used as identifiers:
\begin{verbatim}
\end{verbatim}

%%%%%%%%%%%%%%%%%%%%%%%%%%%%%%%%%%%%%%%%%%%%%%%%%%%%%%%%%%%%%%%%%%%%%%%%%%%%%%%
%                                                                     Operators 
%%%%%%%%%%%%%%%%%%%%%%%%%%%%%%%%%%%%%%%%%%%%%%%%%%%%%%%%%%%%%%%%%%%%%%%%%%%%%%%
\subsubsection*{Operators.}
\begin{ecgrammar}

<op_char> ::= `=' | `<' | `>' | `~' | `+' | `-' | `*' | `/' | `\%'
          \alt `!' | `\$' | `&' | `?' | `@' |  `^' | `.' | `:' | `|' |  `#'

<bin_op> ::= <op_char>$^+$

<u_op> :: = `-' | `!'

<op_ident> ::= <ident> | `(' <bin_op>$^+$ `)'
\end{ecgrammar}





%%%%%%%%%%%%%%%%%%%%%%%%%%%%%%%%%%%%%%%%%%%%%%%%%%%%%%%%%%%%%%%%%%%%%%%%%%%%%%%
%                                                              Type Expressions 
%%%%%%%%%%%%%%%%%%%%%%%%%%%%%%%%%%%%%%%%%%%%%%%%%%%%%%%%%%%%%%%%%%%%%%%%%%%%%%%
\section{Type Expressions.}
\begin{ecgrammar}
<type> ::=  <ident>
       \alt ' <ident>
       \alt <type> <ident>
       \alt ( <type> (`,' <type>)$^+$ ) <ident>
       \alt ( <type> (`*' <type>)$^+$ )
       \alt `bitstring' `{' <type> `}'
       \alt `(' <type> `)'

<typed_vars> ::=  <ident_list> `:'  <type>

<typed_var_list> ::= <typed_vars> | <typed_vars> `,'  <typed_var_list>

<param_list> ::= <empty> | <typed_var_list> 

<param_decl> ::= `(' <param_list> `)'

<type_list> ::=  <type> `,' <type>
           \alt <type> `,' <type_list>

<type_list0> ::=  <type>
             \alt `(' <type_list> `)'
             \alt `()'

<fun_type> ::= <type_list0> `->' <type> 

<fun_type_list> ::= <fun_type> | <fun_type> `;' <fun_type_list>

<fun_type_list0> ::= <empty> | <fun_type_list>

\end{ecgrammar}



%%%%%%%%%%%%%%%%%%%%%%%%%%%%%%%%%%%%%%%%%%%%%%%%%%%%%%%%%%%%%%%%%%%%%%%%%%%%%%%
%                                                                         Terms
%%%%%%%%%%%%%%%%%%%%%%%%%%%%%%%%%%%%%%%%%%%%%%%%%%%%%%%%%%%%%%%%%%%%%%%%%%%%%%%
\section{Expressions.}

\subsubsection*{Simple expressions:}
\begin{ecgrammar}
<simpl_exp> ::=  <number>
            \alt <ident>
            \alt <simpl_exp> `[' <exp> `]'  
            \alt <simpl_exp> `[' <exp> `<-' <exp> `]' 
            \alt <ident> `(' <exp_list0> `)'
            \alt <simpl_exp> `{' `{'<number>`}' `}' 
            \alt <simpl_exp> `\%r' 
            \alt <qualif_fct_name> `[' <exp> `]' 
            \alt `(' <exp> `,' <exp_list> `)' 
            \alt `(' <exp> `)'
            \alt `[' <exp_list> `]'
            \alt `=' `{' <pos_ident_list> `}'
            \alt `|' <exp> `|'
            \alt <simpl_exp> `{'<number>`}'
\end{ecgrammar}

\subsubsection*{Random expressions:}
\begin{ecgrammar}
<rnd_exp> ::=  `{' <number> `,' <number> `}' 
          \alt `{' <number> `,' <number> `}^' <exp>
          \alt `[' <exp> `..' <exp> `]'
          \alt `('<rnd_exp> `\\' <exp> `)' 
\end{ecgrammar}

\subsubsection*{General expressions:}
\begin{ecgrammar}
<exp> ::=  <exp> <bin_op>  <exp>
      \alt <u_op> <exp>   
      \alt <exp> `?' <exp> `:' <exp> 
      \alt `if' <exp> `then' <exp> `else' <exp>
      \alt `forall' <param_decl> [`['<trigger_list>`]'] `,' <exp>
      \alt `exists' <param_decl> [`['<trigger_list>`]'] `,' <exp> 
      \alt `let'  <ident_list> `=' <exp> `in' <exp>
      \alt <simpl_exp>
      \alt <rnd_exp>

<trigger_list> ::= <trigger> |  <trigger> `|' <trigger_list> 

<trigger> ::= <exp> | <exp> `,' <trigger>


\end{ecgrammar}



%%%%%%%%%%%%%%%%%%%%%%%%%%%%%%%%%%%%%%%%%%%%%%%%%%%%%%%%%%%%%%%%%%%%%%%%%%%%%%%
%                                                               Global Elements 
%%%%%%%%%%%%%%%%%%%%%%%%%%%%%%%%%%%%%%%%%%%%%%%%%%%%%%%%%%%%%%%%%%%%%%%%%%%%%%%
\section{Declarations.}

%                                                                  Element.type
%%%%%%%%%%%%%%%%%%%%%%%%%%%%%%%%%%%%%%%%%%%%%%%%%%%%%%%%%%%%%%%%%%%%%%%%%%%%%%%
\subsubsection*{type}
\begin{ecgrammar}
<poly_type> ::= `(' <prim_ident_list> `)' | <prim_ident>

<type_elem> ::=  `type' [<poly_type>] <ident> 
            \alt `type' [<poly_type>] <ident> `=' <type>
\end{ecgrammar}

%                                                                  Element.cnst
%%%%%%%%%%%%%%%%%%%%%%%%%%%%%%%%%%%%%%%%%%%%%%%%%%%%%%%%%%%%%%%%%%%%%%%%%%%%%%%
\subsubsection*{cnst}
\begin{ecgrammar}
<cnst_elem> ::=  `cnst' <ident_list> `:' <type>
            \alt `cnst' <ident_list> `:' <type>  `=' <exp> 
\end{ecgrammar}

%                                                                    Element.op
%%%%%%%%%%%%%%%%%%%%%%%%%%%%%%%%%%%%%%%%%%%%%%%%%%%%%%%%%%%%%%%%%%%%%%%%%%%%%%%
\subsubsection*{op}
\begin{ecgrammar}
<op_body> ::= `:' <fun_type> 
          \alt <param_decl> `=' <exp> 

<op_elem> ::= `op' <op_ident> <op_body> 
          \alt`op' <op_ident> <op_body> `as' <ident> 
\end{ecgrammar}

%                                                                   Element.pop
%%%%%%%%%%%%%%%%%%%%%%%%%%%%%%%%%%%%%%%%%%%%%%%%%%%%%%%%%%%%%%%%%%%%%%%%%%%%%%%
\subsubsection*{pop}
\begin{ecgrammar}
<pop_elem> ::= `pop' <op_ident> `:' <fun_type>
\end{ecgrammar}

%                                                                  Element.pred
%%%%%%%%%%%%%%%%%%%%%%%%%%%%%%%%%%%%%%%%%%%%%%%%%%%%%%%%%%%%%%%%%%%%%%%%%%%%%%%
\subsubsection*{pred}
\begin{ecgrammar}
<pred_elem> ::=  `pred' <ident> <param_decl> `=' <exp>
            \alt `pred' <ident> `:' <type_ist> 
\end{ecgrammar}

%                                                                 Element.axiom
%%%%%%%%%%%%%%%%%%%%%%%%%%%%%%%%%%%%%%%%%%%%%%%%%%%%%%%%%%%%%%%%%%%%%%%%%%%%%%%
\subsubsection*{axiom}
\begin{ecgrammar}
<axiom_elem> ::=  `axiom' <ident> `:' <exp>
             \alt `lemma' <ident> `:' <exp>
\end{ecgrammar}

%                                                             Element.adversary
%%%%%%%%%%%%%%%%%%%%%%%%%%%%%%%%%%%%%%%%%%%%%%%%%%%%%%%%%%%%%%%%%%%%%%%%%%%%%%%
\subsubsection*{adversary}
\begin{ecgrammar}
<adv_elem> ::= `adversary' <fun_decl> `{' <fun_type_list0> `}' 
\end{ecgrammar}

%                                                                  Element.game
%%%%%%%%%%%%%%%%%%%%%%%%%%%%%%%%%%%%%%%%%%%%%%%%%%%%%%%%%%%%%%%%%%%%%%%%%%%%%%%
\subsubsection*{Games.}
\begin{ecgrammar}

<base_instr> ::= <ident> `('<exp_list0> `)'
 \alt <ident> `=' <exp> 
 \alt `(' <ident_list> `)' `=' <exp> 
 \alt <ident> `[' <exp> `]' `=' <exp> 

<instr> ::= <base_instr> ;
 \alt `if' `(' <exp> `)' <block> `else' <block> 
 \alt `if' `(' <exp> `)' <block>
 \alt `while' `(' <exp> `)' <block> 

<block> ::= <base_instr> `;' 
 \alt `{' <stmt> `}' 

<stmt> ::= <instr> <stmt>
 \alt <empty>


<ret_stmt> ::= `return' <exp> `;'

<loc_decl> ::= `var' <ident_list> `:' <type> [`=' <exp> ]`;'

<loc_decl_list> ::= <loc_decl>$^+$ 

<fun_def_body> ::= `{' [<loc_decl_list>] <stmt> [<ret_stmt>] `}' 

<fun_decl> ::= <ident> <param_decl> `:' <type> 

<pg_elem> ::= 
      `var' <ident_list> `:' <type> 
 \alt `fun' <fun_decl> `=' <fun_def_body>
 \alt `fun' <ident> `=' <qualif_fct_name> 
 \alt `abs' <ident> `=' <ident> `{' <ident_list0> `}' 

<game_elem> ::= `game' <ident> `=' `{' <pg_elem>$^*$ `}'
 \alt `game' <ident> `=' <ident> <var_modifier> `where' <redef_list>

\end{ecgrammar}




%                                                                 Element.equiv
%%%%%%%%%%%%%%%%%%%%%%%%%%%%%%%%%%%%%%%%%%%%%%%%%%%%%%%%%%%%%%%%%%%%%%%%%%%%%%%
\section{pRHL judgments}
\subsubsection*{equiv}
\begin{ecgrammar}

<inv_info> ::=
  `(' <exp> `)'          
 \alt `upto' `(' <exp> `)' [`and' `(' <exp> `)' ] [`with' `(' <exp> `)'] 


<auto_info> ::=
  [<inv_info>] [`using' <ident_list>]

 

<equiv_concl> ::=
      <exp> `==>' <exp>
 \alt <exp> `=' `(' <exp> `:' <exp> `)' `=>' <exp>
 \alt <inv_info> 

<equiv_elem> ::= 
      `equi' <ident> `:' <qualif_fct_name> `~' <qualif_fct_name> `:' <equiv_concl>
 \alt `equi' <ident> `:' <qualif_fct_name> `~' <qualif_fct_name> `:' <equiv_concl> `by' `auto' <auto_info> 
 \alt `equi' <ident> `:' <qualif_fct_name> `~' <qualif_fct_name> `:' <equiv_concl> `by' `eager' <block>

\end{ecgrammar}



%                                                               Element.tactics
%%%%%%%%%%%%%%%%%%%%%%%%%%%%%%%%%%%%%%%%%%%%%%%%%%%%%%%%%%%%%%%%%%%%%%%%%%%%%%%
\section{Tactics}
\begin{ecgrammar}
<interval> ::= `[' <number> `-' <number> `]' | <number> 

<rnd_info> ::= `(' <exp> `)' `,' `(' <exp> `)' | `(' <exp> `)' | `{' <number>`}'

<side_at_pos> ::= [`{'<number>`}'] [`at' <number_list> | `last']
 
<inline_info> ::= `at' <number_list> | `last' | <ident_list>

<tactic> ::= `idtac'
 \alt `call' <auto_info>
 \alt `inline' [`{'<number>`}'] [<inline_info>]
 \alt `asgn'
 \alt `rnd' [<rnd_info>]
 \alt `swap' [`{'<number>`}'] <interval> <znumber> 
 \alt `swap' [`{'<number>`}'] <znumber>  
 \alt `simpl'
 \alt `trivial'
 \alt `auto'  <auto_info>
 \alt `rauto' <auto_info>
 \alt `derandomize' [`{'<number>`}']
 \alt `wp'
 \alt `case'  [`{'<number>`}'] `:' <exp>
 \alt `if'    [`{'<number>`}']
 \alt `condt'   <side_at_pos> 
 \alt `condf'   <side_at_pos> 
 \alt `while' <side_at_pos> `:' <exp>
 \alt `while' <side_at_pos> `:' <exp> `:' <exp> `,' <exp> 
 \alt `while' <exp> `,' <exp> `,' <exp>  `,' <exp> `,' <exp> `:' <exp>     
 \alt `apply' <ident> `(' <exp_list0> `)'
 \alt `pRHL'
 \alt `apRHL'
 \alt `unroll'     <side_at_pos>
 \alt `strengthen' <side_at_pos>  `:' <exp>
 \alt `app' <number> <number> <exp>
 \alt `app' <number> <number> <exp> `:' <exp> `,' <exp> `:' <exp> `,' <exp> 
 \alt `try' <tactics_paren>
 \alt `*' <tactics_paren>
 \alt `!' <number> <tactics_paren>
 \alt `admit'
 \alt `expand' <ident_list0>
 \alt `let' <side_at_pos>  <ident> `:' <type> `=' <exp> 


<subgoal_tactics> ::= [<tactics>] `|' <subgoal_tactics> | [<tactics>]

<tactic2> ::= <tactic> | `[' <subgoal_tactics> `]' | `('<tactics>`)'

<tactic_list> ::= <tactic2> `;' <tactic_list> | <tactic2>

<tactics> ::= <tactic> `;' <tactic_list> | <tactic>

<tactics_paren> ::= <tactic> | `(' <tactics> `)'

\end{ecgrammar}

%                                                                 Element.claim
%%%%%%%%%%%%%%%%%%%%%%%%%%%%%%%%%%%%%%%%%%%%%%%%%%%%%%%%%%%%%%%%%%%%%%%%%%%%%%%
\section{Probability claims}
\subsubsection*{claim}
\begin{ecgrammar}
<claim_elem> ::=  
      `claim' <ident> `:' <exp>
 \alt `claim' <ident> `:' <exp> `admit'
 \alt `claim' <ident> `:' <exp> `compute'
 \alt `claim' <ident> `:' <exp> `split'
 \alt `claim' <ident> `:' <exp> `same'
 \alt `claim' <ident> `:' <exp> `using' <ident>
 \alt `claim' <ident> `:' <exp> `compute' <number> <exp> `,' <exp>
\end{ecgrammar}


%%%%%%%%%%%%%%%%%%%%%%%%%%%%%%%%%%%%%%%%%%%%%%%%%%%%%%%%%%%%%%%%%%%%%%%%%%%%%%%
%                                                                       Program
%%%%%%%%%%%%%%%%%%%%%%%%%%%%%%%%%%%%%%%%%%%%%%%%%%%%%%%%%%%%%%%%%%%%%%%%%%%%%%%
\subsection{Program}
\begin{ecgrammar}

<global_elem> ::=  `include' `"' <string> `"'          %{ Ginclude $2 } 
              \alt <type_elem>                         %{ Gtype $1 }
              \alt <cnst_elem>                         %{ Gcnst $1 }
              \alt <op_elem>                           %{ Gop (get_pos(),$1) }
              \alt <pop_elem>                          %{ Gpop $1 }
              \alt <pred_elem>                         %{ Gpred $1 }
              \alt <axiom_elem>                        %{ Gaxiom $1 }
              \alt <adv_elem>                          %{ Gadv $1 }
              \alt <game_elem>                         %{ Ggame $1 }
              \alt <equiv_elem>                        %{ Gequiv $1 }
              \alt <claim_elem>                        %{ Gclaim $1  }
              \alt <tactics>                           %{ Gtactic $1 }
              \alt `save'                              %{ Gsave }
              \alt `abort'                             %{ Gabort }
              \alt `set' <ident_list>                  %{ Gset $1 }
              \alt `unset' <ident_list>                %{ Gset $1 }
              \alt `prover' <prover_list>              %{ Gprover $2}
              \alt `checkproof'                        %{ Gwithproof }
              \alt `transparent' <ident_list>          %{ Gopacity(false,$2) }
              \alt `opaque' <ident_list>               %{ Gopacity(true,$2) }
              \alt `timeout' <number>                  %{ Gtimeout $2 }
              \alt `check' <check>                     %{ Gcheck $1 }
              \alt `print' <print>                     %{ Gprint $1 }

<program> ::=  <global_elem> `.'
          \alt <global_elem> `.' <program>

\end{ecgrammar} 



%%% Local Variables: 
%%% mode: latex
%%% TeX-master: "easycrypt"
%%% End: 


\iffalse
\part{Experimental Features}
  \chapter{Coq backend}
  \chapter{Approximate Probabilistic Relational Hoare Logic}
  There is preliminary support in EasyCrypt for handling an
  approximate variant of RHL, which can be used to reason about
  statistical distance and differential privacy. Note that some of the
  tactics described above have approximate variants that usually take
  extra arguments. For instance, the tactic 'app' has an approximate
  variant whose extra arguments are used to specify the 'skew' and the
  'slack'.

  \subsection{Tactic Support}
  The following commands can be used to switch between the approximate
  and exact variants of RHL during a proof:

  \begin{itemize}
  \item \verb+pRHL+

    Translates an approximate goal into the standard variant of RHL,
    when the skew is 1 and the slack is 0.

  \item \verb+apRHL+

    Translates a pRHL goal into its approximate form, with skew 1 and
    slack 0.
  \end{itemize}

  \section{Probabilistic operators and specs}



\subsection{Probabilistic operators}
Probabilistic operators are introduced with the following syntax:
\begin{verbatim}
pop gen_secret_key : unit -> secret_key.
pop encrypt : (plaintext, key) -> ciphertext
pop laplacian : (int, int, real) -> real.
\end{verbatim}

Probabilistic operators can be specified either by two-sided or one
sided rules. Two-sided rules adhere to the following syntax
\begin{verbatim}
spec lap_spec(v1:int,k:int,eps:real,v2:int) :
  x1=lap(v1,k,eps) ~ x2=lap(v2,k,eps):
  (v1-v2<=k && v2-v1<=k) ==[exp(eps);0%r]==> x1=x2.
\end{verbatim}
The skew \verb|exp(eps)| and the slack \verb|0%r| are optional.
We also support assert statements to specify the probabilistic
operator restricted to a condition on the sampled value:
\begin{verbatim}
spec choose_tu(g1:graph,g2:graph,n:int,i1:int,i2:int,eps:real) : 
  v1=choose(g1,eps,n,i1); assert (t=v1 || u=v1) ~ 
  v2=choose(g2,eps,n,i2); assert (t=v2 || u=v2) :
i1=i2 ==[exp(eps/4%r);0%r]==> v1=v2.
\end{verbatim}

One sided specifications are given using the following syntax:
\begin{verbatim}
type plaintext.
type key.
type ciphertext.

pop gen_secret_key :  unit -> key.
pop encrypt : (plaintext, key) -> ciphertext.
op decrypt : (ciphertext,key) -> plaintext.

aspec dec_spec(a:plaintext,k:key) : x = encrypt(a,k) : true ==> decrypt(x,k)=a.
\end{verbatim}

Both one-sided and two-sided specifications can be given for any
distribution expression, not only for those defined by a probabilistic
operator.


  \subsection{Tactic Support}
  \verb+apply[{1|2}]: <spec> (e1,..,e2)+

  Applies a probabilistic operator specification previously introduced
  using the \verb|spec| directive. Restrictions on the usage of
  \emph{side} parameters may apply due to the two-sided or one
  sided-nature of the specification. In addition to the optional side
  parameter, this tactic takes a list of arguments to instantiate the
  rule, and generate the corresponding verification conditions.
  
\todo{this can also be used outside of the app logic. Add examples of
  specs and usage for pRHL.}
\fi

% \clearpage
% \addcontentsline{toc}{chapter}{Index}
% \printindex




\end{document}
